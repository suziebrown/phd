\documentclass{article}
\usepackage[utf8]{inputenc}
\usepackage[margin=2.5cm]{geometry}

% bibliography
\usepackage[citestyle=authortitle-comp, backend=biber]{biblatex}
\addbibresource{../smc.bib}
\renewbibmacro*{cite:title}{%
  \printtext[bibhyperref]{%
    \printdate%
    \setunit{\addcomma\space}
    \printfield[citetitle]{labeltitle}%
    }}
    
% maths
\usepackage{amsmath}
\usepackage{amssymb}
\newcommand{\E}{\mathbb{E}}
\newcommand{\V}{\operatorname{Var}}

% metadata
\title{Annotated Bibliography}
\author{Suzie Brown}
\date{}


\begin{document}
\maketitle

%
\section*{SEQUENTIAL MONTE CARLO}

\subsection*{\cite{gordon1993}}
Original reference for SMC.

\subsection*{\cite{kitagawa1996}}
Nice introduction to SMC. Review of other nonlinear filtering techniques: extensions to Kalman filtering.

\subsection*{\cite{delmoral2013}}
Loads of rigorous results about SMC e.g.\ convergence, rates, CLTs.

\subsection*{\cite{doucet2009}}


\subsection*{\cite{andrieu2010}}
Introduces particle MCMC methods, including particle Gibbs with conditional SMC.

%
\section*{RESAMPLING}

\subsection*{\cite{kitagawa1996}}
Comparison of multinomial, stratified \& systematic resampling. And the effect of presorting. [in appendix]

\subsection*{\cite{douc2005}}
Comparison of Monte Carlo variance between mutli, res-multi, strat, syst. CLTs for resampled particles.

\subsection*{\cite{lee2019}}
Implementation of low variance resampling within conditional SMC.


\subsection*{\cite{fox2003}}


\subsection*{\cite{gandy2016}}


\subsection*{\cite{whitley1994}}


\subsection*{\cite{carpenter1999}}


\subsection*{\cite{gerber2017}}


\subsection*{\cite{delmoral2012}}




%
\section*{MATRIX RESAMPLING}

\subsection*{\cite{webber2019}}
\begin{itemize}
\item Allows number of particles to vary from one iteration to the next
\item Allows resampling such that the weights after resampling are unequal
\item Introduces the class of ``matrix resampling schemes'', and then the extension of this to encompass resampling schemes that cause the population size to vary randomly (but not unboundedly). 
\item To be representable by a matrix (i.e. in this class) a resampling scheme must satisfy the property that parental indices are conditionally independent given weights.
\item Includes NO RESAMPLING as an example resampling scheme! The reason being that it actually is included in the class of matrix resampling schemes
\item Also includes as an example the simple implementation of stochastic rounding, where you don't enforce constant $N$ so everything is just rounded independently (in this paper this is called ``Bernoulli resampling''). This is not in the class of matrix resampling schemes, but it is in the extended class (Defn 2.2).
\item It is shown that all matrix resampling schemes are unbiased
\item Under additional assumptions, it is also shown that such resampling schemes have properties: MC estimates converge; and the variance of these estimates is bounded above.
\item The above results hold even if the resampling scheme to use at each iteration is chosen on-the-fly --- notably including adaptive resampling, parallel resampling, adaptive pruning/enrichment, and you can imagine some other eccentric mixes of resampling schemes!
\item \emph{Complete} resampling schemes are defined as the subclass of matrix resampling schemes for which the weights after resampling are equal.
\item Three comparisons are made between resampling schemes, in terms of conditional variance: multi $\geq$ strat; multi $\geq$ res-multi; multi $\geq$ res-strat. None of these examples are novel results, but perhaps the proof technique for strat could be useful as (at first glance anyway) it seems more understandable than that of DCM05.
\item
\end{itemize}


\subsection*{\cite{li2020} (arXiv v2)}
\begin{itemize}
\item Resampling is not assumed to preserve the number of particles
\item Resampling schemes where the parental indices are conditionally independent (but not necessarily i.i.d.) --- notably multi, strat, res-multi and res-strat --- can be represented as a matrix conditional on the weights, where each element $P_{ij}$ is the conditional probability of particle $i$ choosing parent $j$ (so the rows sum to $1$, and column $j$ sums to $Nw_j$).
\item For multinomial resampling, all parental indices are conditionally i.i.d., so the rows of $P_{ij}$ are identical, each containing the vector of weights.
\item For stratified resampling, $P$ is a ``staircase matrix'': see Figure2(b) in the paper for an illustration. Upon thinking a little it is clear why you end up with a staircase matrix.
\item Residual resampling sets some rows to have a single $1$ element and the rest $0$ (these correspond to the deterministic assignments) and the remaining rows are constructed according to which scheme is used for the residuals, where the weights are now replaced with residual weights. See Fig2(c)(d) in the paper for an illustration
\item Equation (1) gives an explicit expression for the conditional variance induced by resampling, the same metric used in DCM05. This expression, in the case of stratified resampling, seems much nicer than the expression used in DCM05, which could provide a more elegant proof that multi dominates strat. (EDIT: this expression no longer appears in arxiv v2, so possibly it wasn't correct)
\item In Section 2.5 the authors admit that there is generally little to be gained by resampling such that the weights after resampling are unequal, even though I think it was one of the authors (Liu) who was proposing in previous works that this could be a useful strategy. (EDIT: this remark has been removed in v2)
\item The authors prove that ordered stratified resampling is optimal (in certain senses) among the class of resampling schemes considered. But their class only includes resampling schemes where the parental indices are conditionally independent. This excludes syst, res-syst, SSP,...
\item Theorem 3 tightens the convergence rate bound for Hilbert-ordered stratified resampling from $O(N^{-1-1/d})$ (in GCW19) to $O(N^{-1-2/d})$.
\item Proposition 2 verifies the conjecture of GCW19 that the Hilbert curve is an optimal way to order the particles in dimension $d>1$
\end{itemize}



%
\section*{PARALLEL RESAMPLING}
See also:
Verg\'e, Dubarry, Del Moral and Moulines (2013) ``On parallel implementation of sequential Monte Carlo methods: The island particle model''.

\subsection*{\cite{whiteley2016}}
\begin{itemize}
\item Introduces the $\alpha$SMC framework, a generalisation encompassing e.g.\ SIS, bootstrap PF, adaptive resampling.
\item Sections 3--4 present convergence results for generic $\alpha$SMC.
\item In Section 5.1 they use the $\alpha$SMC framework to demonstrate that running multiple independent PFs and then averaging them is not a good solution to distributed PFing. The multiple independent PFs can be expressed as an instance of $\alpha$SMC.
\item ``Degree of interaction'' is represented by graph degree: SIS corresponds to the identity matrix or graph with self-loops only (degree$=1$, the minimum possible); Bootstrap PF corresponds to the matrix with $1/N$ in every element, or the complete graph on $N$ vertices with self-loops (degree$=N$, the maximum possible); adaptive resampling chooses either the SIS or bootstrap graph at each iteration.
\item The structure of these graphs (see definition of "B-matrices" on p.514) is such that (1) each node has the same degree, (2) all self-loops are present, (3) all the connected components are complete subgraphs. These conditions are sufficient to ensure all the convergence properties of Theorem 2.
\item Algorithm 4 is a procedure for choosing the interaction matrix on-the-fly. By construction the resulting matrix is a B-matrix. The algorithm partitions $[N]$ into interaction blocks, i.e.\ the connected components of the corresponding graph. The sooner the while loop terminates, the smaller the connected components and the less interaction.
\item Section 5.4 contains simulation results. They demonstrate that the ``random'' and ``greedy'' variants of Algorithm 4 keep the level of interaction very low, the ``simple'' variant less so, but all much lower than the full interaction in the bootstrap PF.
\item They also compare the MSE of the various techniques (Figure 4), seeming to show that adaptive resampling outperforms the others as long as its threshold is not too high (e.g.\ the usual $N/2$ threshold is good in this example).
\item The authors' ``random'' and ``greedy'' methods do perform comparably to adaptive resampling though, and they seem to rarely require interaction between more than 2 or 4 particles at a time, so they might be more suitable for distributed settings?
\end{itemize}


\subsection*{\cite{lee2016}}
\begin{itemize}
\item This paper extends the work of WLH16 on $\alpha$SMC
\item The idea is to find practical $\alpha$SMC algorithms, where the on-the-fly choice of resampling type doesn't require too much particle interaction, so it really does work on-the-fly in parallel computing. In WLH16 they only considered removing interaction within the actual resampling step, not while choosing the resampler.
\item The permissable matrix/graph representations of resampling are limited to B-matrices as in WLH16, corresponding to \emph{disjoint unions of complete graphs}
\item They find that the partial ordering on such graphs (merging some connected components to create a coarser graph) corresponds to an ordering on the respective ESS's (using an extended defn of ESS that holds on restrictions of $[N]$ and may be weighted by some numbers $c$)
\item This suggests Algorithm 2, which recursively searches for a graph satisfying the convergence criterion of WLH16, coarsening the graph at each recursion
\item They then drill down into the implementation detail, managing somehow to be simultaneously nitty-gritty and abstract.
\item TBH I didn't understand a lot of this paper...
\end{itemize}


\subsection*{\cite{murray2016}}
\begin{itemize}
\item Motivates the need for parallelisable resampling schemes
\item Introduces \emph{Metropolis resampling}, a Metropolis approximation of Multinomial resampling, i.e.\ run a Markov chain targetting the Categorical distribution parametrised by the normalised weights. 
\item The Metropolis resampler is parallelisable because it doesn't require the weights to be normalised (i.e.\ their sum calculated) and it only compares weights pairwise (for the accept/reject step), so no operations require access to all the particles at once.
\item The Metropolis resampler is biased, because the Markov chain is only run for a finite number of iterations, so it leads to biased likelihood estimates (so it's no use for pseudo-marginal-type PMCMC).
\item The complexity of the Metropolis resampler is $O(NB)$ where $B$ is the number of Metropolis steps used per iteration, which may scale as a function of $N$. $B$ is a tuning parameter, trading off speed vs. accuracy; the authors provide some guidelines for choosing its value, which require that an upper bound on weights is available.
\item Introduces \emph{rejection resampling}, which uses rejection sampling to choose the parents. Only possible when an upper bound on the weights is known. Two variants: the initial proposal is uniform over $[N]$, or is deterministically equal to the offspring index.
\item Runtime of rejection resampling for each particle is random --- it keeps proposing until it gets an acceptance --- so it's not as parallel as Metropolis resampling; some particles will take longer to resample and leave other processors idling.
\item Rejection resampling is unbiased.
\item If the deterministic first proposal is used, rejection sampling has lower variance than Multinomial resampling, although this gain decreases as the variance of weights increases. With the Uniform first proposal, it samples the parents exactly from the Categorical distribution.
\item \emph{Partial rejection resampling} can be used to reduce the expected runtime of rejection sampling, or in cases where no upper bound on weights is available but some almost-upper-bound $V$ is known. Instead of resampling with the weights $w_i$, resample with $w_i \wedge V$, and remove the bias by setting the weights after resampling to the ratio of the true weight and the resampling weight. This means any particularly high-weight particles for which $w_i > V$ do not have their weights completely reset, just decreased by a factor of $V$ (before normalisation: recall that parallel resampling avoids normalising weights). Taking $V$ small reduces the runtime but also reduces the effectiveness of resampling by resetting fewer weights.
\item Lots of simulation results showing how performance compares between multinomial, stratified, systematic, Metropolis and rejection resampling, and their dependence on the number of particles and variance of weights. In their example a tight analytic upper bound on weights is available; I don't know how realistic this is in general.
\end{itemize}



%
\section*{BACKWARD SIMULATION}

\subsection*{\cite{kitagawa1996}}
Some solutions to ancestral degeneracy: fixed lag smoother, forward-backward-type algorithm.

\subsection*{\cite{doucet2009}}


\subsection*{\cite{lindsten2013}}
A whole book on backward simulation. [Chapter 5] describes backward simulation and ancestor sampling in particle MCMC.

\subsection*{\cite{whiteley2010}}
In the discussion, Nick Whiteley introduces (remarkably briefly) the idea of ancestor sampling in particle Gibbs.





\section*{CONVERGENCE OF GENEALOGIES}

Also consider looking at: 
Tavar\'e 1984 ``Line-of-descent and genealogical processes, and their applications in population genetics models''; 
Donnelly 1991 ``Weak convergence to  a Markov chain with an entrance boundary: ancestral processes in population genetics''; 
Donnelly Tavar\'e 1995 ``Coalescents and genealogical structure under neutrality''; 
Griffiths Tavar\'e 1994 ``Sampling theory for neutral alleles in a varying environment''; 
Marjoram 1992 ``Correlation structures in applied probability'' (PhD thesis, UCL);
Schweinsberg 2003 ``Coalescent processes obtained from supercritical Galton-Watson processes''

\subsection*{\cite{kingman1982gene}}
\begin{itemize}
\item Introduces the $n$-coalescent (in a very nice clear way) with the same notation we still use
\item \textbf{Theorem:} suppose $\nu_{1:N}$ are exchangeable and independent across generations and $\V[\nu_1]\to\sigma^2 \in (0,\infty)$ and $\E[\nu_1^m] \leq M_m$ for all $m\in\mathbb{N}$. Then the $n$-genealogies scaled by $\lfloor N\sigma^{-2}t \rfloor$ converge to the $n$-coalescent in the sense of FDDs.
\item $n$-coalescent also applies for models where $\nu_j$ are not exchangeable or independent across generations, as long as the genealogies are Markov at least up to error $O(N^{-1})$ and the transitions satisfy $p_{\xi\eta} = q_{\xi\eta}\sigma^2 N^{-1} + o(N^{-1})$, where $q$'s are transition probs of $n$-coalescent
\item The $n$-coalescent is a good robust model for \emph{large neutral} populations
\item Genealogy decouples into a jump chain and a pure death process
\item There exists the Kingman coalescent as infinite-dimensional embedding of the $n$-coalescents
\end{itemize}


\subsection*{\cite{kingman1982coal}}
Broadly, this paper introduces the Kingman coalescent (as opposed to $n$-coalescent) and proves some properties


\subsection*{\cite{tavare1984}}
\begin{itemize}
\item Review article on genealogical models and their application to population genetics
\end{itemize}


\subsection*{\cite{mohle1998}}
Necessary \& sufficient conditions for convergence of Cannings model to a coalescent process more general than Kingman. Allowing large mergers but not simultaneous mergers.
\begin{itemize}
\item Population size can vary over time, but only deterministically
\item Individuals are not exchangeable, just assume the random assignment condition
\item Offspring counts must be independent but not necessarily identically distributed across generations
\item This means time scale must be allowed to vary over time: $\tau_N$ becomes $\tau_N(t)$ and $c_N$ becomes $c_N(t)$, or $c(t)$ in M\"ohle's notation
\item Only proves convergence of FDDs
\item Conditions of theorem are still very strong, requiring infinitely many moments to be bounded. There is now a condition on one mixed moment that wasn't needed in Kingman1982.
\item The conditions are sufficient but not necessary
\item Time scale $\tau_N(t)$ is allowed to be chosen freely, but the theorem only holds when it is an appropriate function (i.e.\ an inverse of $c_N$ similar to the usual) so this is not a very great generalisation over defining $\tau$ in the usual way.
\end{itemize}


\subsection*{\cite{mohlesagitov1998}}
\begin{itemize}
\item Assume $\nu_{1:N}$ are exchangeable, and i.i.d. across generations, and the population size $N$ is constant
\item Necessary \& sufficient conditions are given for (FDD) convergence of the genealogies to a $\Lambda$-coalescent, and the correct measure $\Lambda$ is uniquely constructed from infinitely many moment limits. (Note: M\"ohle's notation uses $\mu$ for the measure rather than $\Lambda$.)
\item The conditions are: (I) infinitely many pure moment limits exist, slightly different from the condition 2a of M\"ohle1998; (II) the exchangeable version of the mixed moment condition 2b of M\"ohle1998.
\item Under the additional condition $c_N\to0$ we also get weak convergence to the $\Lambda$-coalescent, although the proof is not explicit in this paper (it just refers to the methods of another work).
\end{itemize}


\subsection*{\cite{sagitov1999}}
\begin{itemize}
\item Assume $\nu_{1:N}$ are exchangeable, and i.i.d. across generations, and the population size $N$ is constant
\item $k$-mergers (but not simultaneous mergers) are allowed
\item 3 necessary conditions are given for FDD convergence to some limit of the form $p_{\xi\eta} = \delta_{\xi\eta} + V_N q_{\xi\eta} + o(V_N)$, where $Q=(q_{\xi\eta})$ is some Markov generator, and $V_N \to 0$.
\item Additionally, a subset of the necessary conditions are shown to be sufficient for the above asymptotic relation to hold with specific $Q$ corresponding to a $\Lambda$-coalescent, on the (constant, deterministic) time scale $T_N^{-1} \sim V_N$.
\item Unfortunately I didn't really understand the second and third conditions...
\item As one would expect, the necessary conditions can be shown to hold when Kingman's condition $\sup_N \E[\nu_1^k]<\infty \forall k\geq 2$ applies (see Remark 1)
\end{itemize}


\subsection*{\cite{pitman1999}}
\begin{itemize}
\item Defines $\Lambda$-coalescents in the notation still used, proves its existence.
\item Notes how Kingman coalescent and Bolthausen-Sznitman coalescent (indeed the whole class of $\beta$-coalescents) can be seen as special cases
\item Properties: exchangeable random partition; Exponential waiting times with parameter determined by $\Lambda$; criterion for coming down from infinity; restriction to $[n]$
\item This paper studies the $\Lambda$-coalescents from the point of view of coalescent theory / exchangeable random partitions. It doesn't make any reference to population dynamics that might lead to such a limit. That angle was covered in Sagitov1999 (who independently discovered the $\Lambda$-coalescents).
\end{itemize}


\subsection*{\cite{mohle1999}}
\begin{itemize}
\item Population size can vary over time, but only deterministically
\item Offspring counts must be independent but not necessarily identically distributed across generations --- this ensures the genealogical process is Markovian, but not time-homogeneous
\item Individuals are not necessarily exchangeable, but our standing assumption (here called the ``random assignment condition'') is assumed
\item Time scale $\tau_N(t)$ is allowed to be chosen freely, but the theorem only holds when it is an appropriate function (i.e.\ an inverse of $c_N$ similar to the usual) so this is not a very great generalisation over defining $\tau$ in the usual way.
\item Theorem conditions are exactly the same as in M\"ohle1998
\item Convergence is now proved additionally in the weak sense, as opposed to just FDDs
\item Proof technique is overall the one I used for weak convergence of SMC genealogies, except M\"ohle's doesn't have the external expectations. These expectations appear in my proof because the coupled process can only be defined conditionally on $\mathcal{F}_\infty$, because the generations are not independent
\end{itemize}


\subsection*{\cite{mohle2000}}
\begin{itemize}
\item Fixed population size $N$
\item Offspring counts are exchangeable and i.i.d. across generations
\item Treats full characterisation of possible limiting coalescents (I think the class is known as $\Xi$-coalescents), where there are possibly simultaneous and/or multiple mergers. The particular limiting coalescent can be characterised by (mixed) factorial moments, here denoted $\Phi_a(b_1,\dots,b_a)$. These quantities are central to the analysis, and results are given in terms of them, so can be applied to any exchangeable model.
\item It is shown that the Kingman coalescent appears as the limit if and only if $ \lim \frac{\E[(\nu_1)_3]}{N\E[(\nu_1)_2]} =0$. (See equation (14) or (16).) This is the exchangeable version of the main theorem condition I use in my SMC genealogies work. I think this work is the earliest use of this simplified condition.
\item Total variation bounds are proved in the general case, left in terms of the particular transition probabilities of the model (see Thm 1).
\item More specific bounds are calculated for the case where the model is in the domain of attraction of Kingman's coalescent (i.e. second moments dominate third moments, as above / in equation (14)). The bounds have a simple form and are left in terms of $c_N$ and some $\Phi$ functions which correspond to $\E[(\nu_1)_3]$ and $\E[(\nu_1)_2(\nu_2)_2]$, which could easily be computed for a given model. It is clear that under the conditions of convergence to KC, the given TV bound converges to zero, as expected.
\end{itemize}


\subsection*{\cite{mohle2001}}
\begin{itemize}
\item Constant population size $N$
\item Offspring counts are exchangeable within generations and i.i.d. between generations --- implies the genealogical process is a time-homogeneous Markov chain
\item Treats full variety of possible limits for exchangeable models (again, I think the family is $\Xi$-coalescents)
\item Presents necessary and sufficient conditions for weak convergence to an appropriate $\Xi$-coalescent, and the particular $\Xi$ is uniquely determined by moments of the offspring distribution
\item A proof of FDD convergence is given. When $c_N\to c\neq0$ the limiting process is discrete and weak convergence follows immediately from FDDs. When $c_N\to0$ the limit is continuous and tightness is needed for weak comvergence; the proof is not presented, just a reference to M\"ohle1999.
\item Since the conditions given are necessary and sufficient, and can be applied to any exchangeable (Cannings-type) model, this constitutes a complete characterisation of the limiting genealogies for this class of models.
\item Section 6 gives a cute bit of history of coalescent theory
\end{itemize}


\subsection*{\cite{mohle2002coal}}


\subsection*{\cite{mohle2003}}


\subsection*{\cite{schweinsberg2003}}





\section*{SMC GENEALOGIES}

\subsection*{\cite{jacob2015}}
Description of ancestries as trunk+crown. Upper bound on storage cost via an approximate multinomial resampling scheme that is independent of weights. Numerical simulations suggesting similar results for stratified and systematic resampling (including an ordering on the schemes?). 

\subsection*{\cite{koskela2018}}


%
\section*{VARIANCE ESTIMATION}

\subsection*{\cite{chan2013}}


\subsection*{\cite{lee2018}}


\subsection*{\cite{olsson2019}}


%

\end{document}