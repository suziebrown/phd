\documentclass[fleqn]{article}
\usepackage[margin=2.5cm]{geometry}

% custom header/footer
\usepackage{fancyhdr}
\pagestyle{fancy}
\renewcommand{\headrulewidth}{0pt}
\fancyhf{}
\rfoot{\textsf{\thepage}}
\lfoot{\textsf{Suzie Brown}}

% bibliography
\usepackage[round, sort&compress]{natbib}
\usepackage{har2nat}
\bibliographystyle{agsm}

\usepackage{amsmath}
\usepackage{amssymb}
\usepackage{bbm}
\usepackage{amsthm}
\newtheorem{thm}{Theorem}
\newtheorem{lemma}{Lemma}
\newtheorem{remark}{Remark}
\newtheorem{corollary}{Corollary}
\newtheorem{conj}{Conjecture}
\newtheorem{prop}{Proposition}
\theoremstyle{definition}
\newtheorem{defn}{Definition}

% useful math symbols
\newcommand{\PR}{\mathbb{P}}
\newcommand{\E}{\mathbb{E}}
\newcommand{\V}{\operatorname{Var}}
\newcommand{\eqdist}{\overset{d}{=}}
\newcommand{\I}[1]{\mathbb{I}\{#1\}}
\newcommand{\1}[1]{\mathbbm{1}_{\{#1\}}}
\newcommand{\limNtoinfty}{\underset{N\to\infty}{\lim}}

% distributions
\newcommand{\Cat}{\operatorname{Categorical}}
\newcommand{\Unif}{\operatorname{Uniform}}
\newcommand{\Mn}{\operatorname{Multinomial}}
\newcommand{\Bin}{\operatorname{Binomial}}

% project-specific commands
\newcommand{\F}{\mathcal{F}_{t-1}}
\newcommand{\vt}[2][t]{v_{#1}^{(#2)}}
\newcommand{\wt}[2][t]{w_{#1}^{(#2)}}
\newcommand{\wbar}[2][t]{\bar{w}_{#1}^{(#2)}}
\newcommand{\vttilde}[2][t]{\tilde{v}_{#1}^{(#2)}}
\newcommand{\flnw}{\lfloor N\wt{i} \rfloor }

\title{Corollaries 1--3 proofs with details...}
\author{Suzie Brown}
\date{\today}

\begin{document}
\maketitle
\thispagestyle{fancy}

\begin{defn}
A function $f$ is said to be $i$-increasing if it is an increasing function in $\vt{i} = |\{j : a_t^{(j)} = i \}|$.
\end{defn}

\section*{Multinomial resampling}

\begin{corollary}\label{thm:mn_newassns}
Under the time scaling \eqref{eq:tauN}, supposing there exist constants $0<\varepsilon\leq 1\leq a<\infty$ such that
\begin{align}
\frac{1}{a} \leq &g_t(x, x^\prime) \leq a \label{eq:bounded_g}\\
\varepsilon h(x^\prime) \leq &q_t(x, x^\prime) \leq \frac{1}{\varepsilon} h(x^\prime) ,\label{eq:bounded_q}
\end{align}
genealogies of SMC algorithms with multinomial resampling converge to Kingman's $n$-coalescent in the sense of finite-dimensional distributions as $N\to\infty$.
\end{corollary}

\begin{lemma}\label{lem:i_increasing}
Let $a_t^{(i)}$ be the parental indices from a SMC algorithm with multinomial resampling. For any function $f$ that is $i$-increasing, 
\begin{align*}
& \E[f(\mathbf{a}_t) \mid \mathcal{H}_t] \leq \E[f(\mathbf{A}_1)] \\
& \E[f(\mathbf{a}_t) \mid \mathcal{H}_t] \geq \E[f(\mathbf{A}_2)]
\end{align*}
where the elements of $\mathbf{A}_1, \mathbf{A}_2$ are all mutually independent and independent of $\mathcal{F}_{\infty}$, and distributed according to
\begin{align*}
& A_1^{(j)} \sim \Cat\left( \left( \frac{a}{\varepsilon} \right)^{\1{i=1} -\1{i\neq 1}} ,\dots, \left( \frac{a}{\varepsilon} \right)^{\1{i=N} -\1{i\neq N}} \right) \\
& A_2^{(j)} \sim \Cat\left( \left( \frac{\varepsilon}{a} \right)^{\1{i=1} -\1{i\neq 1}} ,\dots, \left( \frac{\varepsilon}{a} \right)^{\1{i=N} -\1{i\neq N}} \right),
\end{align*}
where the arguments of Categorical and Multinomial distributions are given up to a normalising constant here and throughout this document.
\end{lemma}
\begin{proof}
The result follows using the bounds given in equations \eqref{eq:bounded_g}, \eqref{eq:bounded_q} with a balls-in-bins coupling, and cancelling $h$ from the top and bottom.
\end{proof}

Define the corresponding ``family sizes'' $V_1^{(i)} := |\{j: A_1^{(j)}=i\}|$ and $V_2^{(i)} := |\{j: A_2^{(j)}=i\}|$ for $i=1,\dots,N$. The distributions of $\mathbf{A}_1, \mathbf{A}_2$ imply the following:
\begin{align*}
& \mathbf{V}_1 \sim \Mn\left(N, \left( \frac{a}{\varepsilon} \right)^{\1{i=1} -\1{i\neq 1}} ,\dots, \left( \frac{a}{\varepsilon} \right)^{\1{i=N} -\1{i\neq N}} \right) \\
& \mathbf{V}_2 \sim \Mn\left(N, \left( \frac{\varepsilon}{a} \right)^{\1{i=1} -\1{i\neq 1}} ,\dots, \left( \frac{\varepsilon}{a} \right)^{\1{i=N} -\1{i\neq N}} \right),
\end{align*}

Notice that the function $f_i(\mathbf{a}_t) := (\vt{i})_2$ is $i$-increasing for each $i=1,\dots,N$. Applying Lemma \ref{lem:i_increasing} and the Multinomial moments formula \citep{mosimann1962}, we obtain the following lower bound:
\begin{align*}
\E_t[f_i(\mathbf{a}_t)] &\geq \E[f_i(\mathbf{A}_2)] = \E[(V_2^{(i)})_2] \\
&= \frac{(N)_2 (\varepsilon/a)^2}{[(\varepsilon/a) + (N-1)(a/\varepsilon)]^2}
\geq \frac{(N)_2 (\varepsilon/a)^2}{N^2(a/\varepsilon)^2}
= \frac{(N)_2}{N^2}\frac{\varepsilon^4}{a^4}.
\end{align*}
So we can lower bound the denominator by
\begin{equation*}
\E_t[c_N(t)] = \frac{1}{(N)_2} \sum_{i=1}^N \E_t[(\vt{i})_2]
\geq \frac{N}{(N)_2} \frac{(N)_2}{N^2}\frac{\varepsilon^4}{a^4}
= \frac{\varepsilon^4}{Na^4}.
\end{equation*}

To upper bound the numerator, consider the function $f_i(\mathbf{a}_t) := (\vt{i})_3$, which is $i$-increasing for each $i=1,\dots,N$.
Again using Lemma \ref{lem:i_increasing} and \citep{mosimann1962}, we obtain the following lower bound:
\begin{align*}
\E_t[f_i(\mathbf{a}_t)] &\leq \E[f_i(\mathbf{A}_1)] = \E[(V_1^{(i)})_3] \\
&= \frac{(N)_3 (a/\varepsilon)^3}{[(a/\varepsilon) + (N-1)(\varepsilon/a)]^3}
\geq \frac{(N)_3 (a/\varepsilon)^3}{N^3(\varepsilon/a)^3}
= \frac{(N)_3}{N^3}\frac{a^6}{\varepsilon^6}.
\end{align*}
and the numerator is therefore bounded above by
\begin{equation*}
\frac{1}{(N)_3} \sum_{i=1}^N \E_t[(\vt{i})_3]
\leq \frac{N}{(N)_3} \frac{(N)_3}{N^3}\frac{\varepsilon^6}{a^6}
= \frac{\varepsilon^6}{N^2a^6}.
\end{equation*}

The ratio is therefore bounded above by
\begin{equation*}
\frac{\frac{1}{(N)_3} \sum_{i=1}^N \E_t[(\vt{i})_3]}{\frac{1}{(N)_2} \sum_{i=1}^N \E_t[(\vt{i})_2]}
\leq \frac{N}{(N)_3} \frac{(N)_3}{N^3}\frac{\varepsilon^6}{a^6}
= \frac{\varepsilon^6}{N^2a^6} \frac{Na^4}{\varepsilon^4} = \frac{a^{10}}{N\varepsilon^{10}} =: b_N \limNtoinfty 0.
\end{equation*}

We can thus conclude the proof of Corollary \ref{thm:mn_newassns} by applying Theorem 1.


\section*{Conditional SMC with multinomial resampling}
We can apply the same technique to tackle conditional SMC, but it requires an adjustment of the bounding distributions.
We assume wlog that the immortal particle always takes index 1.

\begin{lemma}\label{lem:i_increasing_csmc}
Let $a_t^{(i)}$ be the parental indices from a  conditional SMC algorithm with multinomial resampling. For any function $f$ that is $i$-increasing, 
\begin{align*}
& \E[f(\mathbf{a}_t) \mid \mathcal{H}_t] \leq \E[f(\mathbf{A}_1)] \\
& \E[f(\mathbf{a}_t) \mid \mathcal{H}_t] \geq \E[f(\mathbf{A}_2)]
\end{align*}
where the elements of $\mathbf{A}_1, \mathbf{A}_2$ are all mutually independent and independent of $\mathcal{F}_{\infty}$, and distributed according to
\begin{align*}
& A_1^{(j)} \sim \begin{cases}
& \delta_1  \qquad j=1\\
& \Cat\left( \left( \frac{a}{\varepsilon} \right)^{\1{i=1} -\1{i\neq 1}} ,\dots, \left( \frac{a}{\varepsilon} \right)^{\1{i=N} -\1{i\neq N}} \right) \qquad j\neq 1
 \end{cases}\\
& A_2^{(j)} \sim \begin{cases}
& \delta_1 \qquad j=1 \\
& \Cat\left( \left( \frac{\varepsilon}{a} \right)^{\1{i=1} -\1{i\neq 1}} ,\dots, \left( \frac{\varepsilon}{a} \right)^{\1{i=N} -\1{i\neq N}} \right) \qquad j\neq 1 .
\end{cases}
\end{align*}
\end{lemma}

As before, we can define the corresponding ``family sizes'' $V_1^{(i)} := |\{j: A_1^{(j)}=i\}|$ and $V_2^{(i)} := |\{j: A_2^{(j)}=i\}|$ for $i=1,\dots,N$. They now have the following distributions:
\begin{align*}
& \mathbf{V}_1 \eqdist (1,0,\dots,0) + \Mn\left(N-1, \left( \frac{a}{\varepsilon} \right)^{\1{i=1} -\1{i\neq 1}} ,\dots, \left( \frac{a}{\varepsilon} \right)^{\1{i=N} -\1{i\neq N}} \right) \\
& \mathbf{V}_2 \eqdist (1,0,\dots, 0) + \Mn\left(N-1, \left( \frac{\varepsilon}{a} \right)^{\1{i=1} -\1{i\neq 1}} ,\dots, \left( \frac{\varepsilon}{a} \right)^{\1{i=N} -\1{i\neq N}} \right) .
\end{align*}

Now consider again the $i$-increasing function $f_i(\mathbf{a}_t) := (\vt{i})_2$. In the conditional SMC case, we can apply Lemma \ref{lem:i_increasing_csmc} to obtain the lower bound
\begin{equation*}
\frac{1}{(N)_2} \sum_{i=1}^N \E_t[(\vt{i})_2] \geq \frac{1}{(N)_2} \sum_{i=1}^N \E[(V_2^{(i)})_2]
= \frac{1}{(N)_2} \left[ \sum_{i=1}^N \frac{(N-1)_2 (\varepsilon/a)^2}{[(\varepsilon/a) + (N-2)(a/\varepsilon)]^2} + 2 \frac{(N-1)(\varepsilon/a)}{(\varepsilon/a) + (N-2)(a/\varepsilon)} \right]
\end{equation*}
using the Multinomial moments as before, along with the identity $(X+1)_2 \equiv (X)_2 + 2(X)_1$.
This is further bounded by
\begin{equation*}
\frac{1}{(N)_2} \sum_{i=1}^N \E_t[(\vt{i})_2] 
\geq \frac{1}{(N)_2} \left[ \frac{(N)_3 (\varepsilon/a)^2}{(N-1)^2(a/\varepsilon)^2} + \frac{2(N-1)(\varepsilon/a)}{(N-1)(a/\varepsilon)} \right]
= \frac{1}{(N)_2} \left[ \frac{(N)_3}{(N-1)^2}\frac{\varepsilon^4}{a^4} + \frac{2\varepsilon^2}{a^2} \right] 
\end{equation*}

Similarly, we derive an upper bound on $f_i(\mathbf{a}_t) := (\vt{i})_3$, yielding
\begin{align*}
\frac{1}{(N)_3} \sum_{i=1}^N \E_t[(\vt{i})_3] 
&\leq \frac{1}{(N)_3} \left[ \sum_{i=1}^N \frac{(N-1)_3 (a/\varepsilon)^3}{[(a/\varepsilon) + (N-2)(\varepsilon/a)]^3} + 3 \frac{(N-1)_2 (a/\varepsilon)^2}{[(a/\varepsilon) + (N-2)(\varepsilon/a)]^2} \right] \\
&\leq \frac{1}{(N)_3} \left[ ... \right]
\end{align*}





\bibliography{../../smc.bib}
\end{document}