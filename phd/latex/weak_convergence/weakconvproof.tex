\documentclass{article}
\usepackage[utf8]{inputenc}
\usepackage[margin=2cm]{geometry}

\usepackage{graphicx}
\usepackage{enumitem}

% custom header/footer
\usepackage{fancyhdr}
\pagestyle{fancy}
\renewcommand{\headrulewidth}{0pt}
\fancyhf{}
\rfoot{\textsf{\thepage}}
\lfoot{\textsf{Suzie Brown}}

%annotations
\usepackage{color}
\usepackage{xspace}
\newcommand{\seb}[1]{\xspace\textcolor{red}{#1}\xspace}

% bibliography
\usepackage[round, sort&compress]{natbib}
\usepackage{har2nat}
\bibliographystyle{agsm}

% maths
\usepackage{amsmath}
\usepackage{amssymb}
\usepackage{amsthm}
\newtheorem{theorem}{Theorem}
\newtheorem{lemma}{Lemma}
\newcommand{\Prob}{\mathbb{P}}
\newcommand{\E}{\mathbb{E}}
\newcommand{\Et}{\mathbb{E}_t}
\newcommand{\V}{\operatorname{Var}}
\newcommand{\I}[1]{\mathbb{I}_{\{#1\}}}
\newcommand{\1}[1]{\mathbb{I}_{#1}}
\newcommand{\Mn}{\operatorname{Multinomial}}

\title{Weak convergence proof (in progress)}
\author{Suzie Brown}

\begin{document}
\maketitle
\thispagestyle{fancy}

\begin{theorem}\label{thm:weakconv}
Let $\nu_t^{(1:N)}$ denote the offspring numbers in an interacting particle system satisfying the standing assumption and such that, for any $N$ sufficiently large, $\Prob\{ \tau_N(t) = \infty \} =0$ for all finite $t$. Suppose that there exists a deterministic sequence $(b_N)_{N\geq1}$ such that ${\lim}_{N\to\infty} b_N =0$ and
\begin{equation}\label{eq:mainthmcond}
\frac{1}{(N)_3} \sum_{i = 1}^N \Et\{ (\nu_t^{(i)})_3 \}  \leq b_N \frac{1}{(N)_2} \sum_{i = 1}^N \Et\{ (\nu_t^{(i)})_2 \}
\end{equation}
for all $N$, uniformly in $t \geq 1$.
Then the rescaled genealogical process $(G_{\tau_N(t)}^{(n,N)})_{t\geq0}$ converges weakly to Kingman's $n$-coalescent as $N \to \infty$.
\end{theorem}

\begin{proof}
Define $p_t := \max_{\xi\in E} \{1 - p_{\xi\xi}(t)\} = 1 - p_{\Delta\Delta}(t)$, where $\Delta$ denotes the trivial partition of $\{1,\dots,n\}$ into singletons. For a proof that the maximum is attained at $\xi = \Delta$, see Lemma \ref{thm:maximum_pr}. 
Following \citet{mohle1999}, we now construct the two-dimensional Markov process $(Z_t, S_t)_{t \in \mathbb{N}}$ with transition probabilities
\begin{equation}
\Prob(Z_t = j , S_t = \eta \mid Z_{t-1} = i, S_{t-1} = \xi)
= \begin{cases}
1 - p_t &\quad \text{if } j=i \text{ and } \xi=\eta \\
p_{\xi\xi}(t) + p_t - 1  &\quad \text{if } j=i+1 \text{ and } \xi=\eta \\
p_{\xi\eta}(t) &\quad \text{if } j=i+1 \text{ and } \xi\neq\eta \\
0 &\quad \text{otherwise} .
\end{cases}
\end{equation}
The construction is such that the marginal $(S_t)$ has the same distribution as the genealogical process of interest, and $(Z_t)$ has jumps at all the times $(S_t)$ does plus some extra jumps. (The definition of $p_t$ ensures that the probability in the second case is non-negative, attaining the value zero when $\xi=\Delta$.)

Denote by $0=T_0^{(N)}<T_1^{(N)}<\dots$ the jump times of the rescaled process $(Z_{\tau_N(t)})_{t\geq0}$, and $\omega_i^{(N)} := T_i^{(N)} - T_{i-1}^{(N)}$ the corresponding holding times ($i\in\mathbb{N}$).

...
\end{proof}

\begin{lemma}\label{thm:maximum_pr}
$\max_{\xi\in E} (1 - p_{\xi\xi}(t)) = 1 - p_{\Delta\Delta}(t)$.
\end{lemma}
%%% This below proof is incorrect.
%\begin{proof}
%From the definition of $p_{\xi\eta}(t)$ \citep[Equation (1)]{koskela2018},
%\begin{equation}
%p_{\xi\xi}(t) = \frac{1}{(N)_{|\xi|}} \sum_{\substack{i_1,\dots,i_{|\xi|} \\ \text{all distinct}}} \nu_t^{(i_1)} \cdots \nu_t^{(i_{|\xi|})} .
%\end{equation}
%In the case $\xi = \Delta$, this simplifies to
%\begin{equation}
%p_{\Delta\Delta}(t) = \prod_{i=1}^N \nu_t^{(i)} 
%= \begin{cases}
%1 &\quad \text{if } \nu_t^{(1)}=\cdots=\nu_t^{(n)} \\
%0 &\quad \text{otherwise} .
%\end{cases}
%\end{equation}
%Whenever $p_{\Delta\Delta}(t) = 1$, $p_{\xi\xi}(t) = 1$ for all $\xi$ also. 
%Otherwise, $p_{\xi\xi}(t) \geq p_{\Delta\Delta}(t) = 0$ for all $\xi$, since $p_{\xi\xi}(t) \in [0,1]$.
%Hence $p_{\xi\xi}(t)$ attains its minimum at $\xi=\Delta$, and the result follows.
%\end{proof}
\begin{proof}
Consider any $\xi \in E$ consisting of $k$ blocks ($1\leq k\leq n-1$), and any $\xi^\prime\in E$ consisting of $k+1$ blocks. 
From the definition of $p_{\xi\eta}(t)$ \citep[Equation (1)]{koskela2018},
\begin{equation}
p_{\xi\xi}(t) = \frac{1}{(N)_k} \sum_{\substack{i_1,\dots,i_k \\ \text{all distinct}}} \nu_t^{(i_1)} \cdots \nu_t^{(i_k)} .
\end{equation}
Similarly,
\begin{align*}
p_{\xi^\prime\xi^\prime}(t) &= \frac{1}{(N)_{k+1}} \sum_{\substack{i_1,\dots,i_k, i_{k+1} \\ \text{all distinct}}} \nu_t^{(i_1)} \cdots \nu_t^{(i_k)} \nu_t^{(i_{k+1})} \\
&= \frac{1}{(N)_k(N-k)} \sum_{\substack{i_1,\dots,i_k \\ \text{all distinct}}} \left\{ \nu_t^{(i_1)} \cdots \nu_t^{(i_k)} \sum_{\substack{i_{k+1}=1 \\ \text{also distinct}}}^N \nu_t^{(i_{k+1})} \right\} .
\end{align*}
Discarding the zero summands,
\begin{equation}
p_{\xi^\prime\xi^\prime}(t) = \frac{1}{(N)_k(N-k)} \sum_{\substack{i_1,\dots,i_k \\ \text{all distinct:} \\ \nu_t^{(i_1)},\dots,\nu_t^{(i_k)} > 0 }} \left\{ \nu_t^{(i_1)} \cdots \nu_t^{(i_k)} \sum_{\substack{i_{k+1}=1 \\ \text{also distinct}}}^N \nu_t^{(i_{k+1})} \right\} .
\end{equation}
The inner sum is
\begin{equation}
\sum_{\substack{i_{k+1}=1 \\ \text{also distinct}}}^N \nu_t^{(i_{k+1})} =
\left\{ \sum_{i=1}^N \nu_t^{(i)} -  \sum_{i\in\{i_1,\dots,i_k\} } \nu_t^{(i)} \right\}
\leq N - k
\end{equation}
since $\nu_t^{(i_1)},\dots,\nu_t^{(i_k)} $ are all at least 1.
Hence
\begin{equation}
p_{\xi^\prime\xi^\prime}(t)
\leq  \frac{N-k}{(N)_k(N-k)} \sum_{\substack{i_1,\dots,i_k \\ \text{all distinct:} \\ \nu_t^{(i_1)},\dots,\nu_t^{(i_k)} > 0 }} \nu_t^{(i_1)} \cdots \nu_t^{(i_k)} 
= p_{\xi\xi}(t) .
\end{equation}
Thus $p_{\xi\xi}(t)$ is decreasing in the number of blocks of $\xi$, and is therefore minimised by taking $\xi = \Delta$, which achieves the maximum $n$ blocks. This choice in turn maximises $1-p_{\xi\xi}(t)$, as required.
\end{proof}

\begin{lemma}
\begin{equation}
\lim_{N\to\infty} \E\left[ \prod_{r=1}^{\tau_N(t)} (1-p_r) \right] = e^{-\alpha t}
\end{equation}
where $\alpha := n(n-1)/2$.
\end{lemma}
\begin{proof}
%\textbf{\\Lower Bound}\\
%From \citet[Lemma 1 Case 1]{koskela2018}, taking $\xi=\Delta$, we have
%\begin{equation}
%1-p_t = p_{\Delta\Delta}(t) \geq 1 - \alpha (1+O(N^{-1})) \left[ \frac{(3n-1)(n-2)}{6N^2} + c_N(t) \right] .
%\end{equation}
%Hence, by a multinomial expansion,
%\begin{align*}
%\prod_{r=1}^{\tau_N(t)} (1-p_r)
%&\geq \prod_{r=1}^{\tau_N(t)} \left\{ 1 - \alpha (1+O(N^{-1}) \left[ \frac{(3n-1)(n-2)}{6N^2} + c_N(r) \right] \right\} \\
%&= 1 + \sum_{k=1}^\infty \sum_{\substack{r_1<\dots<r_k \\ =1}}^{\tau_N(t)}\prod_{j=1}^k 
%\left\{ - \alpha (1+O(N^{-1})) \left[ \frac{(3n-1)(n-2)}{6N^2} + c_N(r_j) \right] \right\} \\
%&= 1 + \sum_{k=1}^\infty \left\{- \alpha (1+O(N^{-1}))\right\}^k \sum_{\substack{r_1<\dots<r_k \\ =1}}^{\tau_N(t)}\prod_{j=1}^k 
%\left\{ \frac{(3n-1)(n-2)}{6N^2} + c_N(r_j) \right\} 
%\end{align*}
%where the empty sum is taken to be zero.
%Taking expectations,
%\begin{equation}\label{eq:9}
%\E \left[ \prod_{r=1}^{\tau_N(t)} (1-p_r) \right]
%\geq 1 + \sum_{k=1}^\infty \left\{- \alpha (1+O(N^{-1}))\right\}^k \E \left[ \sum_{\substack{r_1<\dots<r_k \\ =1}}^{\tau_N(t)}\prod_{j=1}^k 
%\left\{ \frac{(3n-1)(n-2)}{6N^2} + c_N(r_j) \right\} \right]
%\end{equation}
%(the infinite sum has only finitely many non-zero summands, since the inner sum is empty for $k>\tau_N(t)$, which justifies swapping the sum and expectation.)
%From \citet[Equation (8)]{koskela2018},
%\begin{align*}
%\sum_{\substack{r_1<\dots<r_k \\ =1}}^{\tau_N(t)}\prod_{j=1}^k 
%\left\{ \frac{(3n-1)(n-2)}{6N^2} + c_N(r_j) \right\}
%&\geq \sum_{\substack{r_1<\dots<r_k \\ =1}}^{\tau_N(t)}\prod_{j=1}^k c_N(r_j) \\
%&\geq \frac{1}{k!} \left( \sum_{s=1}^{\tau_N(t)} c_N(s) \right)^k 
%- \frac{1}{k!} \binom{k}{2} \left( \sum_{s=1}^{\tau_N(t)} c_N(s)^2 \right)
%\left( \sum_{s=1}^{\tau_N(t)} c_N(s) \right)^{k-2} \\
%&\geq \frac{1}{k!} t^k
% - \frac{1}{k!} \binom{k}{2} (t+1)^{k-2} \left( \sum_{s=1}^{\tau_N(t)} c_N(s)^2 \right) .
%\end{align*}
%Then 
%\begin{equation}\label{eq:10}
%\E\left[ \sum_{\substack{r_1<\dots<r_k \\ =1}}^{\tau_N(t)}\prod_{j=1}^k 
%\left\{ \frac{(3n-1)(n-2)}{6N^2} + c_N(r_j) \right\}  \right] 
%\geq \frac{1}{k!} t^k - \frac{1}{k!} \binom{k}{2} (t+1)^{k-2} \E\left[ \sum_{s=1}^{\tau_N(t)} c_N(s)^2 \right]
% \longrightarrow \frac{1}{k!} t^k
%\end{equation}
%as $N\to\infty$ using \citet[Equation (5)]{brown2020}, via lemmata 1 and 3 therein.
%Similarly, applying \citet[Equation (9)]{koskela2018},
%\begin{equation}
%\sum_{\substack{r_1<\dots<r_k \\ =1}}^{\tau_N(t)}\prod_{j=1}^k c_N(r_j)
%\leq \frac{1}{k!} \left( \sum_{r=1}^{\tau_N(t)} c_N(r) \right)^k
%\leq \frac{1}{k!} \{t+ c_N(\tau_N(t)) \}^k
%\end{equation}
%Note that since $c_N(s) \in [0,1]$ for all $s$, $\E[c_N(s)^k] \leq \E[c_N(s)]$ for all $k\geq 1$. Hence,
%as $N\to\infty$, using \citet[Equation (3)]{brown2020}.
%
%Combining these upper and lower limits, we conclude that
%\begin{equation}
%1 + \sum_{k=1}^\infty \left\{- \alpha (1+O(N^{-1}))\right\}^k \E \left[ \sum_{\substack{r_1<\dots<r_k \\ =1}}^{\tau_N(t)}\prod_{j=1}^k 
%\left\{ \frac{(3n-1)(n-2)}{6N^2} + c_N(r_j) \right\} \right]
%\longrightarrow 1+ \sum_{k=1}^\infty (-\alpha)^k \frac{1}{k!} t^k
%= e^{-\alpha t}
%\end{equation}
%as $N\to\infty$.

\textbf{\\Lower Bound}\\ %(with weaker assumptions than above)
From \citet[Equation (14)]{brown2020}, taking $\xi=\Delta$, we have
\begin{equation}
1-p_t = p_{\Delta\Delta}(t) \geq 1 - \alpha (1+O(N^{-1})) \left[\frac{B_n}{\alpha} D_N(t) + c_N(t) \right]
\end{equation}
where $B_n >0$.
Hence, by a multinomial expansion,
\begin{align*}
\prod_{r=1}^{\tau_N(t)} (1-p_r)
&\geq \prod_{r=1}^{\tau_N(t)} \left\{ 1 - \alpha (1+O(N^{-1})\left[\frac{B_n}{\alpha} D_N(r) + c_N(r) \right] \right\} \\
&= 1 + \sum_{k=1}^\infty \sum_{\substack{r_1<\dots<r_k \\ =1}}^{\tau_N(t)}\prod_{j=1}^k 
\left\{ - \alpha (1+O(N^{-1})) \left[\frac{B_n}{\alpha} D_N(r_j) + c_N(r_j) \right] \right\} \\
&= 1 + \sum_{k=1}^\infty \left\{- \alpha (1+O(N^{-1}))\right\}^k \sum_{\substack{r_1<\dots<r_k \\ =1}}^{\tau_N(t)}\prod_{j=1}^k 
\left\{ \frac{B_n}{\alpha} D_N(r_j) + c_N(r_j) \right\}
\end{align*}
where the empty sum is taken to be zero.
Taking expectations,
\begin{equation}\label{eq:9}
\E \left[ \prod_{r=1}^{\tau_N(t)} (1-p_r) \right]
\geq 1 + \sum_{k=1}^\infty \left\{- \alpha (1+O(N^{-1}))\right\}^k \E \left[ \sum_{\substack{r_1<\dots<r_k \\ =1}}^{\tau_N(t)}\prod_{j=1}^k 
\left\{ \frac{B_n}{\alpha} D_N(r_j) + c_N(r_j) \right\} \right]
\end{equation}
(the infinite sum has only finitely many non-zero summands, since the inner sum is empty for $k>\tau_N(t)$, which justifies swapping the sum and expectation.)
We want to show that the expectation on the right converges to $t^k/k!$, for reasons that will become clear. The strategy is to upper and lower bound this expectation by quantities that converge to $t^k/k!$. 

First the lower bound.
From \citet[Equation (8)]{koskela2018},
\begin{align*}
\sum_{\substack{r_1<\dots<r_k \\ =1}}^{\tau_N(t)}\prod_{j=1}^k 
\left\{ \frac{B_n}{\alpha} D_N(r_j) + c_N(r_j) \right\}
&\geq \sum_{\substack{r_1<\dots<r_k \\ =1}}^{\tau_N(t)}\prod_{j=1}^k c_N(r_j) \\
&\geq \frac{1}{k!} \left( \sum_{s=1}^{\tau_N(t)} c_N(s) \right)^k 
- \frac{1}{k!} \binom{k}{2} \left( \sum_{s=1}^{\tau_N(t)} c_N(s)^2 \right)
\left( \sum_{s=1}^{\tau_N(t)} c_N(s) \right)^{k-2} \\
&\geq \frac{1}{k!} t^k
 - \frac{1}{k!} \binom{k}{2} (t+1)^{k-2} \left( \sum_{s=1}^{\tau_N(t)} c_N(s)^2 \right) .
\end{align*}
Then 
\begin{equation}\label{eq:10}
\E\left[ \sum_{\substack{r_1<\dots<r_k \\ =1}}^{\tau_N(t)}\prod_{j=1}^k 
\left\{ \frac{B_n}{\alpha} D_N(r_j) + c_N(r_j) \right\}  \right] 
\geq \frac{1}{k!} t^k - \frac{1}{k!} \binom{k}{2} (t+1)^{k-2} \E\left[ \sum_{s=1}^{\tau_N(t)} c_N(s)^2 \right]
 \longrightarrow \frac{1}{k!} t^k
\end{equation}
as $N\to\infty$ using \citet[Equation (5)]{brown2020}, via lemmata 1 and 3 therein.

Now for the upper bound. 
\begin{align*}
\sum_{\substack{r_1<\dots<r_k \\ =1}}^{\tau_N(t)}\prod_{j=1}^k 
\left\{ \frac{B_n}{\alpha} D_N(r_j) + c_N(r_j) \right\}
&= \frac{1}{k!} \sum_{\substack{r_1\neq\dots\neq r_k \\\text{all distinct}}}^{\tau_N(t)}
\prod_{j=1}^k 
\left\{ \frac{B_n}{\alpha} D_N(r_j) + c_N(r_j) \right\} \\
&= \frac{1}{k!} \sum_{\substack{r_1\neq\dots\neq r_k \\\text{all distinct}}}^{\tau_N(t)} 
\sum_{\mathcal{I}\subseteq \{1,\dots ,k\}}  
\left( \frac{B_n}{\alpha} \right)^{k-|\mathcal{I}|}
\left\{ \prod_{i\in\mathcal{I}} c_N(r_i) \right\} \left\{ \prod_{j\notin \mathcal{I}} D_N(r_j) \right\} \\
&= \frac{1}{k!} 
\sum_{\mathcal{I}\subseteq \{1,\dots ,k\}}  
\left( \frac{B_n}{\alpha} \right)^{k-|\mathcal{I}|}
\sum_{\substack{r_1\neq\dots\neq r_k \\\text{all distinct}}}^{\tau_N(t)}
\left\{ \prod_{i\in\mathcal{I}} c_N(r_i) \right\} \left\{ \prod_{j\notin \mathcal{I}} D_N(r_j) \right\} \\
&= \frac{1}{k!}
\sum_{I=0}^k  \binom{k}{I}
\left( \frac{B_n}{\alpha} \right)^{k-I}
\sum_{\substack{r_1\neq\dots\neq r_k \\\text{all distinct}}}^{\tau_N(t)} 
\left\{ \prod_{i=1}^{I} c_N(r_i) \right\} \left\{ \prod_{j=I+1}^k D_N(r_j) \right\} \\
&= \frac{1}{k!} \sum_{\substack{r_1\neq\dots\neq r_k \\\text{all distinct}}}^{\tau_N(t)} 
\left\{ \prod_{i=1}^{k} c_N(r_i) \right\} \\
&\qquad + \frac{1}{k!}
\sum_{I=0}^{k-1}  \binom{k}{I}
\left( \frac{B_n}{\alpha} \right)^{k-I}
\sum_{\substack{r_1\neq\dots\neq r_k \\\text{all distinct}}}^{\tau_N(t)} 
\left\{ \prod_{i=1}^{I} c_N(r_i) \right\} \left\{ \prod_{j=I+1}^k D_N(r_j) \right\} \\
&\leq  \frac{1}{k!} \left( \sum_{r=1}^{\tau_N(t)} c_N(r) \right)^k
+ \frac{1}{k!}
\sum_{I=0}^{k-1}  \binom{k}{I}
\left( \frac{B_n}{\alpha} \right)^{k-I}
\sum_{\substack{r_1\neq\dots\neq r_k \\\text{all distinct}}}^{\tau_N(t)} 
\left\{ \prod_{i=1}^{k-1} c_N(r_i) \right\} \left\{ D_N(r_k) \right\} \\
&\leq  \frac{1}{k!} \left\{ t + c_N(\tau_N(t)) \right\}^k
+ \frac{1}{k!}
\sum_{I=0}^{k-1}  \binom{k}{I}
\left( \frac{B_n}{\alpha} \right)^{k-I}
\left\{ \sum_{\substack{r_1\neq\dots\neq r_{k-1} \\\text{all distinct}}}^{\tau_N(t)} 
\prod_{i=1}^{k-1} c_N(r_i) \right\} 
\left\{ \sum_{r_k =1}^{\tau_N(t)} D_N(r_k) \right\} \\
&\leq  \frac{1}{k!} \left\{ t + c_N(\tau_N(t)) \right\}^k
+ \frac{1}{k!}
\sum_{I=0}^{k-1}  \binom{k}{I}
\left( \frac{B_n}{\alpha} \right)^{k-I}
\left( \sum_{r=1}^{\tau_N(t)} c_N(r) \right)^{k-1}
\left( \sum_{r=1}^{\tau_N(t)} D_N(r) \right) \\
&\leq  \frac{1}{k!} \left\{ t + c_N(\tau_N(t)) \right\}^k
+ \frac{1}{k!}
\sum_{I=0}^{k-1}  \binom{k}{I}
\left( \frac{B_n}{\alpha} \right)^{k-I}
(t+1)^{k-1}
\left( \sum_{r=1}^{\tau_N(t)} D_N(r) \right) .
%
%&\leq \frac{1}{k!}
%\sum_{I=0}^k  \binom{k}{I}
%\left( \frac{B_n}{\alpha} \right)^{k-I}
%\left\{\sum_{\substack{r_1\neq\dots\neq r_I \\\text{all distinct}}}^{\tau_N(t)}  \prod_{i=1}^{I} c_N(r_i) \right\} 
%\left\{ \sum_{\substack{r_{I+1}\neq\dots\neq r_k \\\text{all distinct}}}^{\tau_N(t)}  \prod_{j=I+1}^k D_N(r_j) \right\} \\
%&\leq \frac{1}{k!}
%\sum_{I=0}^k  \binom{k}{I}
%\left( \frac{B_n}{\alpha} \right)^{k-I}
%\left( \sum_{r=1}^{\tau_N(t)} c_N(r) \right)^I
%\left( \sum_{s=1}^{\tau_N(t)} D_N(s) \right)^{k-I} \\
%&\leq \frac{1}{k!}
%\sum_{I=0}^k  \binom{k}{I}
%\left( \frac{B_n}{\alpha} \right)^{k-I}
%\left( t+ c_N(\tau_N(t)) \right)^I
%\left( \sum_{s=1}^{\tau_N(t)} D_N(s) \right)^{k-I} \\
%&= \frac{1}{k!} \{ t+c_N(\tau_N(t)) \}^k
%+ \frac{1}{k!}
%\sum_{I=0}^{k-1}  \binom{k}{I}
%\left( \frac{B_n}{\alpha} \right)^{k-I}
%\left( t+ c_N(\tau_N(t)) \right)^I
%\left( \sum_{s=1}^{\tau_N(t)} D_N(s) \right)^{k-I} \\
%&\leq \frac{1}{k!} \{ t+c_N(\tau_N(t)) \}^k
%+ \frac{1}{k!}
%\sum_{I=0}^{k-1}  \binom{k}{I}
%\left( \frac{B_n}{\alpha} \right)^{k-I}
%(t+1)^I
%\left( \sum_{s=1}^{\tau_N(t)} D_N(s) \right)^{k-I} 
\end{align*}
Taking expectations, 
\begin{align*}
\lim_{N\to\infty} \E \left[ \sum_{\substack{r_1<\dots<r_k \\ =1}}^{\tau_N(t)}\prod_{j=1}^k 
\left\{ \frac{B_n}{\alpha} D_N(r_j) + c_N(r_j) \right\} \right]
&\leq \frac{1}{k!} \lim_{N\to\infty} \E[ \left\{ t + c_N(\tau_N(t)) \right\}^k ] \\
&\qquad+ \frac{1}{k!}
\sum_{I=0}^{k-1}  \binom{k}{I}
\left( \frac{B_n}{\alpha} \right)^{k-I}
(t+1)^{k-1}
\lim_{N\to\infty} \E\left[ \left( \sum_{r=1}^{\tau_N(t)} D_N(r) \right) \right] \\
&= \frac{1}{k!} t^k .
\end{align*}
The limit follows from \citet[Equations (3),(4)]{brown2020} along with the fact that, since $c_N(s) \in [0,1]$ for all $s$, $\E[c_N(s)^k] \leq \E[c_N(s)]$ for all $k\geq 1$, and the expansion
\begin{equation}
\E\left[ \frac{1}{k!} \{t+ c_N(\tau_N(t)) \}^k \right]
= \E\left[ \frac{1}{k!} \sum_{i=0}^k \binom{k}{i} t^i c_N(\tau_N(t))^{k-i} \right]
= \frac{1}{k!} \left\{ t^k + k t^{k-1} \E[c_N(\tau_N(t))] + \dots \right\}
\longrightarrow \frac{1}{k!}t^k .
\end{equation}

Combining these upper and lower limits, we conclude that
\begin{equation}
1 + \sum_{k=1}^\infty \left\{- \alpha (1+O(N^{-1}))\right\}^k \E \left[ \sum_{\substack{r_1<\dots<r_k \\ =1}}^{\tau_N(t)}\prod_{j=1}^k 
\left\{ \frac{B_n}{\alpha} D_N(r_j) + c_N(r_j) \right\} \right]
\longrightarrow 1+ \sum_{k=1}^\infty (-\alpha)^k \frac{1}{k!} t^k
= e^{-\alpha t}
\end{equation}
as $N\to\infty$.


\textbf{\\Upper Bound}\\
From \citet[Lemma 1 Case 1]{koskela2018}, taking $\xi=\Delta$, we have
\begin{equation}
1-p_t = p_{\Delta\Delta}(t) \leq 1 - \alpha (1+O(N^{-1})) \left[ c_N(t) - \binom{n-1}{2} D_N(t) \right] .
\end{equation}
A multinomial expansion as before yields
\begin{equation}
\prod_{r=1}^{\tau_N(t)} (1-p_r)
\leq 1 + \sum_{k=1}^\infty \left\{- \alpha (1+O(N^{-1}))\right\}^k \sum_{\substack{r_1<\dots<r_k \\ =1}}^{\tau_N(t)}\prod_{j=1}^k 
\left\{ c_N(r_j) - \binom{n-1}{2} D_N(r_j) \right\} .
\end{equation}
Similarly to \eqref{eq:10}, an upper bound for the inner sum is
\begin{equation}
\sum_{\substack{r_1<\dots<r_k \\ =1}}^{\tau_N(t)}\prod_{j=1}^k 
\left\{ c_N(r_j) - \binom{n-1}{2} D_N(r_j) \right\}
\leq \sum_{\substack{r_1<\dots<r_k \\ =1}}^{\tau_N(t)}\prod_{j=1}^k c_N(r_j)
\leq \frac{1}{k!} \left( \sum_{r=1}^{\tau_N(t)} c_N(r) \right)^k
\leq \frac{1}{k!} \{ t + c_N(\tau_N(t)) \}^k 
\end{equation}
with $\E[ \{ t + c_N(\tau_N(t)) \}^k /k! ] \longrightarrow t^k/k!$.

For the lower bound, stealing some results from the mega-align earlier,
\begin{align*}
\sum_{\substack{r_1<\dots<r_k \\ =1}}^{\tau_N(t)}\prod_{j=1}^k 
\left\{ c_N(r_j) - \binom{n-1}{2} D_N(r_j) \right\}
&= \frac{1}{k!} \sum_{\substack{r_1\neq\dots\neq r_k \\ \text{all distinct}}}^{\tau_N(t)}
\left\{ \prod_{i=1}^k c_N(r_i) \right\} \\
&\qquad + \frac{1}{k!} \sum_{I=0}^{k-1} \binom{k}{I} \left( -\binom{n-1}{2} \right)^{k-I}
\sum_{\substack{r_1\neq\dots\neq r_k \\ \text{all distinct}}}
\left\{ \prod_{i=1}^I c_N(r_i) \right\}
\left\{ \prod_{j=I+1}^k D_N(r_j) \right\} .
\end{align*}
First let us treat the first term:
\begin{align*}
\frac{1}{k!} \sum_{\substack{r_1\neq\dots\neq r_k \\ \text{all distinct}}}^{\tau_N(t)}
\left\{ \prod_{i=1}^k c_N(r_i) \right\} 
&\geq \frac{1}{k!} \left( \sum_{s=1}^{\tau_N(t)} c_N(s) \right)^k
- \frac{1}{k!} \binom{k}{2}  \left( \sum_{s=1}^{\tau_N(t)} c_N(s)^2 \right)  \left( \sum_{s=1}^{\tau_N(t)} c_N(s) \right)^{k-2} \\
&\geq \frac{1}{k!} t^k - \frac{1}{k!} \binom{k}{2} (t+1)^{k-2}  \left( \sum_{s=1}^{\tau_N(t)} c_N(s)^2 \right)
\end{align*}
and as before we have 
\begin{equation}
\lim_{N\to\infty} \E \left[ \frac{1}{k!} t^k - \frac{1}{k!} \binom{k}{2} (t+1)^{k-2}  \left( \sum_{s=1}^{\tau_N(t)} c_N(s)^2 \right) \right] = \frac{1}{k!} t^k .
\end{equation}
It remains to show that the expectation of the second term converges to zero.
\begin{align*}
 \frac{1}{k!} \sum_{I=0}^{k-1} \binom{k}{I} \left( -\binom{n-1}{2} \right)^{k-I}
&\sum_{\substack{r_1\neq\dots\neq r_k \\ \text{all distinct}}}
\left\{ \prod_{i=1}^I c_N(r_i) \right\}
\left\{ \prod_{j=I+1}^k D_N(r_j) \right\} \\
&= \frac{1}{k!} \sum_{\substack{I=0 \\ (k-I) \text{even}}}^{k-1}  \binom{k}{I} \binom{n-1}{2}^{k-I}
\sum_{\substack{r_1\neq\dots\neq r_k \\ \text{all distinct}}}
\left\{ \prod_{i=1}^I c_N(r_i) \right\}
\left\{ \prod_{j=I+1}^k D_N(r_j) \right\} \\
&\qquad - \frac{1}{k!} \sum_{\substack{I=0 \\ (k-I) \text{odd}}}^{k-1}  \binom{k}{I} \binom{n-1}{2}^{k-I}
\sum_{\substack{r_1\neq\dots\neq r_k \\ \text{all distinct}}}
\left\{ \prod_{i=1}^I c_N(r_i) \right\}
\left\{ \prod_{j=I+1}^k D_N(r_j) \right\} \\
&\geq 0 
- \frac{1}{k!} \sum_{\substack{I=0 \\ (k-I) \text{odd}}}^{k-1}  \binom{k}{I} \binom{n-1}{2}^{k-I}
(t+1)^{k-1}
\left\{ \sum_{s=1}^{\tau_N(t)} D_N(s) \right\} \\
\end{align*}
using that $c_N(r), D_N(r) \geq 0$ for all $r$ to bound the even terms below, and arguments from the mega-align earlier to bound the odd terms above.
Taking the expectation and limit yields the desired result:
\begin{align*}
\lim_{N\to\infty} &\E \left[ \frac{1}{k!} \sum_{I=0}^{k-1} \binom{k}{I} \left( -\binom{n-1}{2} \right)^{k-I}
\sum_{\substack{r_1\neq\dots\neq r_k \\ \text{all distinct}}}
\left\{ \prod_{i=1}^I c_N(r_i) \right\}
\left\{ \prod_{j=I+1}^k D_N(r_j) \right\} \right] \\
&\qquad\qquad \geq - \frac{1}{k!} \sum_{\substack{I=0 \\ (k-I) \text{odd}}}^{k-1}  \binom{k}{I} \binom{n-1}{2}^{k-I}
(t+1)^{k-1}
\lim_{N\to\infty} \E \left[ \sum_{s=1}^{\tau_N(t)} D_N(s) \right]
= 0 .
\end{align*}

Combining these upper and lower limits, we conclude that
\begin{equation}
1 + \sum_{k=1}^\infty \left\{- \alpha (1+O(N^{-1}))\right\}^k \E \left[ \sum_{\substack{r_1<\dots<r_k \\ =1}}^{\tau_N(t)}\prod_{j=1}^k 
\left\{ c_N(r_j) - \binom{n-1}{2} D_N(r_j) \right\} \right]
\longrightarrow 1+ \sum_{k=1}^\infty (-\alpha)^k \frac{1}{k!} t^k
= e^{-\alpha t}
\end{equation}
as $N\to\infty$.

We now have upper and lower bounds on $\lim_{N\to\infty} \E\left[ \prod_{r=1}^{\tau_N(t)} (1-p_r) \right]$, both of which are equal to $e^{-\alpha t}$, so we're done.
\end{proof}

\bibliography{../smc.bib}
\end{document}