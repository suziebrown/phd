\documentclass[fleqn]{article}
\usepackage[utf8]{inputenc}
\usepackage[margin=2.5cm]{geometry}

% bibliography
\usepackage[round, sort&compress]{natbib}
\usepackage{har2nat}
\bibliographystyle{agsm}

% custom header/footer
\usepackage{fancyhdr}
\pagestyle{fancy}
\renewcommand{\headrulewidth}{0pt}
\fancyhf{}
\rfoot{\textsf{\thepage}}
\lfoot{\textsf{Suzie Brown}}

% miscellaneous formatting
\usepackage{xcolor}
\usepackage[font=small]{caption}
\usepackage{subfig}

% tikz
\usepackage{tikz}
\usetikzlibrary{positioning}

% pseudocode
\usepackage{algorithm}
\usepackage{algorithmicx}
\usepackage{algpseudocode}

% maths
\usepackage{amsmath}
\usepackage{amssymb}
\usepackage{amsthm}
\newtheorem{corollary}{Corollary}
\theoremstyle{definition}
\newtheorem{defn}{Definition}

% useful math symbols
\newcommand{\PR}{\mathbb{P}}
\newcommand{\E}{\mathbb{E}}
\newcommand{\V}{\operatorname{Var}}
\newcommand{\eqdist}{\overset{d}{=}}
\newcommand{\I}[1]{\mathbb{I}\{#1\}}
\newcommand{\Ntoinfty}{\overset{N\to\infty}{\longrightarrow}}
\newcommand{\limNtoinfty}{\underset{N\to\infty}{\lim}}
\newcommand\indep{\protect\mathpalette{\protect\independenT}{\perp}}
\def\independenT#1#2{\mathrel{\rlap{$#1#2$}\mkern2mu{#1#2}}}

% distributions
\newcommand{\Cat}{\operatorname{Categorical}}
\newcommand{\Unif}{\operatorname{Uniform}}
\newcommand{\Mn}{\operatorname{Multinomial}}
\newcommand{\Bin}{\operatorname{Binomial}}

% project-specific commands
\newcommand{\F}{\mathcal{F}_{t-1}}
\newcommand{\vt}[2][t]{v_{#1}^{(#2)}}
\newcommand{\wt}[2][t]{w_{#1}^{(#2)}}
\newcommand{\wbar}[2][t]{\bar{w}_{#1}^{(#2)}}
\newcommand{\vttilde}[2][t]{\tilde{v}_{#1}^{(#2)}}

\title{Counter-example to multinomial dominating residual coalescence rate}
\author{Suzie Brown}
\date{\today}

\begin{document}
\maketitle
\thispagestyle{fancy}

We have conjectured that the expected coalescence rate should be higher for multinomial resampling than for residual resampling. This is definitely true asymptotically (equation 9 in the report), but here is a counter-example in the finite case.

Take the scenario with $N=3$ particles, and weights $\wt{1:3} = (0.3, 0.3, 0.4)$.
The possible outcomes of resampling are listed in the table below, along with their probabilities under multinomial and residual resampling.\\

\begin{tabular}{c | c | c | c | c}
Outcome & No. of cases & $\sum (\vt{i})_2$ & Probability (multinomial) & Probability (residual) \\
\hline
$(3,0,0)$ or $(0,3,0)$ & 2 & 6 & 27/1000 & 0\\
$(0,0,3)$ & 1 & 6 & 64/1000 & 4/64 \\
$(2,1,0)$ or $(1,2,0)$ & 2 & 2 & 81/1000 & 0 \\
$(2,0,1)$ or $(0,2,1)$ & 2 & 2  & 108/1000 & 9/64 \\
$(1,0,2)$ or $(0,1,2)$ & 2 & 2 & 144/1000 & 12/64 \\
$(1,1,1)$ & 1 & 0 & 216/1000 & 18/64 \\
\end{tabular}\\

Actually not really, I just added it up wrong the first time.
\begin{align*}
& \E[c_N^m(t) |\F] = 0.34 \\
& \E[c_N^r(t) |\F] = 18/64 \simeq 0.28
\end{align*}
\end{document}