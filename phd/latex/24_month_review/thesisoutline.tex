\documentclass{article}
\usepackage[utf8]{inputenc}
\usepackage[margin=2.5cm]{geometry}

% bibliography
\usepackage[round, sort&compress]{natbib}
\usepackage{har2nat}
\bibliographystyle{agsm}

\title{Thesis Outline at 24 months}
\author{Suzie Brown}
\date{30 September 2019}

\begin{document}
\maketitle

\section{Introduction (done)}
\subsection{Interacting Particle Systems}
General description of what an IPS is, what they are useful for, and how to simulate one. SMC as a specific class of IPS.
\subsection{Coalescent Theory}
Review of literature from population genetics, including the relevant population models and results about their coalescents.
\subsection{SMC Genealogies}
Description of how genealogies are induced by SMC algorithms and how this affects the performance of the algorithms. Existing results characterising these genealogies. Algorithms and variants previously proposed to mitigate ancestral degeneracy.

\section{Limiting Coalescents for SMC Genealogies (done)}
\subsection{General Result for IPSs}
A refinement of \citet[Theorem 1]{koskela2018} with more tractable conditions. Proof of the theroem. 
\subsection{Application to Stochastic Rounding-based Resampling}
Introduction of stochastic roundings, and how they can be used for resampling in SMC. Key examples of resampling schemes that can be seen as stochastic rounding. Comparison of this type of resampling with more naive methods, supported by results from the SMC literature. Implementation and usage in practice. Proof that the Theorem hold for SMC with stochastic rounding-based resampling, and interpretation of that result.
\subsection{Application to Conditional SMC}
Introduction of conditional SMC as a key component of particle MCMC, and what type of problem this is useful for. Proof that the theorem holds for conditional SMC. Discussion of behaviour in the pre-limiting regime, supported by simulation studies.

\section{Stronger Mode of Convergence (future work)}
So far I only proved convergence in the sense of finite-dimensional distributions. In SMC applications we are typically interested in expectations of test functions, so an upgrade to weak convergence is desirable. The required tightness argument has been set out in the case of neutral IPSs \citep{mohle1999}, but an extension to the non-neutral case required for application to SMC is non-trivial. I will attempt to extend the techniques of \citet{mohle1999} and \citet{gerber2017} to prove weak convergence for these SMC algorithms.

\section{Variance Estimation for SMC Algorithms (future work)}
Variance estimators for SMC are intimately linked to the genealogies of samples (e.g.\ \citet{chan2013}, \citet{olsson2019}). Analysis of genealogies can therefore provide some insights into this area in terms of tuning and analysing the performance of existing estimators and perhaps proposing new estimators based on proven genealogical properties. I will review the variance estimation literature to find out exactly how this is related to genealogies, then see what results I could contribute to this area.

\bibliography{../smc.bib}
\end{document}