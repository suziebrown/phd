\documentclass{article}
\usepackage[utf8]{inputenc}
\usepackage[margin=2.5cm]{geometry}

% bibliography
\usepackage[round, sort&compress]{natbib}
\usepackage{har2nat}
\bibliographystyle{agsm}

\title{Asymptotic genealogies of sequential Monte Carlo algorithms\\Thesis outline at 36 months}
\author{Suzie Brown}
\date{21 October 2020}

\begin{document}
\maketitle

\section{Introduction}

\section{Background}
\subsection{Interacting particle systems / sequential Monte Carlo}
General description of what an IPS is, what they are useful for, and how to simulate one. SMC as a specific class of IPS. Relevant literature about SMC (e.g.\ convergence results).
\subsection{Coalescent theory}
Review of literature from population genetics, introducing the relevant population models (Wright--Fisher, Moran, Cannings) and Kingman's coalescent / $n$-coalescent. Domain of attraction of Kingman's coalescent, as far as previous works have shown. Properties of such models (neutrality, Markov property) that may be violated by SMC systems.
\subsection{SMC genealogies}
Description of how genealogies are induced by SMC algorithms and how this is related to the performance of the algorithms (ancestral degeneracy, variance estimation). Existing results characterising these genealogies.
\subsection{Resampling} % could be its own chapter if I have a lot of material
Definition of a `valid' resampling scheme and justification for these restrictions. Tour of key resampling schemes (multinomial, residual, stratified, systematic, ...), with discussion of their properties, implementation and usage in practice. Idea of `optimality' in resampling, description of so-called optimal schemes. Existing results and conjectures comparing the performance of different schemes. Introduction of stochastic rounding as a class of resampling schemes. Adaptive resampling.
%% Examples and discussion of resampling schemes that violate the three properies; optimal transport resampling, along with the others I mentioned in previous writings.

\section{Limiting coalescents for SMC genealogies (done)}
\subsection{General result for IPSs}
Definition of the relevant quantities. Proof that finite-dimensional distributions of the rescaled genealogies converge to the $n$-coalescent (a refinement of \citet[Theorem 1]{koskela2018} with more tractable conditions). Interpretation of the conditions.
\subsection{Application to multinomial resampling}
Proof of corollary for multinomial resampling, the simplest/most tractable case. This was already proved in \citet{koskela2018} but having simplified conditions for the main theorem, the proof is made briefer and more elegant.
\subsection{Application to stochastic rounding}
Proof of corollary for SMC with any stochastic rounding-based resampling. Comparison of time scale versus multinomial resampling.
%% Also include in this chapter a note on the effect of adaptive resampling; probably trivial enough to be a brief remark.

\section{Conditional SMC (done)}
Introduction of conditional SMC as key component of particle MCMC / particle Gibbs, and what type of problem this is useful for. Proof of corollary for conditional SMC. Discussion of pre-limiting behaviour (possibly with simulation study).
\subsection{Ancestor sampling}
Motivation for ancestor sampling within particle Gibbs. Description of algorithm. Conditions under which it is possible. Effect of ancestor sampling on asymptotic genealogies.

\section{Weak convergence (in progress)}
Weak convergence result under the same conditions as the convergence in finite-dimensional distributions. I am currently working on the required tightness argument, somewhat following M\"ohle's proof in the neutral case \citep{mohle1999}, but with non-trivial adjustments because our process is not Markovian.

%\section{Variance Estimation for SMC Algorithms (future work)}
%Variance estimators for SMC are intimately linked to the genealogies of samples (e.g.\ \citet{chan2013}, \citet{olsson2019}). Analysis of genealogies can therefore provide some insights into this area in terms of tuning and analysing the performance of existing estimators and perhaps proposing new estimators based on proven genealogical properties. I will review the variance estimation literature to find out exactly how this is related to genealogies, then see what results I could contribute to this area.

\section{Conclusions / discussion}

\bibliography{../latex/smc.bib}
\end{document}