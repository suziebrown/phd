\documentclass{article}
\usepackage[utf8]{inputenc}
\usepackage[margin=2cm]{geometry}

\usepackage{graphicx}
\usepackage{enumitem}

% custom header/footer
\usepackage{fancyhdr}
\pagestyle{fancy}
\renewcommand{\headrulewidth}{0pt}
\fancyhf{}
\rfoot{\textsf{\thepage}}
\lfoot{\textsf{Suzie Brown}}

%annotations
\usepackage{color}
\usepackage{xspace}
\newcommand{\seb}[1]{\xspace\textcolor{red}{#1}\xspace}

% bibliography
\usepackage[round, sort&compress]{natbib}
\usepackage{har2nat}
\bibliographystyle{agsm}

% maths
\usepackage{amsmath}
\usepackage{amssymb}
\usepackage{amsthm}
\newtheorem{thm}{Theorem}
\newtheorem{lemma}{Lemma}
\newtheorem{prop}{Proposition}
\newcommand{\Prob}{\mathbb{P}}
\newcommand{\E}{\mathbb{E}}
\newcommand{\Et}{\mathbb{E}_t}
\newcommand{\V}{\operatorname{Var}}
\newcommand{\I}[1]{\mathbb{I}_{\{#1\}}}
\newcommand{\1}[1]{\mathbb{I}_{#1}}
\newcommand{\Mn}{\operatorname{Multinomial}}
\newcommand{\flnw}{\lfloor N w_t^{(i)} \rfloor}

\title{Non-triviality condition (fuller details of paper appendix)}
\author{Suzie Brown}

\begin{document}
\maketitle
\thispagestyle{fancy}

The following theorem will be used in each section. It is a filtered version of the second Borel--Cantelli lemma, which can be found for instance in \citet[Theorem 4.3.4]{durrett2019}.
\begin{lemma}\label{lem:BC2filtered}
Let $ (\mathcal{F}_t)_{t\geq 0}$ be a filtration with $\mathcal{F}_0 = \{\emptyset, \Omega\}$. Let $(B_t)_{t\geq 0}$ be a sequence of events such that $B_t \in \mathcal{F}_t$ for all $t$.
Then the events $\{ B_t \text{ i.o.} \}$ and $\{ \sum_{t=1}^\infty \Prob[B_t \mid \mathcal{F}_{t-1} ] =\infty \}$ are almost surely equal.
\end{lemma}

We will also use the following equivalence in each section.
\begin{lemma}\label{lem:nontriv_sufficientcond}
Let $\tau_N$ denote the generalised inverse of $c_N$, i.e.\
\begin{equation*}
\tau_N(t) = \min \left\{ s\geq 1 : \sum_{r=1}^s c_N(r) \geq t \right\} .
\end{equation*}
Suppose that there exists $N_0 \in \mathbb{N}$ such that almost surely for all $N>N_0$, $c_N(t)$ is bounded away from zero for infinitely many $t$.
Then for all $N>N_0$, for all finite $t$, $\Prob[\tau_N(t) = \infty] =0$.
\end{lemma}

\begin{proof}
Applying the definition of $\tau_N(t)$,
\begin{align*}
\Prob[\tau_N(t) = \infty] =0 &\Leftrightarrow \Prob[\tau_N(t) < \infty] =1 \\
&\Leftrightarrow \Prob\left[ \min \left\{s \geq 1 : \sum_{r=1}^s c_N(r) \geq t \right\} < \infty \right] =1 \\
&\Leftrightarrow \Prob\left[ \exists s<\infty : \sum_{r=1}^s c_N(r) \geq t \right] =1
\end{align*}
A sufficient condition for the last line is that, almost surely for all $N>N_0$, $c_N(r)$ is bounded away from zero for infinitely many $r$.
\end{proof}

Combining Lemmata \ref{lem:BC2filtered} and \ref{lem:nontriv_sufficientcond}, we see that a sufficient condition for non-triviality is
\begin{equation}\label{eq:nontriv_sufficientcond}
\sum_{t=0}^\infty \Prob[ c_N(t) > 2/N^2 \mid \mathcal{F}_{t-1}] = \infty .
\end{equation}


\section*{Multinomial resampling}

\begin{prop}
In standard SMC with multinomial resampling, there exists $N_0$ such that for all $N>N_0$, for all finite $t$, $\Prob[\tau_N(t) = \infty] =0$.
\end{prop}

\begin{proof}
We have the following lower bound:
\begin{equation*}
\E [c_N(t) \mid \mathcal{F}_{t-1}] \geq \frac{\varepsilon^4}{Na^4}.
\end{equation*}
Since $c_N(t) \in [0,1]$ almost surely, for any fixed $N$ the ``worst-case'' distribution of $c_N(t)$ (i.e.\ maximising $\Prob[c_N(t)=0 \mid \mathcal{F}_{t-1}]$) is two atoms, at 0 and 1. To ensure the correct expectation, the atom at 1 must have weight $\E[c_N(t) \mid \mathcal{F}_{t-1}]$, which is bounded below by the above inequality.
Hence for any finite $N$,
\begin{equation*}
\sum_{t=0}^\infty \Prob[ c_N(t) > 2/N^2 \mid \mathcal{F}_{t-1}] 
\geq \sum_{t=0}^\infty \frac{\varepsilon^4}{Na^4}
= \infty .
\end{equation*}
Applying Lemmata \ref{lem:BC2filtered} and \ref{lem:nontriv_sufficientcond} yields the result.
\end{proof}


\section*{Conditional SMC with multinomial resampling}

\begin{prop}
In conditional SMC with multinomial resampling, there exists $N_0$ such that for all $N>N_0$, for all finite $t$, $\Prob[\tau_N(t) = \infty] =0$.
\end{prop}

\begin{proof}
We have the following lower bound:
\begin{equation*}
\E [c_N(t) \mid \mathcal{F}_{t-1}] \geq \frac{1}{(N)_2} \left\{ \frac{2\varepsilon^2}{a^2} + \frac{(N)_3 \varepsilon^4}{(N-1)^2 a^4} \right\} .
\end{equation*}
Since $c_N(t) \in [0,1]$ almost surely, for any fixed $N$ the ``worst-case'' distribution of $c_N(t)$ (i.e.\ maximising $\Prob[c_N(t)=0 \mid \mathcal{F}_{t-1}]$) is two atoms, at 0 and 1. To ensure the correct expectation, the atom at 1 must have weight $\E[c_N(t) \mid \mathcal{F}_{t-1}]$, which is bounded below by the above inequality.
Hence for any finite $N$,
\begin{equation*}
\sum_{t=0}^\infty \Prob[ c_N(t) > 2/N^2 \mid \mathcal{F}_{t-1}] 
\geq \sum_{t=0}^\infty \frac{1}{(N)_2} \left\{ \frac{2\varepsilon^2}{a^2} + \frac{(N)_3 \varepsilon^4}{(N-1)^2 a^4} \right\}
= \infty .
\end{equation*}
Applying Lemmata \ref{lem:BC2filtered} and \ref{lem:nontriv_sufficientcond} yields the result.
\end{proof}


\section*{Stochastic rounding}

The stochastic rounding case is more involved than the others, because there is not a positive lower bound on $\Et[c_N(t)]$ that holds for all weight vectors. (In particular, $\Et[c_N(t)]=0$ when the weights are all exactly equal.)
First notice the following.

\begin{lemma} \label{lem:extreme_w_coal_as}
Let $\mathbf{w} = (w_1,\dots,w_N) \in \mathcal{S}_{N-1}$ and resample by stochastic rounding.
\begin{enumerate}[label=(\roman*)]
\item If $w_i \geq 2/N$ for some $i$, then $\Prob[c_N(t) > 2/N^2 \mid \mathcal{H}_t ] =1$. \label{item:SR_weight_2}
\item If $w_i= 0$ for some $i$, then $\Prob[c_N(t) > 2/N^2 \mid \mathcal{H}_t ] =1$. \label{item:SR_weight_0}
\end{enumerate}
\end{lemma}

\begin{proof}
For any $N$ it is easy to see, by considering the possible values of $c_N(t)$, that $c_N(t) > 2/N^2$ if and only if $c_N(t) \neq 0$. The only way to attain $c_N(t) = 0$ is to assign exactly one offspring to each particle.
In case \ref{item:SR_weight_2} particle $i$ is assigned at least two offspring, so $c_N(t)$ cannot be equal to zero.
In case \ref{item:SR_weight_0} particle $i$ is assigned zero offspring, so $c_N(t)$ cannot be equal to zero.
\end{proof}

The upshot of Lemma \ref{lem:extreme_w_coal_as} is that we need only prove \eqref{eq:nontriv_sufficientcond} in the case where all weights are in $(0, 2/N)$, since it holds automatically otherwise.

\begin{lemma}\label{lem:weps_cN_prob}
Define $\mathbf{w}^\delta := \frac{1}{N}\{(1,\dots,1) + \delta \mathbf{e}_i - \delta \mathbf{e}_j \}$ for any $i \neq j$ and $0< \delta < 1$. Then $\Prob[c_N(t) > 2/N^2 | \mathcal{H}_t, \mathbf{w}_t = \mathbf{w}_\delta] \geq \delta \varepsilon^3$.
\end{lemma}

\begin{proof}
We use a bound on $\Prob[ \nu_t^{(i)} = \lfloor N w_t^{(i)} \rfloor ]$ from the proof of Corollary 1 in the draft paper:
\begin{equation*}
\Prob[ \nu_t^{(i)} = \lfloor N w_t^{(i)} \rfloor \mid \mathcal{H}_t] =: p_0 = 1-p_1 \leq 1 - (Nw_t^{(i)} - \lfloor N w_t^{(i)} \rfloor ) \varepsilon^{(2\lfloor N w_t^{(i)} \rfloor +1)} .
\end{equation*}
Then
\begin{align*}
\Prob[c_N(t) \leq 2/N^2 | \mathcal{H}_t, \mathbf{w}_t = \mathbf{w}_\delta]
&= \Prob[\nu_t^{(1:N)} = (1,\dots, 1) \mid \mathcal{H}_t, \mathbf{w}_t = \mathbf{w}_\delta] \\
&= \Prob[\nu_t^{(i)} = 1, \nu_t^{(j)} = 1 \mid \mathcal{H}_t, \mathbf{w}_t = \mathbf{w}_\delta] \\
&= \Prob[\nu_t^{(i)} = 1 \mid \mathcal{H}_t, \mathbf{w}_t = \mathbf{w}_\delta] \\
&\leq 1- (N w_\delta^{(i)} - \lfloor N w_\delta^{(i)} \rfloor) \varepsilon^{(2\lfloor N w_\delta^{(i)} \rfloor +1)} \\
&= 1- \{N(1+ \delta)/N - 1\}\varepsilon^3 \\
&= 1 - \delta\varepsilon^3 ,
\end{align*}
since the offspring counts are deterministically equal to one apart from particles $i$ and $j$, and it remains that $\nu_t^{(i)} = 1$ if and only if $\nu_t^{(j)} = 1$.
\end{proof}




\begin{lemma}
For any $\delta \in (0, 1)$, denote $\mathcal{S}_{N-1}^\delta := \{ \mathbf{w} \in \mathcal{S}_{N-1} :  \forall i, \, 0 <w_i <\frac{2}{N} ;\, \max_i w_i \geq \frac{1 + \delta}{N} \}$.
Then for all $\mathbf{w} \in \mathcal{S}_{N-1}^\delta$, 
$\Prob[c_N(t) > 2/N^2 | \mathbf{w} ] \geq \Prob[c_N(t) > 2/N^2 | \mathbf{w}_\delta ]$.
\end{lemma}

\begin{proof}
Fix arbitrary $\mathbf{w} \in \mathcal{S}_{N-1}^\delta$. Let $i^*$ be then index of the particle with the largest weight.
Denote $\mathcal{I} := \{i \in \{1,\dots,N\} : w_i > 1/N \}$.
Notice that 
\begin{equation*}
\Prob[ c_N(t) \leq 2/N^2 | \mathbf{w} ] 
= \Prob[ \nu_t^{(i)} =1 \,\forall i\in\{1,\dots,N\} | \mathbf{w}] 
= \Prob[ \nu_t^{(i)} =1 \,\forall i\in \mathcal{I} | \mathbf{w}] .
\end{equation*}
This is true because all weights are in $(0, 2/N)$, so for $i \in \mathcal{I}, \nu_t^{(i)} \in \{1,2\}$, and for $i \notin \mathcal{I}, \nu_t^{(i)} \in \{0,1\}$; and the offspring counts must sum to $N$ (a generalisation of the argument used in Lemma \ref{lem:weps_cN_prob}).

We can then decompose this probability into a product of conditional probabilities:
\begin{align*}
\Prob[ \nu_t^{(i)} =1 \,\forall i\in \mathcal{I} | \mathbf{w}]
&= \prod_{i \in \mathcal{I}} \Prob[ \nu_t^{(i)} =1 | \nu_t^{(j)}=1 \,\forall j <i \in \mathcal{I}; \mathbf{w}] \\
&= \Prob[\nu_t^{(i^*)} =1 | \mathbf{w}] \prod_{i \neq i^* \in \mathcal{I}} \Prob[ \nu_t^{(i)} =1 | \nu_t^{(i^*)}=1; \nu_t^{(j)}=1 \,\forall j <i \in \mathcal{I}; \mathbf{w}] \\
&\leq \Prob[\nu_t^{(i^*)} =1 | \mathbf{w}] .
\end{align*}
The last line is equal to the probability $\Prob[ c_N(t) \leq 2/N^2 | \mathbf{w} ] $ in the case where $|\mathcal{I}| =1$, i.e.\ the only weight larger than $1/N$ is $w_{i^*}$.

In other words, $\Prob[ c_N(t) > 2/N^2 | \mathbf{w} ]$ is minimised on $\mathcal{S}_{N-1}^\delta$ by having only one weight larger than $1/N$, in which case the values of the other weights do not affect this probability. 

We therefore find that a minimum of $\Prob[ c_N(t) > 2/N^2 | \mathbf{w} ]$ on $\mathcal{S}_{N-1}^\delta$ is given by $\mathbf{w}_{\delta^\prime}$, for some $\delta^\prime \geq \delta$. 
It only remains to show that taking $\delta^\prime > \delta$ does not decrease the probability. This is a consequence of Lemma \ref{lem:weps_cN_prob}, where we see that $\Prob[ c_N(t) > 2/N^2 | \mathbf{w}_{\delta^\prime}]$ is monotonically increasing in $\delta^\prime$.
Thus the minimum of $\Prob[ c_N(t) > 2/N^2 | \mathbf{w} ]$ is attained at $\mathbf{w} = \mathbf{w}_\delta$, as required. (Although this minimum is not unique, we have shown explicitly that it is a global minimum on $\mathcal{S}_{N-1}^\delta$.)
\end{proof}

\begin{thm}
Consider a sequential Monte Carlo algorithm using any stochastic rounding as its resampling scheme.
If there exists $\mu>0$ such that $\Prob\{\max_i w_t^{(i)} \geq (1+\delta)/N \mid \mathcal{H}_t\} \geq \mu$ for infinitely many $t$ then $\Prob\{\tau_N(t) = \infty \}=0$ for all $N>1$ and for all finite $t$.
\end{thm}

\begin{proof}
Combining Lemmata [7--9] we see that, for any $\mathbf{w} \in \mathcal{S}_{N-1}$ such that $\max_i w_i \geq \frac{1 + \delta}{N}$, we have the bound $\Prob[ c_N(t) > 2/N^2 | \mathbf{w} ] \geq \delta\varepsilon^3$.
By the law of total probability,
\begin{equation*}
\Prob[c_N(t) > 2/N^2 \mid \mathcal{H}_t] 
\geq \Prob[c_N(t) > 2/N^2 \mid \mathcal{H}_t, \max w_i \geq (1+\delta)/N ]\, \Prob[\max w_i \geq (1+\delta)/N \mid \mathcal{H}_t]
\geq \mu \delta \varepsilon^3 
\end{equation*}
for those infinitely many $t$ where $\Prob\{\max_i w_t^{(i)} \geq (1+\delta)/N \mid \mathcal{H}_t\} \geq \mu$.
Using the D-separation established in [draft paper, Cor 1 proof], we can write
\begin{align*}
\Prob[c_N(t) > 2/N^2 \mid \mathcal{F}_{t-1}] 
&= \E[ \I{c_N(t) > 2/N^2} \mid \mathcal{F}_{t-1}] \\
&= \E[ \E[ \I{c_N(t) > 2/N^2} \mid \mathcal{H}_t ]\mid \mathcal{F}_{t-1}] \\
&= \E[ \Prob[c_N(t) > 2/N^2 \mid \mathcal{H}_t ]\mid \mathcal{F}_{t-1}] .
\end{align*}
Hence this probability is bounded below by $\mu\delta\varepsilon^3$ for infinitely many $t$. We therefore have
\begin{equation}
\sum_{t=0}^\infty \Prob[c_N(t) > 2/N^2 \mid \mathcal{F}_{t-1}]  \geq \sum_{j=0}^\infty \mu\delta\varepsilon^3 = \infty ,
\end{equation}
and applying Theorem [1 - that BC2 statement], almost surely $c_N(t) >2/N^2$ for infinitely many $t$.
As argued in Lemma [2], this is sufficient for the result.
\end{proof}

The lemma below is here to clear up any uncertainty about the tower property / D-separation argument, as used in this proof in the paper.
\begin{lemma}
Let $A,B$ be events such that $A$ is measurable with respect to $\mathcal{F}_{t}$, and $B$ is measurable with respect to $\mathcal{H}_{t}$ (but not vice versa), and neither event is measurable with respect to $\mathcal{F}_{t-1}$. 
(In the real proof we have $A:= \{c_N(t)>2/N^2\}$ and $B:= \{ \max_i w_t^{(i)} - \min_i w_t^{(i)} \geq 2\delta/N \}$).
Then
\begin{equation}
\Prob[A \mid \mathcal{F}_{t-1}, B] 
=\E[ \Prob[ A \mid \mathcal{H}_t] \mid \mathcal{F}_{t-1}, B ] 
\end{equation}
and
\begin{equation}
\Prob[A \mid \mathcal{H}_t] 
\geq \Prob[ A \mid \mathcal{H}_t, B] \, \1{B}.
\end{equation}
\end{lemma}

\begin{proof}
For the first point,
\begin{align*}
\Prob[A \mid \mathcal{F}_{t-1}, B] 
&= \E[ \1{A} \mid \mathcal{F}_{t-1}, B]
= \frac{\E[ \1{A} \1{B} \mid \mathcal{F}_{t-1}]}{\Prob[B \mid \mathcal{F}_{t-1}]}
=\frac{\E[ \E[ \1{A} \1{B} \mid \mathcal{H}_t, \mathcal{F}_{t-1} ] \mid \mathcal{F}_{t-1}]}{\Prob[B \mid \mathcal{F}_{t-1}]}
=\frac{\E[ \E[ \1{A} \1{B} \mid \mathcal{H}_t ] \mid \mathcal{F}_{t-1}]}{\Prob[B \mid \mathcal{F}_{t-1}]}\\
&=\frac{\E[ \E[ \1{A} \mid \mathcal{H}_t ] \1{B} \mid \mathcal{F}_{t-1}]}{\Prob[B \mid \mathcal{F}_{t-1}]}
=\frac{\E[ \E[ \1{A} \mid \mathcal{H}_t ] \mid \mathcal{F}_{t-1}, B ] \Prob[B \mid \mathcal{F}_{t-1}]}{\Prob[B \mid \mathcal{F}_{t-1}]} \\
&=\E[ \E[ \1{A} \mid \mathcal{H}_t ] \mid \mathcal{F}_{t-1}, B ] 
=\E[ \Prob[ A \mid \mathcal{H}_t] \mid \mathcal{F}_{t-1}, B ].
\end{align*}
For the second point,
\begin{align*}
\Prob[A \mid \mathcal{H}_t] 
=  \Prob[ A \mid \mathcal{H}_t, B] \Prob[B \mid \mathcal{H}_t]
+ \Prob[ A \mid \mathcal{H}_t, B^c] \Prob[B^c \mid \mathcal{H}_t]
\geq  \Prob[ A \mid \mathcal{H}_t, B] \Prob[B \mid \mathcal{H}_t]
=  \Prob[ A \mid \mathcal{H}_t, B] \1{B}
\end{align*}
since $B$ is $\mathcal{H}_t$-measurable.
\end{proof}

The next Lemma shows how these two results are helpful in our scenario of Corollary 1.

\begin{lemma}
Let $A := \{ c_N(t) > 2/N^2 \}$. Let $B:= \{ \max_i w_t^{(i)} - \min_i w_t^{(i)} \geq 2\delta/N \}$. (Notice that these events satisfy the measurability properties in the previous Lemma.)
As an assumption in Corollary 1 we have that $\Prob[B \mid \mathcal{F}_{t-1}] \geq \zeta >0$ for infinitely many $t$.
We showed in the proof of Corollary 1 that $\Prob[A \mid \mathcal{H}_t, B] \geq \delta\varepsilon^3$.
Then, under this set-up, we have
$\Prob[A \mid \mathcal{F}_{t-1}] \geq \zeta\delta\varepsilon^3$ for infinitely many $t$.
\end{lemma}

\begin{proof}
\begin{align*}
\Prob[ A \mid \mathcal{F}_{t-1} ]
&= \E[ \1{A} \mid \mathcal{F}_{t-1} ]
= \E[ \E[ \1{A} \mid \mathcal{F}_{t-1}, B ]  \mid \mathcal{F}_{t-1} ]
= \E[ \Prob[ A \mid \mathcal{F}_{t-1}, B ]  \mid \mathcal{F}_{t-1} ]
= \E[ \E[ \Prob[ A \mid \mathcal{H}_t ] \mid \mathcal{F}_{t-1}, B ] \mid \mathcal{F}_{t-1} ] \\
&\geq \E[ \E[ \Prob[ A \mid \mathcal{H}_t, B ] \1{B} \mid \mathcal{F}_{t-1}, B ] \mid \mathcal{F}_{t-1} ]
= \E[ \E[ \Prob[ A \mid \mathcal{H}_t, B ] \mid \mathcal{F}_{t-1}, B ] \1{B} \mid \mathcal{F}_{t-1} ] \\
&\geq \E[ \E[ \delta\varepsilon^3 \mid \mathcal{F}_{t-1}, B ] \1{B} \mid \mathcal{F}_{t-1} ]
= \E[ \delta\varepsilon^3 \1{B} \mid \mathcal{F}_{t-1} ]
= \delta\varepsilon^3 \E[ \1{B} \mid \mathcal{F}_{t-1} ]
= \delta\varepsilon^3 \Prob[ B \mid \mathcal{F}_{t-1} ] \\
&\geq \zeta\delta\varepsilon^3 \text{ for infinitely many $t$}.
\end{align*}
\end{proof}


%%%----------------------------------------
DUMPED FROM PAPER:

\begin{proof}
Let $\mathcal{H}_t$ be defined as in \eqref{eq:defn_Ht}. The first step is to show that whenever $\max_i w_t^{(i)} \geq (1+\delta)/N$, $\Prob\{ c_N(t) > 2/N^2 | \mathcal{H}_t \} = \Prob\{ c_N(t) \neq 0 | \mathcal{H}_t \} \geq \delta\varepsilon^3$.
For this purpose we need consider only weight vectors such that $w_t^{(i)} \in (0,2/N)$ for all $i$; otherwise $\Prob\{ c_N(t) \neq 0 | \mathcal{H}_t \} =1$ by the definition of stochastic rounding.

Denote $\mathcal{S}_{N-1}^\delta = \{ w^{(1:N)} \in \mathcal{S}_{N-1} :  \forall i, \, 0 <w^{(i)} <2/N ;\, \max_i w^{(i)} \geq (1 + \delta)/N \}$ for any $\delta \in (0, 1)$, where $\mathcal{S}_{k}$ denotes the $k$-dimensional simplex.
Fix arbitrary $w_t^{(1:N)} \in \mathcal{S}_{N-1}^\delta$. Set $i^\star = \arg\max_i w_t^{(i)}$ and denote $\mathcal{I} = \{i \in \{1,\dots,N\} : w^{(i)} > 1/N \}$.
Since all weights are in $(0, 2/N)$, for $i \in \mathcal{I}, \nu_t^{(i)} \in \{1,2\}$ and for $i \notin \mathcal{I}, \nu_t^{(i)} \in \{0,1\}$; and since the offspring counts must sum to $N$, we can write
\begin{align*}
\Prob\{ c_N(t) \leq 2/N^2 | \mathcal{H}_t \}
&= \Prob( \nu_t^{(i)} =1 \,\forall i\in\{1,\dots,N\} | \mathcal{H}_t )\\
&= \Prob( \nu_t^{(i)} =1 \,\forall i\in \mathcal{I} | \mathcal{H}_t ) \\
&= \prod_{i \in \mathcal{I}} \Prob( \nu_t^{(i)} =1 | \nu_t^{(j)}=1 \,\forall j \in \mathcal{I}: j<i; \mathcal{H}_t ) \\
&= \Prob( \nu_t^{(i^\star)} =1 | \mathcal{H}_t ) \prod_{\substack{i \in \mathcal{I} \\ i \neq i^\star}} \Prob( \nu_t^{(i)} =1 | \nu_t^{(i^\star)}=1; \nu_t^{(j)}=1 \,\forall j \in \mathcal{I}: j<i ; \mathcal{H}_t ) \\
&\leq \Prob( \nu_t^{(i^\star)} =1 | \mathcal{H}_t ) .
\end{align*}
The final inequality holds with equality when $|\mathcal{I}| =1$, i.e.\ the only weight larger than $1/N$ is $w_t^{(i^\star)}$.
Thus $\Prob\{ c_N(t) > 2/N^2 | \mathcal{H}_t \}$ is minimised on $\mathcal{S}_{N-1}^\delta$ when only one weight is larger than $1/N$, in which case the values of the other weights do not affect this probability. 

Define $w_{\delta^\prime} = \{(1,\dots,1) + \delta^\prime e_{i^\star} - \delta^\prime e_{j^\star} \} /N$ for fixed $i^\star \neq j^\star$ and $\delta^\prime \in (0,1)$, where $e_i$ denotes the $i$th canonical basis vector in $\mathbb{R}^N$. 
Using the lower bound \eqref{eq:SR_p1_UL_bounds} on $p_1(w_t^{(i)}) = \Prob(\nu_t^{(i)} = \flnw +1 \mid \mathcal{H}_t)$,
\begin{align*}
\Prob\{ c_N(t) \leq 2/N^2 \mid \mathcal{H}_t, w_t^{(1:N)} = w_{\delta^\prime} \}
&= \Prob( \nu_t^{(i^\star)} = 1 \mid \mathcal{H}_t, w_t^{(1:N)} = w_{\delta^\prime} ) \\
&= 1-p_1\{ (1+\delta^\prime)/N \}
\leq 1- \delta^\prime \varepsilon^3 .
\end{align*}
We conclude that $\Prob\{ c_N(t) > 2/N^2 | \mathcal{H}_t, \max_i w_t^{(i)} \geq (1+\delta)/N \} \geq \min_{\delta^\prime \geq \delta} \delta^\prime\varepsilon^3 = \delta\varepsilon^3$.

A slight modification of this argument yields $\Prob\{ c_N(t) > 2/N^2 | \mathcal{H}_t, \min_i w_t^{(i)} \leq (1-\delta)/N \} \geq \delta\varepsilon^3$.
Whenever $\max_i w_t^{(i)} - \min_i w_t^{(i)} \geq 2\delta/N$, either $\max_i w_t^{(i)} \geq (1+\delta)/N$ or $\min_i w_t^{(i)} \leq (1-\delta)/N$, so we have 
$\Prob\{ c_N(t) > 2/N^2 | \mathcal{H}_t, \max_i w_t^{(i)} - \min_i w_t^{(i)} \geq 2\delta/N \} \geq \delta\varepsilon^3$.
Thus $ \Prob\{ c_N(t)>2/N^2 \mid \mathcal{H}_t \} \geq \delta\varepsilon^3 \1{\max_i w_t^{(i)} - \min_i w_t^{(i)} \geq 2\delta/N}$.
Using the D-separation established in Appendix \ref{app:dseparation} combined with the tower property, we have
\begin{equation*}
\Prob\{ c_N(t)>2/N^2 \mid \mathcal{F}_{t-1} \}
=\Et[ \Prob\{ c_N(t)>2/N^2 \mid \mathcal{H}_t \} ] 
\geq \delta\varepsilon^3 \Prob( \max_i w_t^{(i)} - \min_i w_t^{(i)} \geq 2\delta/N \mid \mathcal{F}_{t-1} ) ,
\end{equation*}
which is bounded below by $ \zeta \delta \varepsilon^3 $ for infinitely many $t$. 
Hence,
\begin{equation*}
\sum_{t=0}^\infty \Prob\{ c_N(t) > 2/N^2 \mid \mathcal{F}_{t-1} \} = \infty .
\end{equation*}
By a filtered version of the second Borel--Cantelli lemma \citep[see for example][Theorem 4.3.4]{durrett2019}, this implies that $c_N(t) >2/N^2$ for infinitely many $t$, almost surely.
This ensures, for all $t <\infty$, that $\Prob\left\{ \exists s<\infty : \sum_{r=1}^s c_N(r) \geq t \right\} =1$, which by definition of $\tau_N(t)$ is equivalent to $\Prob\{ \tau_N(t) = \infty \} =0$.
\end{proof}

\bibliography{../smc.bib}
\end{document}