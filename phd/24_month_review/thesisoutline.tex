\documentclass{article}
\usepackage[utf8]{inputenc}
\usepackage[margin=2.5cm]{geometry}

% bibliography
\usepackage[round, sort&compress]{natbib}
\usepackage{har2nat}
\bibliographystyle{agsm}

\title{Thesis Outline at 30 months}
\author{Suzie Brown}
\date{8 April 2020}

\begin{document}
\maketitle

\section{Introduction}

\section{Background (done)}
\subsection{Interacting Particle Systems}
General description of what an IPS is, what they are useful for, and how to simulate one. SMC as a specific class of IPS. Relevant background about SMC.
\subsection{Coalescent Theory}
Review of literature from population genetics, including the relevant population models and results about the corresponding coalescent processes.
\subsection{SMC Genealogies}
Description of how genealogies are induced by SMC algorithms and how this is related to the performance of the algorithms (e.g. variance estimation). Existing results characterising these genealogies.

\section{Resampling (done / in progress)} % could be its own chapter if I have a lot of material
Tour of resampling schemes and other ways to possibly improve performance by varying the resampling step, e.g. adaptive resampling. Existing results and conjectures comparing the performance of different schemes. Introduction of stochastic rounding as a class of resampling schemes. Implementation and usage in practice. 

\section{Limiting Coalescents for SMC Genealogies (done)}
\subsection{General Result for IPSs}
A refinement of \citet[Theorem 1]{koskela2018} with more tractable conditions. Proof of the theroem. 
\subsection{Application to Multinomial Resampling}
Corollary for multinomial resampling, as the simplest case.
\subsection{Application to Stochastic Rounding-based Resampling}
Proof that the theorem holds for SMC with stochastic rounding-based resampling, and interpretation of that result including comparison on time scale versus multinomial resampling.

\section{Conditional SMC}
Introduction of conditional SMC as a key component of particle MCMC / particle Gibbs, and what type of problem this is useful for. Proof that the theorem holds for conditional SMC and interpretation of this result.
\subsection{Pre-limiting behaviour (possible future work)}
Discussion of behaviour in the pre-limiting regime, supported by simulation studies.
\subsection{Ancestor Sampling (done)}
Motivate why ancestor sampling is useful in PG. Description of algorithm. When is it possible or not? Interpretation/comparison in coalescent framework.

\section{Stronger Mode of Convergence (possible future work)}
So far I only proved convergence in the sense of finite-dimensional distributions. In SMC applications we are typically interested in expectations of test functions, so an upgrade to weak convergence is desirable. The required tightness argument has been set out in the case of neutral IPSs \citep{mohle1999}, but an extension to the non-neutral case required for application to SMC is non-trivial. I will attempt to extend the techniques of \citet{mohle1999} and \citet{gerber2017} to prove weak convergence for these SMC algorithms.

%\section{Variance Estimation for SMC Algorithms (future work)}
%Variance estimators for SMC are intimately linked to the genealogies of samples (e.g.\ \citet{chan2013}, \citet{olsson2019}). Analysis of genealogies can therefore provide some insights into this area in terms of tuning and analysing the performance of existing estimators and perhaps proposing new estimators based on proven genealogical properties. I will review the variance estimation literature to find out exactly how this is related to genealogies, then see what results I could contribute to this area.

\section{Conclusions / Discussion}

\bibliography{../latex/smc.bib}
\end{document}