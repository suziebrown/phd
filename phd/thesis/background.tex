\chapter{Background}

\epigraph{
Anyone who considers arithmetical methods of producing random digits is, of course, in a state of sin.
}
% For, as has been pointed out several times, there is no such thing as a random number --- there are only methods to produce random numbers, and a strict arithmetic procedure of course is not such a method.
{\textsc{John von Neumann}}


\section{Sequential Monte Carlo}

\subsection{Motivation}
Being Bayesian. SSMs/HMMs. Example(s) of SSM (1D train?).

\subsection{Inference in SSMs}
What quantities do we want to infer? Why is this generally difficult? Filtering, prediction, smoothing, likelihood/normalising constant.

\subsection{Exact solutions}
Which SSMs are tractable? Kalman filter, extended KF, unscented KF, other conjugate models.

\subsection{Feynman-Kac models}
Define a generic FK model. Show that this class includes all SSMs. Example of non-SSM that is FK?

\subsection{Sequential Monte Carlo for Feynman-Kac models}
Present generic algorithm. State the SMC estimators of the quantities of interest.

\subsection{Theoretical justification}
How come SMC works? Convergence results (briefly!) e.g. CLT.


\section{Coalescent theory}

\subsection{Kingman's coalescent}
Define the $n$-coalescent, and Kingman's coalescent as extension of it. (Do I need to introduce random partitions first?)

\subsection{Properties}
Properties of Kingman's coalescent / $n$-coalescent. Distributions of branch length, waiting times, time to MRCA.

\subsection{Models in population genetics}
Talk about KC's ``domain of attraction''. Introduce Wright-Fisher model, Moran model, Cannings models.

\subsection{Particle populations}
Particles = individuals, iterations = generations. In what ways is SMC like a population model? (constant population size, non-overlapping generations, discrete time). In what ways is SMC not like a population model? (non-neutral, non-Markov?)


\section{Sequential Monte Carlo genealogies}

\subsection{From particles to genealogies}
How does the SMC algorithm induce a genealogy? (resampling = parent-child relationship). 

\subsection{Performance}
How do genealogies affect performance? Variance (and variance estimation?), storage cost. Ancestral degeneracy.

\subsection{Mitigating ancestral degeneracy}
Low-variance resampling (save details for next section). Adaptive resampling: idea of balancing weight/ancestral degeneracy; rule of thumb for implementing it; when is it effective or not?; necessary changes to our generic SMC algorithm (calculation of weights in particular). Backward sampling: when is it possible to do this?

\subsection{Asymptotics}
Why are large population asymptotics useful? Existing results (path storage, KJJS).


\section{Resampling}

\subsection{Definition}
The job of resampling (map weights to counts). Define ``valid'' resampling schemes (the three rules). Counter-examples where these rules are violated (the examples I've mentioned in previous writings, plus optimal transport resampling and that one FC told me about recently).

\subsection{What makes a good resampling scheme?}
Low-variance: variance of what? Different criteria/ definitions of optimality. Negative association. Link back to adaptive resampling: interaction between adaptive and low-variance resampling.

\subsection{Examples}
Tour of the key resampling schemes (multinomial, residual-*, stratified, systematic). Comparicon of properties of these, existing results comparing schemes. Implementation considerations. Theoretical justification (or lack of).

\subsection{Stochastic rounding}
Define stochastic rounding. Resampling schemes contained by this class. General properties for this class (marginal distributions, negative association, minimum-variance).


\section{Conditional SMC}

\subsection{Particle MCMC}
Motivate particle MCMC methods. 

\subsection{Particle Gibbs algorithm}
Present particle Gibbs algorithm. Explain why CSMC is required within particle Gibbs.

\subsection{Ancestor sampling}
Algorithm (or required changes to generic algorithm). Relation to backward sampling. When can it be implemented? Effect on performance (when is it effective?).
