\documentclass[fleqn]{article}
\usepackage[utf8]{inputenc}
\usepackage[margin=2cm]{geometry}

\usepackage{enumitem}

% maths
\usepackage{amsmath}
\usepackage{amssymb}
\usepackage{amsthm}

% pseudocode
\usepackage{algorithmicx}
\usepackage{algpseudocode}

% bibliography
\usepackage[round, sort&compress]{natbib}
\usepackage{har2nat}
\bibliographystyle{agsm}

% project-specific commands
\newcommand{\Cat}{\operatorname{Categorical}}
\newcommand{\Unif}{\operatorname{Uniform}}

\title{Conditional SMC definition}
\author{Suzie Brown}
\date{\today}

\begin{document}
\maketitle

Aim: to settle on a definition of conditional SMC with multinomial resampling from which to base results, preferably a definition that is compatible with the standing assumption of \citet{koskela2018}.

The quantities we know upfront are described below. The setting of this algorithm within particle MCMC \citep{andrieu2010} is as a single run of the particle filter within one step of the MCMC algorithm: the `immortal trajectory' on which we are conditioning is sampled from the preceding step.

\begin{description}[style=multiline,leftmargin=2.5cm]
\item [$N$] number of particles
\item [$T$] number of SMC iterations
\item [$\{K_t\}_{t=1,\dots,T}$] Markov kernels
\item [$\{g_t\}_{t=0,\dots,T}$] potentials
\item [$\mu$] initial distribution
\item [$x_{0:T}^*$] positions of immortal trajectory
\item [$a_{0:T}^*$] indices of immortal trajectory
\end{description}

\hrule
\begin{algorithmic}[1]
\Require $N, T, \mu, \{K_t\}, \{g_t\}, x_{0:T}^*, a_{0:T}^*$
\For{$i \in \{1,\dots,N\}$} 
	\State Sample $X_0^{(i)} \sim \mu$ 
\EndFor
\State $X_0^{(a_0^*)} \gets x_0^*$
\For{$i \in \{1,\dots,N\}$}
	\State $w_0^{(i)} \gets \frac{g_0(X_0^{(i)})}{\sum_{j=1}^N g_0(X_0^{(j)})}$
\EndFor
\For{$t \in \{0,\dots, T-1\}$}
	\State Sample $a_t^{(1:N)} \sim \Cat(\{1,\dots,N\}, w_t^{(1:N)})$
	\State $a_t^{(a_{t+1}^*)} \gets a_t^*$
	\For{$i \in \{1,\dots,N\}$}
		\State Sample $X_{t+1}^{(i)} \sim K_{t+1}(X_t^{(a_t^{(i)})}, \cdot)$
	\EndFor
	\State $X_{t+1}^{(a_{t+1}^*)} \gets X_{t+1}^*$
	\For{$i \in \{1,\dots,N\}$}
		\State $w_{t+1}^{(i)} \gets \frac{g_{t+1}(X_t^{(a_t^{(i)})} , X_{t+1}^{(i)})}{\sum_{j=1}^N g_{t+1}(X_t^{(a_t^{(j)})} , X_{t+1}^{(j)})}$
	\EndFor
\EndFor
\end{algorithmic}
\hrule

\begin{itemize}
\item Lines 9--10 okay since marginal of Categorical distribution is still Categorical (so we can sample all N indices and just overwrite the immortal particle's parent index).
\item This algorithm assumes the indices of the immortal line are given, as well as its positions. This is the situation described in \citet{andrieu2010}.
\item For our purposes we can therefore suppose that the immortal line comprises particle 1 in each generation.
\item This can be achieved even in a real scenario by simply relabelling the particles after resampling so that the immortal particle takes the label 1.
\end{itemize}

However, this method for the resampling is not consistent with the standing assumption because, given the immortal indices, the assignment of offspring is biased towards the (fixed) immortal parent.
We can fix this by considering a slightly different framework (which can just as well be applied in practice) where the immortal index is not pre-specified, but is instead sampled uniformly at each generation.\\

\hrule
\begin{algorithmic}[1]
\Require $N, T, \mu, \{K_t\}, \{g_t\}, x_{0:T}^*$
\For{$i \in \{1,\dots,N\}$} 
	\State Sample $X_0^{(i)} \sim \mu$ 
\EndFor
\State Sample $a_0^* \sim \Unif(\{1,\dots,N\})$
\State $X_0^{(a_0^*)} \gets x_0^*$
\For{$i \in \{1,\dots,N\}$}
	\State $w_0^{(i)} \gets \frac{g_0(X_0^{(i)})}{\sum_{j=1}^N g_0(X_0^{(j)})}$
\EndFor
\For{$t \in \{0,\dots, T-1\}$}
	\State Sample $a_t^{(1:N)} \sim \Cat(\{1,\dots,N\}, w_t^{(1:N)})$
	\State Sample $a_{t+1}^* \sim \Unif(\{1,\dots,N\})$
	\State $a_t^{(a_{t+1}^*)} \gets a_t^*$
	\For{$i \in \{1,\dots,N\}$}
		\State Sample $X_{t+1}^{(i)} \sim K_{t+1}(X_t^{(a_t^{(i)})}, \cdot)$
	\EndFor
	\State $X_{t+1}^{(a_{t+1}^*)} \gets X_{t+1}^*$
	\For{$i \in \{1,\dots,N\}$}
		\State $w_{t+1}^{(i)} \gets \frac{g_{t+1}(X_t^{(a_t^{(i)})} , X_{t+1}^{(i)})}{\sum_{j=1}^N g_{t+1}(X_t^{(a_t^{(j)})} , X_{t+1}^{(j)})}$
	\EndFor
\EndFor
\end{algorithmic}
\hrule

\bibliography{smc.bib}
\end{document}

