\documentclass{article}
\usepackage[utf8]{inputenc}
\usepackage[margin=2cm]{geometry}

\usepackage{enumitem}

% custom header/footer
\usepackage{fancyhdr}
\pagestyle{fancy}
\renewcommand{\headrulewidth}{0pt}
\fancyhf{}
\rfoot{\textsf{\thepage}}
\lfoot{\textsf{Suzie Brown}}

% bibliography
\usepackage[round, sort&compress]{natbib}
\usepackage{har2nat}
\bibliographystyle{agsm}

% maths
\usepackage{amsmath}
\usepackage{amssymb}
\usepackage{amsthm}
\newtheorem{thm}{Theorem}
\newtheorem{lemma}{Lemma}
\newcommand{\PR}{\mathbb{P}}
\newcommand{\E}{\mathbb{E}}
\newcommand{\V}{\operatorname{Var}}
\newcommand{\Mn}{\operatorname{Multinomial}}

\title{Non-triviality condition (shortened version)}
\author{Suzie Brown}

\begin{document}
\maketitle
\thispagestyle{fancy}

The following theorem will be used in each section. It is a version of the second Borel--Cantelli lemma, which can be found for instance in [Durrett, Theorem 4.3.4 - make citation].
\begin{thm}
Let$ (\mathcal{F}_n)_{n\geq 0}$ be a filtration with $\mathcal{F}_0 = \{\emptyset, \Omega\}$. Let $(B_n)_{n\geq 0}$ be a sequence of events such that $B_n \in \mathcal{F}_n$ for all $n$.
Then the events $\{ B_n \, i.o. \}$ and $\{ \sum_{n=1}^\infty \PR[B_n \mid \mathcal{F}_{n-1} ] =\infty \}$ are almost surely equal.
\end{thm}


\section*{Multinomial resampling}

\begin{lemma}\label{lem:neutral_cN_LB}
For all $N\geq 2$, for all $t$,
\begin{equation*}
\PR \left[c_N(t) > \frac{2}{N^2} \middle| \mathbf{w}=\left( \frac{1}{N}, \dots, \frac{1}{N} \right) \right] 
= 1- \frac{N!}{N^N}.
\end{equation*}
\end{lemma}

\begin{proof}
Fix arbitrary $t$ and $N\geq 2$. Since $2/(N)_2 > 2/(N^2)$ is the smallest possible non-zero value for $c_N(t)$,
\begin{align*}
\PR \left[c_N(t) > \frac{2}{N^2} \mid \mathbf{w}=(1/N, \dots, 1/N) \right]
&= 1- \PR[c_N(t) = 0  \mid \mathbf{w}=(1/N, \dots, 1/N)] \\
&= 1- \PR[\nu_t^{(1:N)} = (1,\dots, 1) \mid \mathbf{w}=(1/N, \dots, 1/N)].
\end{align*}
Conditional on the weights, $\nu_t^{(1:N)} \sim \Mn(N, (1/N, \dots, 1/N))$, so the probability of interest is
\begin{equation*}
\PR[\nu_t^{(1:N)} = (1,\dots, 1) \mid \mathbf{w}=(1/N, \dots, 1/N)] =
N! \prod_{i=1}^N \frac{1}{N}
= \frac{N!}{N^N}.
\end{equation*}
\end{proof}


\begin{lemma}\label{thm:nontrivial_mn_optimalw}
In the neutral case (i.e.\ when all weights are equal at every time step) with multinomial resampling, there exists $N_0$ such that for all $N>N_0$, for all finite $t$, $\PR[\tau_N(t) = \infty] =0$.
\end{lemma}

\begin{proof}
Let us rewrite the event of interest in a different way.
\begin{align*}
\PR[\tau_N(t) = \infty] =0 &\Leftrightarrow \PR[\tau_N(t) < \infty] =1 \\
&\Leftrightarrow \PR\left[ \min \left\{s>1 : \sum_{r=1}^s c_N(r) >t \right\} < \infty \right] =1 \\
&\Leftrightarrow \PR\left[ \exists s<\infty : \sum_{r=1}^s c_N(r) >t \right] =1
\end{align*}
It is sufficient to show that, for all $N>N_0$, $c_N(r)$ is bounded away from zero infinitely often in $r$.
We consider the sequence of events 
$E_r := \{ c_N(r) > 2/N^2 \}$ for $r \in \mathbb{N}$.
In the neutral case, the resampled family sizes at each generation are independent, hence the events $E_r$ are independent. 
By the second Borel-Cantelli lemma, $E_r$ occurs infinitely often if $\sum_{r=1}^\infty \PR(E_r) = \infty$. 
An expression for $\PR(E_r)$ is given in Lemma \ref{lem:neutral_cN_LB}. 
For any fixed $N\geq 2$, the probability is strictly positive and constant in $r$, so the Borel-Cantelli condition is satisfied, thus we conclude that $E_r$ occurs infinitely often.
Hence, taking $N_0=1$, we have that $\PR[\tau_N(t) = \infty] =0$ for all $N>N_0$ and all finite $t$, as required.
\end{proof}


\begin{lemma}\label{lem:mn_optimal_w}
For all $N\geq 2$, for all $t$, for any weight vector $(w_1, \dots, w_N) \in \mathcal{S}_{N-1}$,
\begin{equation*}
\PR \left[c_N(t) > \frac{2}{N^2} \middle| \mathbf{w}=(w_1, \dots, w_N) \right]
\geq \PR \left[c_N(t) > \frac{2}{N^2} \middle| \mathbf{w}=\left( \frac{1}{N}, \dots, \frac{1}{N} \right) \right] .
\end{equation*}
That is, the probability of having at least one merger is minimised by the vector of equal weights.
\end{lemma}

\begin{proof}
Fix arbitrary $t$ and $N\geq 2$. Recall that
\begin{equation}\label{eq:mn_nomerger_prob}
1-\PR \left[c_N(t) > \frac{2}{N^2} \mid \mathbf{w}=(w_1, \dots, w_N) \right]
= N! \prod_{i=1}^N w_i .
\end{equation}
We will show that the global maximum of this function on the simplex $\mathcal{S}_{N-1}$ is attained at $\mathbf{w}=(1/N,\dots,1/N)$.
This weight vector will therefore minimise the probability of the complementary event, implying the desired result.

First, since we are working on the simplex, we encode the constraint $\sum w_j =1$ by rewriting the function to optimise as
\begin{equation*}
f(\mathbf{w}) :=
\prod_{i=1}^N w_i
= \left(1- \sum_{j=1}^{N-1} w_j \right)\prod_{i=1}^{N-1} w_i 
\end{equation*}
where we have also dropped the constant positive factor $N!$. Note that this objective function is non-negative and obtains its minimal value zero whenever one or more of the weights is equal to zero; since we are looking for a maximum we can assume that $w_i >0$ for all $i$.
Now, for every $k \in \{1,\dots,N-1\}$, we solve
\begin{equation*}
\frac{\partial f(\mathbf{w})}{\partial w_k}
= \left(1- w_k - \sum_{j=1}^{N-1} w_j \right)\prod_{i\neq k}^{N-1} w_i 
=0 .
\end{equation*}
The product over $i \neq k$ is constant and positive for each $k$, so this amounts to solving
\begin{equation*}
w_k = 1- \sum_{j=1}^{N-1} w_j = w_N
\end{equation*}
for all $k$.
The unique solution is
$w_1 = w_2= \dots =  w_N = 1/N$.

To verify that the critical point is a maximum, we evaluate the Hessian $H$:
\begin{align*}
H_{kl}(\mathbf{w})
&= \begin{cases}
-2 \prod_{i\neq k}^{N-1} w_i & k=l \\
\left( 1 - w_k - w_l - \sum_{j=1}^{N-1} w_j \right)\prod_{i\neq k,l}^{N-1}w_i & k\neq l
\end{cases} \\
H_{kl}(1/N, \dots, 1/N)&= \begin{cases}
-2 \left(\frac{1}{N}\right)^{N-2} & k=l \\
- \left(\frac{1}{N}\right)^{N-2} & k\neq l
\end{cases}
\end{align*}
and show that $H$ is negative definite at $(1/N, \dots, 1/N)$: for any $\mathbf{x} \in \mathbb{R}^{N-1} \setminus \{\mathbf{0}\}$,
\begin{align*}
\mathbf{x}^T H\left(\frac{1}{N}, \dots, \frac{1}{N}\right) \mathbf{x} &= \sum_{k=1}^{N-1} \left[ -2\left(\frac{1}{N}\right)^{N-2} x_k^2
- \sum_{l\neq k}^{N-1} \left(\frac{1}{N}\right)^{N-2} x_k x_l \right] 
= \left(\frac{1}{N}\right)^{N-2} \left[ -\sum_{k=1}^{N-1} 2x_k^2 - \sum_{k=1}^{N-1} \sum_{l\neq k}^{N-1} x_k x_l \right] \\
&= \left(\frac{1}{N}\right)^{N-2} \left[ -\sum_{k=1}^{N-1} x_k^2 - \sum_{k=1}^{N-1} \sum_{l=1}^{N-1} x_k x_l \right]
= \left(\frac{1}{N}\right)^{N-2} \left[ - \sum_{k=1}^{N-1} x_k^2 - \left(\sum_{k=1}^{N-1} x_k \right)^2 \right]
< 0 .
\end{align*}
\end{proof}

\begin{thm}
With multinomial resampling, conditional on any sequence of weight vectors $\mathbf{w}_r^{(1:N)} \in \mathcal{S}_{N-1}; r\in\mathbb{N}$, there exists $N_0$ such that for all $N>N_0$, for all finite $t$, $\PR[\tau_N(t) = \infty] =0$.
\end{thm}

\begin{proof}
As in Lemma \ref{thm:nontrivial_mn_optimalw}, denote the sequence of events 
$E_r := \{ c_N(r) > 2/N^2 \}$ for $r \in \mathbb{N}$.
We know from Lemma \ref{thm:nontrivial_mn_optimalw} that, in the neutral case, $E_r$ occurs infinitely often. Lemma \ref{lem:mn_optimal_w} tells us that 
$\PR[E_r \mid \mathbf{w}=(w_1, \dots, w_N)] \geq \PR[E_r \mid \mathbf{w}=(1/N, \dots, 1/N)]$
for all $r$. 
Therefore, by a coupling argument, we conclude that $E_r$ occurs infinitely often in the non-neutral case as well.
\end{proof}


\section*{Conditional SMC with multinomial resampling}

Define $\mathbf{w}^* := \frac{1}{N-1} \left[ (1,\dots,1) - \mathbf{e}_{i^*} \right]$, where $i^*$ is the immortal index at generation $t$, and $\mathbf{e}_i$ denotes the $i^{th}$ canonical basis vector.
 
\begin{lemma}\label{lem:csmc_cN_LB}
For all $N\geq 2$, for all $t$,
\begin{equation*}
\PR \left[c_N(t) > \frac{2}{N^2} \middle| \mathbf{w}=\mathbf{w}^*  \right] 
= 1- \frac{(N-1)!}{(N-1)^{N-1}}
\end{equation*}
\end{lemma}

\begin{proof}
Under $\mathbf{w}^*$, the immortal parent has zero weight and is therefore assigned exactly one offspring (the immortal particle). The remaining $N-1$ offspring are assigned to the remaining $N-1$ parents according to a Multinomial distribution with equal weights. We therefore have
\begin{equation*}
\PR \left[c_N(t) > \frac{2}{N^2} \mid \mathbf{w}=\mathbf{w}^* \right]
= 1-\PR[\nu_t^{(1:N)} = (1,\dots, 1) \mid \mathbf{w}=\mathbf{w}^*] 
= 1- (N-1)! \prod_{i\neq i^*}^{N} \frac{1}{N-1}
= 1- \frac{(N-1)!}{(N-1)^{N-1}} .
\end{equation*}
\end{proof} 

 
\begin{lemma}\label{thm:nontrivial_csmc_optimalw}
In conditional SMC with multinomial resampling, in the optimal case where the weight vector is equal to $\mathbf{w}^*$ at every time step, there exists $N_0$ such that for all $N>N_0$, for all finite $t$, $\PR[\tau_N(t) = \infty] =0$.
\end{lemma}

\begin{proof}
The proof is exactly the same as for Lemma \ref{thm:nontrivial_mn_optimalw}; Lemma \ref{lem:csmc_cN_LB} provides the expression for $P(E_r)$ which is strictly positive and constant in $r$.
\end{proof}


\begin{lemma}\label{lem:csmc_optimal_w}
For all $N\geq 2$, for all $t$, for any weight vector $(w_1, \dots, w_N) \in \mathcal{S}_{N-1}$,
\begin{equation*}
\PR \left[c_N(t) > \frac{2}{N^2} \middle| \mathbf{w}=(w_1, \dots, w_N) \right]
\geq \PR \left[c_N(t) > \frac{2}{N^2} \middle| \mathbf{w}=\mathbf{w}^* \right].
\end{equation*}
\end{lemma}

\begin{proof}
Recall that
\begin{equation*}
1- \PR \left[c_N(t) > \frac{2}{N^2} \mid \mathbf{w}=(w_1, \dots, w_N) \right]
= (N-1)! \prod_{i\neq i^*}^{N} w_i .
\end{equation*}
Ignoring the immortal particles, this is equivalent to multinomial resampling in the standard case \eqref{eq:mn_nomerger_prob}, only with $N-1$ particles rather than $N$. 
As we saw in Lemma \ref{lem:mn_optimal_w}, this function is maximised at the vector of equal weights, in this case $\mathbf{w}_{-i^*}=\frac{1}{N-1} (1,\dots,1)$. 
This leaves zero weight for the immortal particle, so overall the maximum is attained at $\mathbf{w}^*=\frac{1}{N-1} \left[ (1,\dots,1) - \mathbf{e}_{i^*} \right]$ as required.
\end{proof}


\begin{thm}
In conditional SMC with multinomial resampling, conditional on any sequence of weight vectors $\mathbf{w}_r^{(1:N)} \in \mathcal{S}_{N-1}; r\in\mathbb{N}$, there exists $N_0$ such that for all $N>N_0$, for all finite $t$, $\PR[\tau_N(t) = \infty] =0$.
\end{thm}

\begin{proof}
As before, consider the sequence of events 
$E_r := \{ c_N(r) > 2/N^2 \}$ for $r \in \mathbb{N}$.
We know from the argument behind Lemma \ref{thm:nontrivial_csmc_optimalw} (which is completely analogous to Lemma \ref{thm:nontrivial_mn_optimalw}) that, in the case $\mathbf{w}=\mathbf{w}^*$, $E_r$ occurs infinitely often. Lemma \ref{lem:csmc_optimal_w} tells us that 
$\PR[E_r \mid \mathbf{w}=(w_1, \dots, w_N)] \geq \PR[E_r \mid \mathbf{w}=\mathbf{w}^*]$
for all $r$. 
Therefore, by a coupling argument, we conclude that $E_r$ occurs infinitely often in the general case as well.
\end{proof}


\section*{Stochastic rounding}
\begin{lemma} \label{lem:extreme_w_coal_as}
Let $\mathbf{w} = (w_1,\dots,w_N) \in \mathcal{S}_{N-1}$ and resample by stochastic rounding.
\begin{enumerate}[label=(\roman*)]
\item If $w_i \geq 2/N$ for some $i$, then $\PR[c_N(t) =0 |\mathbf{w} ] =0$. \label{item:SR_weight_2}
\item If $w_i= 0$ for some $i$, then $\PR[c_N(t) =0 |\mathbf{w} ] =0$. \label{item:SR_weight_0}
\end{enumerate}
\end{lemma}

\begin{proof}
In case \ref{item:SR_weight_2} particle $i$ is assigned at least two offspring, so $c_N(t)$ cannot be equal to zero.
In case \ref{item:SR_weight_0} particle $i$ is assigned zero offspring, so at least one other particle must be assigned more than one offspring, thus $c_N(t)$ cannot be equal to zero.
\end{proof}

The upshot of Lemma \ref{lem:extreme_w_coal_as} is that in these cases of ``extreme weights'' we have $c_N(t) > 2/N^2$ almost surely, so we can exclude these cases while we go about bounding $\PR[c_N(t) > 2/N^2 | \mathbf{w}]$ away from zero.


\begin{lemma}\label{lem:weps_cN_prob}
Define $\mathbf{w}^\delta := \frac{1}{N}\{(1,\dots,1) + \delta \mathbf{e}_i - \delta \mathbf{e}_j \}$ for any $i \neq j$ and $0< \delta < 1$. Then $\PR[c_N(t) > 2/N^2 | \mathcal{H}_t, \mathbf{w}_t = \mathbf{w}_\delta] \geq \delta \varepsilon^3$.
\end{lemma}

\begin{proof}
We use a bound on $\PR[ \nu_t^{(i)} = \lfloor N w_t^{(i)} \rfloor ]$ from the proof of Corollary 1 in the draft paper:
\begin{equation*}
\PR[ \nu_t^{(i)} = \lfloor N w_t^{(i)} \rfloor \mid \mathcal{H}_t] =: p_0 = 1-p_1 \leq 1 - (Nw_t^{(i)} - \lfloor N w_t^{(i)} \rfloor ) \varepsilon^{(2\lfloor N w_t^{(i)} \rfloor +1)} .
\end{equation*}
Then
\begin{align*}
\PR[c_N(t) \leq 2/N^2 | \mathcal{H}_t, \mathbf{w}_t = \mathbf{w}_\delta]
&= \PR[\nu_t^{(1:N)} = (1,\dots, 1) \mid \mathcal{H}_t, \mathbf{w}_t = \mathbf{w}_\delta] \\
&= \PR[\nu_t^{(i)} = 1, \nu_t^{(j)} = 1 \mid \mathcal{H}_t, \mathbf{w}_t = \mathbf{w}_\delta] \\
&= \PR[\nu_t^{(i)} = 1 \mid \mathcal{H}_t, \mathbf{w}_t = \mathbf{w}_\delta] \\
&\leq 1- (N w_\delta^{(i)} - \lfloor N w_\delta^{(i)} \rfloor) \varepsilon^{(2\lfloor N w_\delta^{(i)} \rfloor +1)} \\
&= 1- \{N(1+ \delta)/N - 1\}\varepsilon^3 \\
&= 1 - \delta\varepsilon^3 ,
\end{align*}
since the offspring counts are deterministically equal to one apart from particles $i$ and $j$, and it remains that $\nu_t^{(i)} = 1$ if and only if $\nu_t^{(j)} = 1$.
\end{proof}




\begin{lemma}
For any $\delta \in (0, 1)$, denote $\mathcal{S}_{N-1}^\delta := \{ \mathbf{w} \in \mathcal{S}_{N-1} :  \forall i, \, 0 <w_i <\frac{2}{N} ;\, \max_i w_i \geq \frac{1 + \delta}{N} \}$.
Then for all $\mathbf{w} \in \mathcal{S}_{N-1}^\delta$, 
$\PR[c_N(t) > 2/N^2 | \mathbf{w} ] \geq \PR[c_N(t) > 2/N^2 | \mathbf{w}_\delta ]$.
\end{lemma}

\begin{proof}
Fix arbitrary $\mathbf{w} \in \mathcal{S}_{N-1}^\delta$. Let $i^*$ be then index of the particle with the largest weight.
Denote $\mathcal{I} := \{i \in \{1,\dots,N\} : w_i > 1/N \}$.
Notice that 
\begin{equation*}
\PR[ c_N(t) \leq 2/N^2 | \mathbf{w} ] 
= \PR[ \nu_t^{(i)} =1 \,\forall i\in\{1,\dots,N\} | \mathbf{w}] 
= \PR[ \nu_t^{(i)} =1 \,\forall i\in \mathcal{I} | \mathbf{w}] .
\end{equation*}
This is true because all weights are in $(0, 2/N)$, so for $i \in \mathcal{I}, \nu_t^{(i)} \in \{1,2\}$, and for $i \notin \mathcal{I}, \nu_t^{(i)} \in \{0,1\}$; and the offspring counts must sum to $N$ (a generalisation of the argument used in Lemma \ref{lem:weps_cN_prob}).

We can then decompose this probability into a product of conditional probabilities:
\begin{align*}
\PR[ \nu_t^{(i)} =1 \,\forall i\in \mathcal{I} | \mathbf{w}]
&= \prod_{i \in \mathcal{I}} \PR[ \nu_t^{(i)} =1 | \nu_t^{(j)}=1 \,\forall j <i \in \mathcal{I}; \mathbf{w}] \\
&= \PR[\nu_t^{(i^*)} =1 | \mathbf{w}] \prod_{i \neq i^* \in \mathcal{I}} \PR[ \nu_t^{(i)} =1 | \nu_t^{(i^*)}=1; \nu_t^{(j)}=1 \,\forall j <i \in \mathcal{I}; \mathbf{w}] \\
&\leq \PR[\nu_t^{(i^*)} =1 | \mathbf{w}] .
\end{align*}
The last line is equal to the probability $\PR[ c_N(t) \leq 2/N^2 | \mathbf{w} ] $ in the case where $|\mathcal{I}| =1$, i.e.\ the only weight larger than $1/N$ is $w_{i^*}$.

In other words, $\PR[ c_N(t) > 2/N^2 | \mathbf{w} ]$ is minimised on $\mathcal{S}_{N-1}^\delta$ by having only one weight larger than $1/N$, in which case the values of the other weights do not affect this probability. 

We therefore find that a minimum of $\PR[ c_N(t) > 2/N^2 | \mathbf{w} ]$ on $\mathcal{S}_{N-1}^\delta$ is given by $\mathbf{w}_{\delta^\prime}$, for some $\delta^\prime \geq \delta$. 
It only remains to show that taking $\delta^\prime > \delta$ does not decrease the probability. This is a consequence of Lemma \ref{lem:weps_cN_prob}, where we see that $\PR[ c_N(t) > 2/N^2 | \mathbf{w}_{\delta^\prime}]$ is monotonically increasing in $\delta^\prime$.
Thus the minimum of $\PR[ c_N(t) > 2/N^2 | \mathbf{w} ]$ is attained at $\mathbf{w} = \mathbf{w}_\delta$, as required. (Although this minimum is not unique, we have shown explicitly that it is a global minimum on $\mathcal{S}_{N-1}^\delta$.)
\end{proof}

[Combining the above three Lemmata we see that, for any $\mathbf{w} \in \mathcal{S}_{N-1}$ such that $\max_i w_i \geq \frac{1 + \delta}{N}$, we have the bound $\PR[ c_N(t) > 2/N^2 | \mathbf{w} ] \geq \delta$. This will probably turn out to be useful, provided we can say something about whether we can expect the constraint $\max_i w_i \geq \frac{1 + \delta}{N}$ to hold.]

\begin{thm}
Consider a sequential Monte Carlo algorithm using any stochastic rounding as its resampling scheme.
If $\PR\{\max_i w_t^{(i)} \geq (1+\delta)/N\} >0$ for infinitely many $t$ then $\PR\{\tau_N(t) = \infty \}=0$ for all finite $t$.
\end{thm}


\end{document}