\documentclass{article}
\usepackage[utf8]{inputenc}
\usepackage[margin=2cm]{geometry}

\usepackage{graphicx}
\usepackage{enumitem}

% custom header/footer
\usepackage{fancyhdr}
\pagestyle{fancy}
\renewcommand{\headrulewidth}{0pt}
\fancyhf{}
\rfoot{\textsf{\thepage}}
\lfoot{\textsf{Suzie Brown}}

%annotations
\usepackage{color}
\usepackage{xspace}
\newcommand{\seb}[1]{\xspace\textcolor{red}{#1}\xspace}

% bibliography
\usepackage[round, sort&compress]{natbib}
\usepackage{har2nat}
\bibliographystyle{agsm}

% maths
\usepackage{amsmath}
\usepackage{amssymb}
\usepackage{amsthm}
\newtheorem{theorem}{Theorem}
\newtheorem{lemma}{Lemma}
\newcommand{\Prob}{\mathbb{P}}
\newcommand{\E}{\mathbb{E}}
\newcommand{\Et}{\mathbb{E}_t}
\newcommand{\V}{\operatorname{Var}}
\newcommand{\I}[1]{\mathbb{I}_{\{#1\}}}
\newcommand{\1}[1]{\mathbb{I}_{#1}}
\newcommand{\Mn}{\operatorname{Multinomial}}

\title{Weak convergence proof (in progress)}
\author{Suzie Brown}

\begin{document}
\maketitle
\thispagestyle{fancy}

\begin{theorem}\label{thm:weakconv}
Let $\nu_t^{(1:N)}$ denote the offspring numbers in an interacting particle system satisfying the standing assumption and such that, for any $N$ sufficiently large, $\Prob\{ \tau_N(t) = \infty \} =0$ for all finite $t$. Suppose that there exists a deterministic sequence $(b_N)_{N\geq1}$ such that ${\lim}_{N\to\infty} b_N =0$ and
\begin{equation}\label{eq:mainthmcond}
\frac{1}{(N)_3} \sum_{i = 1}^N \Et\{ (\nu_t^{(i)})_3 \}  \leq b_N \frac{1}{(N)_2} \sum_{i = 1}^N \Et\{ (\nu_t^{(i)})_2 \}
\end{equation}
for all $N$, uniformly in $t \geq 1$.
Then the rescaled genealogical process $(G_{\tau_N(t)}^{(n,N)})_{t\geq0}$ converges weakly to Kingman's $n$-coalescent as $N \to \infty$.
\end{theorem}

\begin{proof}
Define $p_t := \max_{\xi\in E} \{1 - p_{\xi\xi}(t)\} = 1 - p_{\Delta\Delta}(t)$, where $\Delta$ denotes the trivial partition of $\{1,\dots,n\}$ into singletons. For a proof that the maximum is attained at $\xi = \Delta$, see Lemma \ref{thm:maximum_pr}. 
Following \citet{mohle1999}, we now construct the two-dimensional Markov process $(Z_t, S_t)_{t \in \mathbb{N}}$ with transition probabilities
\begin{equation}
\Prob(Z_t = j , S_t = \eta \mid Z_{t-1} = i, S_{t-1} = \xi)
= \begin{cases}
1 - p_t &\quad \text{if } j=i \text{ and } \xi=\eta \\
p_{\xi\xi}(t) + p_t - 1  &\quad \text{if } j=i+1 \text{ and } \xi=\eta \\
p_{\xi\eta}(t) &\quad \text{if } j=i+1 \text{ and } \xi\neq\eta \\
0 &\quad \text{otherwise} .
\end{cases}
\end{equation}
The construction is such that the marginal $(S_t)$ has the same distribution as the genealogical process of interest, and $(Z_t)$ has jumps at all the times $(S_t)$ does plus some extra jumps. (The definition of $p_t$ ensures that the probability in the second case is non-negative, attaining the value zero when $\xi=\Delta$.)

Denote by $0=T_0^{(N)}<T_1^{(N)}<\dots$ the jump times of the rescaled process $(Z_{\tau_N(t)})_{t\geq0}$, and $\omega_i^{(N)} := T_i^{(N)} - T_{i-1}^{(N)}$ the corresponding holding times ($i\in\mathbb{N}$).

...
\end{proof}

\begin{lemma}\label{thm:maximum_pr}
$\max_{\xi\in E} (1 - p_{\xi\xi}(t)) = 1 - p_{\Delta\Delta}(t)$.
\end{lemma}
%%% This below proof is incorrect.
%\begin{proof}
%From the definition of $p_{\xi\eta}(t)$ \citep[Equation (1)]{koskela2018},
%\begin{equation}
%p_{\xi\xi}(t) = \frac{1}{(N)_{|\xi|}} \sum_{\substack{i_1,\dots,i_{|\xi|} \\ \text{all distinct}}} \nu_t^{(i_1)} \cdots \nu_t^{(i_{|\xi|})} .
%\end{equation}
%In the case $\xi = \Delta$, this simplifies to
%\begin{equation}
%p_{\Delta\Delta}(t) = \prod_{i=1}^N \nu_t^{(i)} 
%= \begin{cases}
%1 &\quad \text{if } \nu_t^{(1)}=\cdots=\nu_t^{(n)} \\
%0 &\quad \text{otherwise} .
%\end{cases}
%\end{equation}
%Whenever $p_{\Delta\Delta}(t) = 1$, $p_{\xi\xi}(t) = 1$ for all $\xi$ also. 
%Otherwise, $p_{\xi\xi}(t) \geq p_{\Delta\Delta}(t) = 0$ for all $\xi$, since $p_{\xi\xi}(t) \in [0,1]$.
%Hence $p_{\xi\xi}(t)$ attains its minimum at $\xi=\Delta$, and the result follows.
%\end{proof}
\begin{proof}
Consider any $\xi \in E$ consisting of $k$ blocks ($1\leq k\leq n-1$), and any $\xi^\prime\in E$ consisting of $k+1$ blocks. 
From the definition of $p_{\xi\eta}(t)$ \citep[Equation (1)]{koskela2018},
\begin{equation}
p_{\xi\xi}(t) = \frac{1}{(N)_k} \sum_{\substack{i_1,\dots,i_k \\ \text{all distinct}}} \nu_t^{(i_1)} \cdots \nu_t^{(i_k)} .
\end{equation}
Similarly,
\begin{align*}
p_{\xi^\prime\xi^\prime}(t) &= \frac{1}{(N)_{k+1}} \sum_{\substack{i_1,\dots,i_k, i_{k+1} \\ \text{all distinct}}} \nu_t^{(i_1)} \cdots \nu_t^{(i_k)} \nu_t^{(i_{k+1})} \\
&= \frac{1}{(N)_k(N-k)} \sum_{\substack{i_1,\dots,i_k \\ \text{all distinct}}} \left\{ \nu_t^{(i_1)} \cdots \nu_t^{(i_k)} \sum_{\substack{i_{k+1}=1 \\ \text{also distinct}}}^N \nu_t^{(i_{k+1})} \right\} .
\end{align*}
Discarding some zero summands,
\begin{equation}
p_{\xi^\prime\xi^\prime}(t) = \frac{1}{(N)_k(N-k)} \sum_{\substack{i_1,\dots,i_k \\ \text{all distinct:} \\ \nu_t^{(i_1)},\dots,\nu_t^{(i_k)} > 0 }} \left\{ \nu_t^{(i_1)} \cdots \nu_t^{(i_k)} \sum_{\substack{i_{k+1}=1 \\ \text{also distinct}}}^N \nu_t^{(i_{k+1})} \right\} .
\end{equation}
The inner sum is
\begin{equation}
\sum_{\substack{i_{k+1}=1 \\ \text{also distinct}}}^N \nu_t^{(i_{k+1})} =
\left\{ \sum_{i=1}^N \nu_t^{(i)} -  \sum_{i\in\{i_1,\dots,i_k\} } \nu_t^{(i)} \right\}
\leq N - k
\end{equation}
since $\nu_t^{(i_1)},\dots,\nu_t^{(i_k)} $ are all at least 1.
Hence
\begin{equation}
p_{\xi^\prime\xi^\prime}(t)
\leq  \frac{N-k}{(N)_k(N-k)} \sum_{\substack{i_1,\dots,i_k \\ \text{all distinct:} \\ \nu_t^{(i_1)},\dots,\nu_t^{(i_k)} > 0 }} \nu_t^{(i_1)} \cdots \nu_t^{(i_k)} 
= p_{\xi\xi}(t) .
\end{equation}
Thus $p_{\xi\xi}(t)$ is decreasing in the number of blocks of $\xi$, and is therefore minimised by taking $\xi = \Delta$, which achieves the maximum $n$ blocks. This choice in turn maximises $1-p_{\xi\xi}(t)$, as required.
\end{proof}



\bibliography{../smc.bib}
\end{document}