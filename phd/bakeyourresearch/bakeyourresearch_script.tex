\documentclass{article}
\usepackage[utf8]{inputenc}
\usepackage[margin=2.5cm]{geometry}

\begin{document}
\section*{Genealogies in Cakeland}
\begin{itemize}
\item To begin with, the population of Cakeland consists of just one lonely cupcake. Its DNA determines what type of cupcake it is: in this case ``chocolate orange'', which we will call genetic type A.
\item The time comes for the cupcake to reproduce. It passes its DNA on to its children, so they generally inherit the same type: in this case type A.
\item (Place two type A cupcakes)
\item But every now and then something goes wrong and a mutation occurs. This corrupts the DNA, so it results in a child of a different type.
\item (Place a type B cupcake)
\item (This) child has a \textbf{mutation}, and has come out ``lemon and raspberry'' flavoured. We'll call this type B.
\item Once it has reproduced, its three children make up the next generation of life in Cakeland.
\item So now the population consists of three cupcakes: two type A and one type B.
\item Not all cupcakes are lucky enough to survive to reproductive age. Some of them get eaten before they're able to have any children.
\item But not all cupcakes are created equal: one type may be more appealing to predators (that's you!) than others. A yummier-looking cupcake is more likely to be eaten before having children. In Cakeland, looking yummy is bad news!
\item (Get audience member to select a cake)
\item So (that) cupcake gets eaten, but the other two survive long enough to have children of their own. And as before, their children will mostly inherit their parent's type.
\item (Place a new generation of cakes...)
\item This means that, because we removed a type (?) cake from the population, that type is likely to be less prominent in the next generation.
\item This phenomenon is called \textbf{selection}. It is sometimes called \textbf{survival of the fittest}: in this case the fittest cupcakes are the least yummy-looking ones, because they are the most likely to survive.
\item If we ignore the types and just look at which cupcakes are the parents or children of each other, we can see the \textbf{genealogy}.
\item So we've seen that there are two mechanisms controlling the types of cakes in the population: \textbf{mutation} and \textbf{selection}.
\item In my research I'm looking at algorithms that use mutation and selection to approximate integrals by averaging over the types in a simulated population after a lot of generations.
\item I assume that the population is really big, and try to find out what the genealogy looks like.
\end{itemize}
\end{document}