\documentclass[fleqn]{article}
\usepackage[utf8]{inputenc}
\usepackage[margin=2cm]{geometry}

\usepackage{tikz}
\usetikzlibrary{positioning}
\usepackage{xcolor}

% maths
\usepackage{amsmath}
\usepackage{amssymb}
\usepackage{amsthm}

% bibliography
\usepackage[round, sort&compress]{natbib}
\usepackage{har2nat}
\bibliographystyle{agsm}

% custom header/footer
\usepackage{fancyhdr}
\pagestyle{fancy}
\renewcommand{\headrulewidth}{0pt}
\fancyhf{}
\rfoot{\textsf{\thepage}}
\lfoot{\textsf{Suzie Brown}}

% useful math symbols
\newcommand{\E}{\mathbb{E}}
\newcommand{\eqdist}{\overset{d}{=}}
\newcommand{\I}[1]{\mathbb{I}\{#1\}}
\newcommand{\indep}{\perp}

% project-specific commands
\newcommand{\F}{\mathcal{F}_{t-1}}
\newcommand{\Mn}{\operatorname{Multinomial}}
\newcommand{\vt}[2][t]{v_{#1}^{(#2)}}
\newcommand{\wt}[2][t]{w_{#1}^{(#2)}}
\newcommand{\wbar}[2][t]{\bar{w}_{#1}^{(#2)}}

\title{Asymptotic analysis of genealogies induced by sequential Monte Carlo algorithms}
\author{Suzie Brown}
\date{\today}

\begin{document}
\maketitle
\thispagestyle{fancy}

\section{Introduction}
\textcolor{red}{
- what is SMC and what is it used for?\\
- basic SMC algorithm\\
- the three main tasks: prediction, filtering, smoothing\\
- the problem with smoothing: ancestral vs. weight degeneracy\\
- motivating plots\\
- organisation of the report\\
}

[SMC is super useful for loads of stuff...]

[Define subscript notation (-t) and (s:t)]

\subsection{Class of models}
For our purposes it is sufficient to describe sequential Monte Carlo (SMC) algorithms in the context of inference on a time-homogeneous state space model.
Consider the following model:
\begin{align*}
& X_0 \sim \mu(\cdot) \\
& X_{t+1} \mid (X_t = x_t) \sim f(\cdot | x_t)  \qquad t=0,\dots,T-1 \\
& Y_t \mid (X_t = x_t) \sim g(\cdot | x_t) \qquad t=0,\dots,T
\end{align*}
where $(X_t)_{t=0}^T$ is an unobservable discrete-time Markov process and the observables $(Y_t)_{t=0}^T$ satisfy $Y_t \indep \{Y_{-t}, X_{-t}\} \mid X_t$. 
The conditional independence structure is described by the following graphical model.
\begin{center}
\begin{tikzpicture}
\node (yt) {$Y_t$};
\node (thet) [below=of yt] {$X_t$};
\node (yt1) [left=of yt] {$Y_{t-1}$};
\node (thet1) [below=of yt1] {$X_{t-1}$};
\node (dot1) [left=of thet1] {$\dots$};
\node (dot2) [right=of thet] {$\dots$};
\draw[->](thet.north)--(yt.south) node[midway, right] {\footnotesize{$g$}};
\draw[->](thet1.north)--(yt1.south) node[midway, right] {\footnotesize{$g$}};
\draw[->](thet1.east)--(thet.west) node[midway, above] {\footnotesize{$f$}};
\draw[->](dot1.east)--(thet1.west) node[midway, above] {\footnotesize{$f$}};
\draw[->](thet.east)--(dot2.west) node[midway, above] {\footnotesize{$f$}};
\end{tikzpicture}
\end{center}
We assume that the \emph{transition} and \emph{emission} kernels have densities which are denoted by $f$ and $g$ respectively, but this is not necessary in general.
We only require that we can sample from $\mu(\cdot)$ and $f(\cdot | x)$, and calculate \emph{unnormalised} potentials $g(y|x)$, for all $x,y$.

\subsection{Inference in state space models}
Suppose we are in a Bayesian setting, where $\mu$ is our prior distribution at time 0, observations $y_t$ arrive sequentially, and we want to infer information about the hidden states (either on- or off-line).
The three main inference problems are:
\begin{description}
\item[Filtering] (where is it now?) $p(x_{t} | y_{0:t})$
\item[Prediction] (where will it go next?) $p(x_{t+1} | y_{0:t})$
\item[Smoothing] (where has it been?) $p(x_{0:t} | y_{0:t})$
\end{description}
In the on-line setting, we take as our prior the posterior distribution from the previous time step $t-1$, and update it using the new observation $y_t$. The inference must be fast enough to keep up with the rate of arrival of observations, so in particular the complexity of our inference algorithm must not increase with $T$.
In the off-line setting, we take $\mu$ as the prior distribution, and infer the set of posteriors once all $T+1$ observations have arrived.

Prediction and filtering are essentially equivalent, because given a filtering distribution, the corresponding predictive distribution can be obtained by applying the transition kernel $f$.
Smoothing is considered a harder task because it requires us to infer many more parameters from the same amount of information; indeed the dimension of the problem increases linearly with $T$.
In general it is not possible to infer the smoothing distributions on-line, because this would require using all of the previous observations at each time step, so the complexity would increase at each step. Algorithms that exist for on-line smoothing [REFs] are typically either approximate or computationally expensive.

In the case of linear Gaussian state space models (i.e.\ where $f$ and $g$ are Gaussian densities that depend only linearly on their arguments), the posterior distributions of interest are available analytically, by way of the Kalman filter [REF] and Rauch-Tung-Striebel (RTS) smoother recursions [REF].
The other analytic case occurs if the state space of $(X_t)_{t=0}^\infty$ is finite, in which case the forward-backward algorithm [REF] yields the exact posteriors.


\section{SMC as a coalescent}
\textcolor{red}{
- k-coalescent \& Kingman coalescent\\
- pop gen literature about large population cts time limits of various models\\
- resampling viewed backwards in time: branching process $\to$ coalescent process\\
- asymptotic properties of SMC lit review: CLT, path storage, coalescence etc.\\
- the gap in knowledge that we aim to fill\\
- (remark: although SMC has other problems in high dimension, the coalescence rate doesn't depend on the dimension...)
}

\section{Conditional SMC}
\textcolor{red}{
- motivation: particle MCMC, need for multiple lineages\\
- conditional (multinomial) SMC algorithm and its context within particle Gibbs\\
- result: coalescence rate etc in terms of standard multinomial one; verification of assumptions of KJJS theorem (but exile horrible calculations to appendix)\\
- simulations \& conclusions thence
}

\section{Alternative resampling schemes}
\textcolor{red}{
- multinomial not really used in practice, but other schemes hard to analyse\\
- overview of the main variance-reducing schemes\\
- results: theorem for residual resampling (hopefully)\\
- maybe results for other schemes\\
- simulation comparing all of them, and conclusions thence
}

\section{Discussion}
\textcolor{red}{
- results so far\\
- impact of this work: to practitioners, to enriching the SMC literature, interpretation within pop gen.\\
- future directions
}


\bibliography{smc.bib}
\end{document}