\documentclass[fleqn]{article}
\usepackage[utf8]{inputenc}
\usepackage[margin=2.5cm]{geometry}

% bibliography
\usepackage[round, sort&compress]{natbib}
\usepackage{har2nat}
\bibliographystyle{agsm}

% custom header/footer
\usepackage{fancyhdr}
\pagestyle{fancy}
\renewcommand{\headrulewidth}{0pt}
\fancyhf{}
\rfoot{\textsf{\thepage}}
\lfoot{\textsf{Suzie Brown}}

% miscellaneous formatting
\usepackage{xcolor}
\usepackage[font=small]{caption}
\usepackage{subcaption}
\usepackage{enumitem}

% maths / theorems
\usepackage{amsmath}
\usepackage{amssymb}
\usepackage{amsthm}
\newtheorem{thm}{Theorem}
\newtheorem{lemma}{Lemma}
\newtheorem{remark}{Remark}
\newtheorem{corollary}{Corollary}
\newtheorem{conj}{Conjecture}
\newtheorem{prop}{Proposition}
\theoremstyle{definition}
\newtheorem{defn}{Definition}

% useful math symbols
\newcommand{\N}{\mathbb{N}}
\newcommand{\Prob}{\mathbb{P}}
\newcommand{\E}{\mathbb{E}}
\newcommand{\V}{\operatorname{Var}}
\newcommand{\eqdist}{\overset{d}{=}}
\newcommand{\I}[1]{\mathbb{I}_{\{#1\}}}
\newcommand{\Ntoinfty}{\overset{N\to\infty}{\longrightarrow}}
\newcommand{\limNtoinfty}{\underset{N\to\infty}{\lim}}
\newcommand\indep{\protect\mathpalette{\protect\independenT}{\perp}}
\def\independenT#1#2{\mathrel{\rlap{$#1#2$}\mkern2mu{#1#2}}}

% distributions
\newcommand{\Cat}{\operatorname{Categorical}}
\newcommand{\Unif}{\operatorname{Uniform}}
\newcommand{\Mn}{\operatorname{Multinomial}}
\newcommand{\Bin}{\operatorname{Binomial}}

% project-specific symbols
\newcommand{\F}{\mathcal{F}_{t-1}}
\newcommand{\vt}[2][t]{\nu_{#1}^{(#2)}}
\newcommand{\wt}[2][t]{w_{#1}^{(#2)}}
\newcommand{\wbar}[2][t]{\bar{w}_{#1}^{(#2)}}
\newcommand{\vttilde}[2][t]{\tilde{\nu}_{#1}^{(#2)}}
\newcommand{\flnw}{\lfloor N\wt{i} \rfloor }

\title{Stochastic roundings --- coalescent proof}
\author{Suzie Brown}
\date{\today}

\begin{document}
\maketitle
\thispagestyle{fancy}

\begin{defn}\label{defn:randround_1D}
Let $X\geq0$. A random variable $Y: \mathbb{R}_+ \to \mathbb{N}$ is a \emph{stochastic rounding} of $X$ if $Y$ takes the values
\begin{equation*}
Y=
\begin{cases}
 \lfloor X \rfloor & \text{with probability } 1- X+ \lfloor X \rfloor \\
  \lfloor X \rfloor +1 & \text{with probability } X- \lfloor X \rfloor 
\end{cases}
\end{equation*}
\end{defn}

\begin{thm}
Under the time scaling of \citet[Theorem 1]{koskela2018} and the conditions of \citet[Lemma 3]{koskela2018}, genealogies of SMC algorithms with stochastic rounding-based resampling schemes converge to Kingman's $n$-coalescent in the sense of finite-dimensional distributions as $N\to\infty$.
\end{thm}

\begin{proof}
We need to show that there exists a deterministic sequence $(b_N)_{N\in\N}$ such that $\limNtoinfty b_N =0$ and
\begin{equation}\label{eq:mn_ratiobound}
\frac{\frac{1}{(N)_3}\sum_{i=1}^N \E[(\vt{i})_3 |\F]}{\frac{1}{(N)_2} \sum_{i=1}^N \E[(\vt{i})_2 |\F]} \leq b_N
\end{equation}
for all $N \in \N$. 
Directly applying Definition \ref{defn:randround_1D}, we find for the denominator
\begin{align*}
\E[(v_i)_2^{(r)} | w_i] &= \flnw (\flnw -1) (1 - Nw_i + \flnw) + (\flnw +1) \flnw (Nw_i - \flnw) \\
&= \flnw \left( 2(Nw_i - \flnw) + \flnw -1 \right)
\end{align*}
For the numerator we find
\begin{align*}
\E[(\vt{i})_3 |\wt{1:N}] 
&= (\flnw)_3(1-N\wt{i} + \flnw) + (\flnw +1)_3(N\wt{i} - \flnw) \\
&= (\flnw)_2 \left\{ (\flnw -2)(1-N\wt{i} + \flnw) + (\flnw +1)(N\wt{i} - \flnw)\right\} \\
&= (\flnw)_2 \left( 3N\wt{i} - 2\flnw -2 \right)
\end{align*}
So for the ratio we have
\begin{align*}
\frac{\frac{1}{(N)_3}\sum_{i=1}^N \E[(\vt{i})_3 |\F]}{\frac{1}{(N)_2} \sum_{i=1}^N \E[(\vt{i})_2 |\F]}  
&= \frac{1}{N-2} \frac{\sum_{i=1}^N \E[(\flnw)_2(3N\wt{i} - 2\flnw -2) |\F]}{ \sum_{i=1}^N \E[\flnw(2N\wt{i} - \flnw -1) |\F]} \\
&\leq \frac{1}{N-2} \frac{\sum_{i=1}^N \E[(\flnw)_2N\wt{i} |\F]}{ \sum_{i=1}^N \E[\flnw(N\wt{i} -1) |\F]} \qquad \textcolor{violet}{\text{since }N\wt{i} - \flnw \in [0,1)}\\
&\leq \frac{1}{N-2} \frac{\sum_{i=1}^N \E[(N\wt{i})_2 N\wt{i} |\F]}{ \sum_{i=1}^N \E[(N\wt{i} -1)^2 |\F]} \qquad \textcolor{violet}{\text{since }N\wt{i} - \flnw \in [0,1)}\\
&= \frac{1}{N-2} \frac{\sum_{i=1}^N \E[(N\wt{i})^3 |\F] - \sum_{i=1}^N \E[(N\wt{i})^2 |\F]}{ N^2 \sum_{i=1}^N \E[(\wt{i})^2 |\F] - 2N \sum_{i=1}^N \E[\wt{i} |\F] + \sum_{i=1}^N \E[1 |\F]} \\
&= \frac{1}{N-2} \frac{N^3 \sum_{i=1}^N \E[(\wt{i})^3 |\F] - N^2 \sum_{i=1}^N \E[(\wt{i})^2 |\F]}{ N^2 \sum_{i=1}^N \E[(\wt{i})^2 |\F] - 2N  + N} \qquad \textcolor{violet}{\text{since }\sum \wt{i} =1}\\
&= \frac{1}{N-2} \frac{N^2 \sum_{i=1}^N \E[(\wt{i})^3 |\F] - N \sum_{i=1}^N \E[(\wt{i})^2 |\F]}{ N \sum_{i=1}^N \E[(\wt{i})^2 |\F] - 1} \\
&= \frac{1}{N-2} \left[ \frac{N^2 \sum_{i=1}^N \E[(\wt{i})^3 |\F]}{ N \sum_{i=1}^N \E[(\wt{i})^2 |\F] - 1} - \frac{N \sum_{i=1}^N \E[(\wt{i})^2 |\F]}{ N \sum_{i=1}^N \E[(\wt{i})^2 |\F] - 1} \right] \\
&= \frac{1}{N-2} \left[-1 + \frac{N^2 \sum_{i=1}^N \E[(\wt{i})^3 |\F] -1}{ N \sum_{i=1}^N \E[(\wt{i})^2 |\F] - 1} \right] \\
\end{align*}
Then, using that $\wt{i} =\Theta (N^{-1})$, the sum in the denominator is $O(N^{-1})$ and the sum in the numerator is $O(N^{-2})$. Hence the whole expression is $O(N^{-1})$ as $N\to\infty$, so we can find a suitable sequence $b_N \Ntoinfty 0$ to satisfy the conditions of Theorem (?) [the one with new assns].
\end{proof}

\textcolor{red}{The conclusion of the above proof is not rigorous... but we can use a simpler argument based on the bounded weights.}

\begin{proof}
Directly applying Definition \ref{defn:randround_1D}, we find firstly
\begin{align*}
\E[(v_i)_2^{(r)} | \wt{1:N}] &= \flnw (\flnw -1) (1 - Nw_i + \flnw) + (\flnw +1) \flnw (Nw_i - \flnw) \\
&= \flnw \left( 2(Nw_i - \flnw) + \flnw -1 \right)
\end{align*}
In the same way,
\begin{align*}
\E[(\vt{i})_3 |\wt{1:N}] 
&= (\flnw)_3(1-N\wt{i} + \flnw) + (\flnw +1)_3(N\wt{i} - \flnw) \\
&= (\flnw)_2 \left\{ (\flnw -2)(1-N\wt{i} + \flnw) + (\flnw +1)(N\wt{i} - \flnw)\right\} \\
&= (\flnw)_2 \left( 2N\wt{i} - \flnw -1 +  (N\wt{i} - \flnw -1) \right) \\
&\leq (\flnw -1)\flnw \left( 2N\wt{i} - \flnw -1 \right).
\end{align*}
In the case where $\flnw \geq 1$, we therefore have
\begin{align*}
\E[(\vt{i})_3 |\wt{1:N}] 
&\leq (\flnw -1) \E[(v_i)_2^{(r)} | \wt{1:N}] \\
&\leq \E[(v_i)_2^{(r)} | \wt{1:N}] \left(\frac{a^4}{\varepsilon^4} - 1 \right)
\end{align*}
with constants $a$ and $\varepsilon$ as before.
The case $\flnw < 1$ is trivial because then $\E[(\vt{i})_3 |\wt{1:N}] =0$ while $\E[(v_i)_2^{(r)} | \wt{1:N}] \geq 0$, so $b_N$ can be any non-negative sequence.
\end{proof}

\bibliography{../smc.bib}
\end{document}
