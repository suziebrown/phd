\documentclass{article}
\usepackage[utf8]{inputenc}
\usepackage[margin=2cm]{geometry}

\usepackage{amsmath}

% custom header/footer
\usepackage{fancyhdr}
\pagestyle{fancy}
\renewcommand{\headrulewidth}{0pt}
\fancyhf{}
\rfoot{\textsf{\thepage}}
\lfoot{\textsf{Suzie Brown}}

\title{Possible future research topics}
\author{Suzie Brown}
\date{15 January 2021}

\begin{document}
\maketitle
\thispagestyle{fancy}

\section*{Extending the theoretical results}
\begin{itemize}
\item Relax condition that potentials are bounded below (e.g.\ see Capp\'e, Moulines, Ryd\'en ``Inference in HMMs'', Chapter 9)
\item Show that the condition of BJJK Theorem 1 is necessary as well as sufficient (e.g.\ see M\"ohle \& Sagitov 1998, 2001, 2003?)
\item What is the effect of adaptive resampling?
\item Corollaries also for residual-multinomial and stratified resampling.
\item Rates of convergence (modifying more of M\"ohle's work).
\item Could badly-behaved potentials produce a non-Kingman Lambda-coalescent? For example, use for potentials some heavy-tailed fitnesses like in Schweinsberg model. This is probably not interesting from an SMC point of view.
\item A way to estimate coalescent rates a priori for some specific tractable class of models, say.
\item Do our results apply to cloning models, or other continuous-time models?
\item See if Jacob \& Rubenthaler's (``path storage in the PF'') brute-force technique can be adapted for use in Conditional SMC, to UB tree height
\item Does CSMC with an ``unlikely'' immortal line converge to a structured coalescent?
\end{itemize}

\section*{Resampling}
\begin{itemize}
\item Compare residual-stratified vs. stratified resampling (and residual-systematic vs systematic): are they equivalent? What if the weights are sorted? Maybe start by coding up an exploratory experiment.
\item Explore more generally the effect of pre-sorting the weights. How does this link with results of Gerber Chopin Whiteley (where they sort by particle position, if I remember rightly)?
\item Compare theoretical computation/storage costs and parallelisability between the different resampling schemes.
\item Is there a nice set of weights that will nicely illustrate all the points I want to make about resampling schemes?
\item Can the comparison of time-scales for different schemes be wrangled into/ related to a direct comparison of the variances? Is this even useful?
\item Don't forget the resampling scheme that's even worse than multinomial: sampling once from Categorical and replicating it N times. This leads to a star-shaped coalescent, which sets a precedent for non-Kingman Lambda coalescents from SMC genealogies.
\item (See marked notebook 3 page circa Jun/Jul 2020 for some more thoughts/ideas.)
\item Does residual-systematic share the same pathological behaviour as systematic resampling? E.g.\ try plugging it into the pathoglogical example from Douc Cappe Moulines.
\item The difference $\Delta_i$, as defined in \texttt{phd/latex/randomised\_rounding/randomised\_rounding.pdf}, seems to tend to a quadratic in the weight as $N\to\infty$. Prove it?
\item (Just for fun:) in stochastic rounding, how many possible ways are there to assign the offspring counts? Consider that each of the N counts takes one of two possible values, but this will overcount a lot because we are also constrained by offspring counts summing to N. It's $\binom{N}{R}$ isn't it? Where $R$ as in residual resampling. Give each parent its minimum number of offspring, and you'll be left with $R$ unassigned offspring that have to be given to $R$ distinct parents among $N$. $\binom{N}{R}$ is maximised if $R\simeq N/2$, in which case $\binom{N}{R}\simeq ??$.
\item (Conjecture:) pre-sorting of weights reduces resampling variance, but increases the coalescence rate. Intuition: when weights are sorted, small weights that sum to less than 1/N are grouped together so only one of them can have a child. (It may be that this effect is entirely cancelled by the reduction in variance elsewhere.)
\end{itemize}

\section*{Simulation experiments}
\begin{itemize}
\item Now we (almost?) have weak convergence, redo similar experiments to those at end of KJJS, but without having to fudge the argument in order to apply FDD convergence.
\item Code up all the different sampling schemes (and sorted/unsorted variants etc.) and come up with some useful (if only illustrative) experiments comparing them.
\item Decide which functions would be most interesting to illustrate the performance of resampling schemes, say using ternary diagrams. (See marked notebook 3 page at 8/7/20.)
\item Explore pre-asymptotic behaviour of CSMC for example. I did some work on this previously.
\item Make edits to my ternary plot ``library'' (see notebook 3 marked page at 8/7/20).
\end{itemize}




\end{document}