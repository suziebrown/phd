\documentclass{article}
\usepackage[utf8]{inputenc}
\usepackage[margin=2cm]{geometry}

% custom header/footer
\usepackage{fancyhdr}
\pagestyle{fancy}
\renewcommand{\headrulewidth}{0pt}
\fancyhf{}
\rfoot{\textsf{\thepage}}
\lfoot{\textsf{Suzie Brown}}

% maths
\usepackage{amsmath}
\usepackage{amssymb}
\usepackage{amsthm}
\usepackage{bbm}
\newtheorem{theorem}{Theorem}
\newtheorem{lemma}{Lemma}
\newcommand{\Prob}{\mathbb{P}}
\newcommand{\E}{\mathbb{E}}
\newcommand{\Et}{\mathbb{E}_t}
\newcommand{\V}{\operatorname{Var}}
\newcommand{\I}[1]{\mathbbm{1}_{\{#1\}}}
\newcommand{\1}[1]{\mathbbm{1}_{#1}}
\newcommand{\flnw}{\lfloor N w_i \rfloor}
\newcommand{\flnj}{\lfloor N w_j \rfloor}
\newcommand{\Ht}{\mathcal{H}_t}

\title{Residual Resampling v2.0 (in progress)}
\author{Suzie Brown}
\date{10 December 2020}

\begin{document}
\maketitle
\thispagestyle{fancy}

\begin{align}
r_i &:= Nw_i - \flnw \\
R &:= \sum_{j=1}^N r_j = N - \sum_{j=1}^N \flnj \\
\end{align}

\section*{Case $R = 0$}

\begin{equation}
\sum_{i=1}^N \E[ (\nu_i)_2 \mid \Ht] = \sum_{i=1}^N (\flnw)_2 \geq 2 | \{ \flnw \geq 2 \} |
\end{equation}
and
\begin{equation}
\sum_{i=1}^N \E[ (\nu_i)_3 \mid \Ht] = \sum_{i=1}^N (\flnw)_3 \leq (a^2)_3 | \{ \flnw \geq 2 \} | \leq a^6 | \{ \flnw \geq 2 \} |
\end{equation}
so
\begin{equation}
b_N := \frac{1}{N-2} \frac{a^6}{2}
\end{equation}
will suffice.
\textbf{NB:} $b_N$ must be deterministic, so we can't actually choose it based on which $R$ case we fall into. We can just set it to its maximum between this case and the next one.

\section*{Case $R \neq 0$}

\begin{equation}
\nu_i \overset{d}{=} \flnw + \operatorname{Bin}(R, r_i/R)
\end{equation}

\begin{equation}
\E[ \nu_i \mid \Ht] = \flnw + r_i = Nw_i
\end{equation}

\begin{align}
\E[ (\nu_i)_2 \mid \Ht] &= (\flnw)_2 + 2\flnw r_i + \frac{R-1}{R} r_i^2
= \flnw^2 - \flnw + 2 \flnw r_i + r_i^2 - \frac{r_i^2}{R} \notag\\
&= \left\{ \flnw + r_i \right\}^2 - \flnw - \frac{r_i^2}{R}
= N^2 w_i^2 - \flnw - \frac{r_i^2}{R}
\geq N^2 w_i^2 - Nw_i
\end{align}

\begin{align}
\E[ (\nu_i)_3 \mid \Ht] 
&= (\flnw)_3 + 3(\flnw)_2 r_i + 3\flnw \frac{R-1}{R} r_i^2 + \frac{(R-1)(R-2)}{R^2} r_i^3 \notag\\
&\leq (\flnw)_3 + 3(\flnw)_2 r_i + 3\flnw r_i^2 +  r_i^3 \notag\\
&\leq \flnw^3 + 3 \flnw^2 r_i + 3 \flnw r_i^2 +  r_i^3
= \left\{ \flnw + r_i \right\}^3
= N^3 w_i^3
\leq a^6
\end{align}


\begin{table}[h]
\centering
\begin{tabular}{| l c l |}
\hline
$(X)_1$ & $=$ & $X$ \\
$(X)_2$ & $=$ & $X^2 - X$ \\
$(X)_3$ & $=$ & $X^3 - 3X^2 + 2X$ \\
$X$ & $=$ & $(X)_1$ \\
$X^2$ & $=$ & $(X)_2 + (X)_1$ \\
$X^3$ & $=$ & $(X)_3 + 3(X)_2 + (X)_1$ \\
\hline
\end{tabular}
\caption{Conversions between standard and factorial powers}
\end{table}

\begin{table}[h]
\centering
\begin{tabular}{| l c l |}
\hline
$(X+Y)_1$ & $=$ & $X + Y$ \\
$(X+Y)_2$ & $=$ & $(X)_2 + 2XY + (Y)_2$ \\
$(X+Y)_3$ & $=$ & $(X)_3 + 3(X)_2 Y + 3X(Y)_2 + (Y)_3$ \\
\hline
\end{tabular}
\caption{Expansions of mixed factorial powers}
\end{table}

\begin{table}[h]
\centering
\begin{tabular}{| l c l |}
\hline
$\E[(X)_1]$ & $=$ & $n p$ \\
$\E[(X)_2]$ & $=$ & $n (n-1) p^2$ \\
$\E[(X)_3]$ & $=$ & $n (n-1) (n-2) p^3$ \\
$\E[X]$ & $=$ & $n p$ \\
$\E[X^2]$ & $=$ & $n p (1 + (n-1)p )$ \\
$\E[X^3]$ & $=$ & $n p (1 + (n-1) p (3 + (n-2) p ) )$ \\
\hline
\end{tabular}
\caption{Moments of the Binomial distirbution; $X\sim \operatorname{Bin}(n,p)$}
\end{table}

\end{document}