\chapter{Limits}

%%% choose an epigraph to go here

\section{Encoding genealogies}

\subsection{The genealogical process}
\draft{Encoding as process on space of partitions $\mathcal{P}_n$. Argue that this encodes everything we need. Initial and absorbing states. Intuit with diagram(s), explain relationship between partition blocks and genealogical tree.}

\subsection{Time scale}
\draft{Introduce $c_N$, $\tau_N$, $D_N$. Contrast to pop gen literature, e.g. our $c_N$/time scale is random. Properties of these quantities: $c_N, D_N \in [0,1]$, and $D_N \leq c_N$ and $\sum_{r=1}^{\tau_N(t)} c_N \in [t, t+1]$ (or rather the version of that with general start time).}

\subsection{Transition probabilities}
\draft{Introduce $p_{\xi\eta}$. Present expression for that (or at least for $p_{\xi\xi}$), and hence the bounds on it that will be used later (keeping big-O terms explicit where possible).}

\section{An existing limit theorem}
\draft{State KJJS theorem. Discuss the conditions in detail. Give outline of proof.}

\section{A new limit theorem}
\draft{State our limit theorem. Give intuition for the new condition. Compare to KJJS: why our conditions might be considered ``weaker'' (Moran model example, and whatever else we said to our referee/in the BJJK article); our condition is easier to check (as demonstrated in later corollaries).}

\subsection{Proof of theorem}
\draft{Proof that KJJS conditions are implied by ours. Modification of KJJS proof (or even write out a complete proof?) using weaker bound on $p_{\xi\xi}$ (that bound should have been stated and proved already in transition probabilities section).}