\documentclass[fleqn]{article}
\usepackage[utf8]{inputenc}
\usepackage[margin=2cm]{geometry}

\usepackage{xcolor}

% maths
\usepackage{amsmath}
\usepackage{amssymb}
\usepackage{amsthm}

% bibliography
\usepackage[round, sort&compress]{natbib}
\usepackage{har2nat}
\bibliographystyle{agsm}

% custom header/footer
\usepackage{fancyhdr}
\pagestyle{fancy}
\renewcommand{\headrulewidth}{0pt}
\fancyhf{}
\rfoot{\textsf{\thepage}}
\lfoot{\textsf{Suzie Brown}}

% generally useful commands
\newcommand{\E}{\mathbb{E}}
\newcommand{\eqdist}{\overset{d}{=}}
\newcommand{\I}[1]{\mathbb{I}\{#1\}}

% project-specific commands
\newcommand{\F}{\mathcal{F}_{t-1}}
\newcommand{\Mn}{\operatorname{Multinomial}}
\newcommand{\Cat}{\operatorname{Categorical}}
\newcommand{\vt}[2][t]{v_{#1}^{(#2)}}
\newcommand{\wt}[2][t]{w_{#1}^{(#2)}}
\newcommand{\wbar}[2][t]{\bar{w}_{#1}^{(#2)}}

\title{Coupling argument for residual resampling}
\author{Suzie Brown}
\date{\today}

\begin{document}
\maketitle
\thispagestyle{fancy}

Consider a population of $N$ particles at time $t$, having fixed weights $(\wt{1},\dots, \wt{N})$ respectively. These will be the parents, and we consider assigning to them $N$ offspring which will form the next generation. 
Let $(a_1, \dots, a_N)$ denote the vector of parental indices of each child, and $(v_1, \dots, v_N)$ the resulting offspring counts for each parent. That is, $v_i = \sum_{j=1}^N \I{a_j = i} $. We will use a superscript $m$ or $r$ to denote the multinomial or residual schemes respectively.\\

In the case of multinomial resampling, we have the following:
\begin{align*}
& (a_1^m, \dots, a_N^m) \sim \Cat(1:N, \wt{1:N}) \\
& (v_1^m, \dots, v_N^m) \sim \Mn(N, \wt{1:N})
\end{align*}

In residual resampling, the first $k := \sum_{i=1}^N \lfloor N\wt{i} \rfloor$ offspring are assigned deterministically, and the remaining $N-k$ are assigned randomly:
\begin{align*}
& (v_1^r, \dots, v_N^r) \eqdist \Mn(N-k, \wt{1:N}) + \lfloor N \wt{1:N} \rfloor
\end{align*}

This distribution can be achieved by doing multinomial resampling as usual, but then overwriting the parental indices of $k$ offspring (to be chosen at random) with the $k$ deterministic assignments (since the offspring are exchangeable and marginals of the Multinomial/Categorical distribution are Multinomial/Categorical).\\

\color{red}
Actually that's not true. Counterexample: suppose the weights are $\wt{1} = 1/2$ and $\wt{i} = 1/(2N-2)$ for $i=2,\dots,N$. Then we have $\E[v_1^m] = N/2$ as expected. But if we then just overwrite some indices as suggested above, we get $\E[v_1^r] = N/2 + (N-N/2)/2 = 3N/4$, where the first term is the expected number of offspring assigned by the original multinomial resampling, and the second term is the expected number of additional offspring assigned by overwriting with the deterministic assignments. The problem is that we are accounting doubly for the high weight of parent 1.

For such a coupling to work we would have t compare to a multinomial scheme where the weights are just the residual weights, not the original weights. In that case we are not comparing the two schemes on the same set of weights, so it is not useful.\\

\color{black}
The probability that a randomly chosen pair of children share the same parent is, with multinomial resampling:
\begin{equation*}
p^m = \sum_{i=1}^N (\wt{i})^2
\end{equation*}
and with residual resampling (by partitioning over the common parent and into the cases of deterministic assignment and random assignment):
\begin{align*}
p^r &= \sum_{i=1}^N  \left[ \frac{N-k}{N} \wt{i} \left( \frac{N-k-1}{N} \wt{i} + \frac{\lfloor N\wt{i}\rfloor}{N}\right) + \frac{\lfloor N\wt{i}\rfloor}{N} \left(\frac{N-k}{N} \wt{i} + \frac{k-1}{k} \frac{\lfloor N\wt{i}\rfloor -1}{N} \right) \right] \\
&= \sum_{i=1}^N \frac{1}{N^2} \left[ \left( \wt{i} (N-k) + \lfloor N\wt{i}\rfloor \right)^2 - (N-k) (\wt{i})^2 - \frac{k-1}{k} \lfloor N\wt{i}\rfloor - \frac{1}{k}  \lfloor N\wt{i}\rfloor^2 \right] \\
&\leq \sum_{i=1}^N \frac{1}{N^2} \left( (N-k) \wt{i} + \lfloor N\wt{i}\rfloor \right)^2
\end{align*}

\bibliography{smc.bib}
\end{document}