\documentclass{article}
\usepackage[utf8]{inputenc}
\usepackage[margin=2cm]{geometry}

\usepackage{graphicx}
\usepackage{enumitem}

% custom header/footer
\usepackage{fancyhdr}
\pagestyle{fancy}
\renewcommand{\headrulewidth}{0pt}
\fancyhf{}
\rfoot{\textsf{\thepage}}
\lfoot{\textsf{Suzie Brown}}

%annotations
\usepackage{color}
\usepackage{xspace}
\newcommand{\seb}[1]{\xspace\textcolor{red}{#1}\xspace}

% bibliography
\usepackage[round, sort&compress]{natbib}
\usepackage{har2nat}
\bibliographystyle{agsm}

% maths
\usepackage{amsmath}
\usepackage{amssymb}
\usepackage{amsthm}
\usepackage[mathscr]{euscript}
\usepackage{bbm}
\newtheorem{theorem}{Theorem}
\newtheorem{lemma}{Lemma}
\newcommand{\Prob}{\mathbb{P}}
\newcommand{\E}{\mathbb{E}}
\newcommand{\Et}{\mathbb{E}_t}
\newcommand{\V}{\operatorname{Var}}
\newcommand{\I}[1]{\mathbbm{1}_{\{#1\}}}
\newcommand{\1}[1]{\mathbbm{1}_{#1}}
\newcommand{\Mn}{\operatorname{Multinomial}}

%\allowdisplaybreaks

\title{Weak convergence proof v.2 (neater) (in progress)}
\author{Suzie Brown}

\begin{document}
\maketitle
\thispagestyle{fancy}

\begin{lemma}
\begin{equation}
\sum_{s_1\neq\dots\neq s_l}^{\tau_N(t)} \prod_{j=1}^l c_N(s_j)
\geq t^l - \left( \sum_{s=1}^{\tau_N(t)} c_N(s)^2 \right) \binom{l}{2} (t+1)^{l-2} .
\end{equation}
\end{lemma}

\begin{proof}
As pointed out in \citet[Equation (8)]{koskela2018}, 
\begin{equation}\label{eq:02}
\sum_{s_1\neq\dots\neq s_l}^{\tau_N(t)} \prod_{j=1}^l c_N(s_j)
\geq \left( \sum_{s=0}^{\tau_N(t)} c_N(s) \right)^l
- \binom{l}{2} \left( \sum_{s=0}^{\tau_N(t)} c_N(s)^2 \right)
\left( \sum_{s=0}^{\tau_N(t)} c_N(s) \right)^{l-2} .
\end{equation}
By definition of $\tau_N$, 
\begin{equation}
t \leq \sum_{s=0}^{\tau_N(t)} c_N(s) \leq t+1 .
\end{equation}
Substituting these bounds into the RHS of \eqref{eq:02} yields the result.
\end{proof}


\begin{lemma}
\begin{equation}
\sum_{s_1\neq\dots\neq s_l}^{\tau_N(t)} \prod_{j=1}^l c_N(s_j)
\leq t^l + c_N(\tau_N(t)) (t+1)^l .
\end{equation}
\end{lemma}

\begin{proof}
It is a true fact that
\begin{equation}
\sum_{s_1\neq\dots\neq s_l}^{\tau_N(t)} \prod_{j=1}^l c_N(s_j)
\leq \left( \sum_{s=0}^{\tau_N(t)} c_N(s) \right)^l ,
\end{equation}
as can be seen by considering the multinomial expansion of the RHS.
This is further bounded by
\begin{equation}
\left( \sum_{s=0}^{\tau_N(t)} c_N(s) \right)^l
\leq \left( \sum_{s=0}^{\tau_N(t) -1} c_N(s) + c_N(\tau_N(t)) \right)^l
\leq \left[ t + c_N(\tau_N(t)) \right]^l ,
\end{equation}
again using the definition of $\tau_N$.
A binomial expansion yields
\begin{equation}
\left[ t + c_N(\tau_N(t)) \right]^l
= t^l + \sum_{i=0}^{l-1} \binom{l}{i} t^i c_N(\tau_N(t))^{l-i}
= t^l + c_N(\tau_N(t)) \sum_{i=0}^{l-1} \binom{l}{i} t^i c_N(\tau_N(t))^{l-1-i} ,
\end{equation}
then since $c_N(s) \leq 1$ for all $s$,
\begin{equation}
\sum_{i=0}^{l-1} \binom{l}{i} t^i c_N(\tau_N(t))^{l-1-i}
\leq \sum_{i=0}^{l-1} \binom{l}{i} t^i
\leq (t+1)^l .
\end{equation}
Putting this together yields the result.
\end{proof}


\begin{lemma}
Let $B$ be a positive constant which may depend on $n$.
\begin{equation}
\sum_{s_1\neq\dots\neq s_l}^{\tau_N(t)} \prod_{j=1}^l 
\left[ c_N(s_j) + B D_N(s_j) \right]
\leq \sum_{s_1\neq\dots\neq s_l}^{\tau_N(t)} \prod_{j=1}^l c_N(s_j)
+ \left( \sum_{s=1}^{\tau_N(t)} D_N(s) \right) (t+1)^{l-1} (1+B)^l .
\end{equation}
\end{lemma}

\begin{lemma}
Let $B$ be a positive constant which may depend on $n$.
\begin{equation}
\sum_{s_1\neq\dots\neq s_l}^{\tau_N(t)} \prod_{j=1}^l 
\left[ c_N(s_j) - B D_N(s_j) \right]
\geq \sum_{s_1\neq\dots\neq s_l}^{\tau_N(t)} \prod_{j=1}^l c_N(s_j)
- \left( \sum_{s=1}^{\tau_N(t)} D_N(s) \right) (t+1)^{l-1} (1+B)^l .
\end{equation}
\end{lemma}






\bibliography{../smc.bib}
\end{document}