%\documentclass[manuscript]{biometrika}
\documentclass[lineno]{biometrika}

\usepackage{amsmath}
\usepackage{amssymb}

%% Please use the following statements for
%% managing the text and math fonts for your papers:
\usepackage{times}
%\usepackage[cmbold]{mathtime}
\usepackage{bm}
\usepackage{natbib}

\usepackage[plain,noend]{algorithm2e}

\makeatletter
\renewcommand{\algocf@captiontext}[2]{#1\algocf@typo. \AlCapFnt{}#2} % text of caption
\renewcommand{\AlTitleFnt}[1]{#1\unskip}% default definition
\def\@algocf@capt@plain{top}
\renewcommand{\algocf@makecaption}[2]{%
  \addtolength{\hsize}{\algomargin}%
  \sbox\@tempboxa{\algocf@captiontext{#1}{#2}}%
  \ifdim\wd\@tempboxa >\hsize%     % if caption is longer than a line
    \hskip .5\algomargin%
    \parbox[t]{\hsize}{\algocf@captiontext{#1}{#2}}% then caption is not centered
  \else%
    \global\@minipagefalse%
    \hbox to\hsize{\box\@tempboxa}% else caption is centered
  \fi%
  \addtolength{\hsize}{-\algomargin}%
}
\makeatother

%%% User-defined macros should be placed here, but keep them to a minimum.
\newcommand{\vt}[2][t]{\nu_{#1}^{(#2)}}
\newcommand{\Et}{E_{t}}

\addtolength\topmargin{35pt}
\DeclareMathOperator{\Thetabb}{\mathcal{C}}

\begin{document}

\jname{Biometrika}
%% The year, volume, and number are determined on publication
\jyear{}
\jvol{}
\jnum{}
%% The \doi{...} and \accessdate commands are used by the production team
%\doi{10.1093/biomet/asm023}
\accessdate{Advance Access publication on }

%% These dates are usually set by the production team
\received{2 January 2017}
\revised{1 April 2017}

%% The left and right page headers are defined here:
\markboth{S. Brown et~al.}{More SMC genealogies}

%% Here are the title, author names and addresses
\title{Some more things about SMC genealogies...}

\author{S. BROWN}
\affil{Department of Statistics, University of Warwick, Coventry CV4 7AL, U.K. \email{s.brown.18@warwick.ac.uk}}

\author{P. JENKINS}
\affil{Department of Statistics, University of Warwick, Coventry CV4 7AL, U.K.
\email{p.jenkins@warwick.ac.uk}}

\author{A. M. JOHANSEN}
\affil{Department of Statistics, University of Warwick, Coventry CV4 7AL, U.K.
\email{a.m.johansen@warwick.ac.uk}}

\author{\and J. KOSKELA}
\affil{Department of Statistics, University of Warwick, Coventry CV4 7AL, U.K. \email{j.koskela@warwick.ac.uk}}

\maketitle

\begin{abstract}

\end{abstract}

\begin{keywords}
Blah; Blah; Blah
\end{keywords}

\section{Introduction}
Notation to introduce at some point:
$\Et[...] = E[... \mid \mathcal{F}_{t-1}]$;
$(a)_b = a(a-1)... (a-b+1)$;

\section{The updated theorem...}
\begin{theorem}\label{thm:generalconv}
Suppose that there exists a deterministic sequence $(b_N)_{N\geq1}$ such that $b_N \rightarrow 0$ and
\begin{equation}\label{eq:mainthmcond}
\frac{1}{(N)_3} \sum_{i = 1}^N \Et[ (\vt{i})_3 ]  \leq b_N \frac{1}{(N)_2} \sum_{i = 1}^N \Et[ (\vt{i})_2  ]
\end{equation}
for all $N$, uniformly in $t \geq 1$.
And the standing assumption...
Then something converges to something in some way as $N \to \infty$ ...
\end{theorem}

\begin{proof}
Theorem \ref{thm:generalconv} has the same conclusion as \citet[Theorem 1]{koskela2018}, but with much more tractable predicates. We will show that these simpler predicates are sufficient.

The conditions for \citet[Theorem 1]{koskela2018} are the following:
\begin{align}
E [c_N(t)] &\rightarrow 0 , \label{eq:oldass1}\\
E \left[ \sum_{ r = \tau_N( s ) + 1 }^{ \tau_N( t ) } D_N( r ) \right] &\rightarrow 0, \label{eq:oldass2}\\
E \left[ \sum_{ r = \tau_N( s ) + 1 }^{ \tau_N( t ) } c_N( r )^2 \right] &\rightarrow 0 , \label{eq:oldass3}\\
E [\tau_N(t) - \tau_N(s)] &\leq C_{t,s} N; \label{eq:oldass4}
\end{align}
as $N\to\infty$, for some strictly positive constant $C_{t,s}$ that does not depend on $N$.

The series of Lemmata \ref{lem:removeass3}--\ref{lem:removeass2} show that the assumptions \eqref{eq:oldass1}--\eqref{eq:oldass3} all follow from \eqref{eq:mainthmcond}.
Lemma \ref{lem:removeass4} shows that assumption \eqref{eq:oldass4} is not needed.
\end{proof}

\begin{lemma} \label{lem:removeass3}
$\eqref{eq:oldass2} \Rightarrow \eqref{eq:oldass3}$.
\end{lemma}

\begin{lemma} \label{lem:removeass1}
$\eqref{eq:mainthmcond} \Rightarrow \eqref{eq:oldass1}$.
\end{lemma}

\begin{lemma} \label{lem:removeass2}
$\eqref{eq:mainthmcond} \Rightarrow \eqref{eq:oldass2}$.
\end{lemma}

\begin{lemma}\label{lem:removeass4}
write a lemma here...
\end{lemma}

Equipped with this simplified statement of the theorem, we can now prove convergence for some more complicated sequential Monte Carlo algorithms.

\section{Resampling with stochastic roundings}
\begin{definition}\label{defn:randround_1D}
Let $X\geq 0$. A random variable $Y: \mathbb{R}_+ \to \mathbb{N}$ is a \emph{stochastic rounding} of $X$ if $Y$ takes the values
\begin{equation*}
Y=
\begin{cases}
 \lfloor X \rfloor & \text{with probability } 1- X+ \lfloor X \rfloor \\
  \lfloor X \rfloor +1 & \text{with probability } X- \lfloor X \rfloor 
\end{cases}
\end{equation*}
\end{definition}


\section{Conditional sequential Monte Carlo updates}


\bibliographystyle{biometrika}
\bibliography{../smc.bib}
\end{document}
