\chapter{Applications}\label{ch:appl}

\epigraph{
Earth's crammed with heaven, \\
And every common bush afire with God, \\
But only he who sees takes off his shoes; \\
The rest sit round and pluck blackberries.
}
{\textsc{Elizabeth Barrett Browning}}

\draft{Add some beginning-of-chapter waffle here.}

\section{Multinomial resampling \seb{$\checkmark$} }
Multinomial resampling is often preferred in theoretical studies of SMC, because it renders the parental indices conditionally i.i.d. given the weights, making it relatively simple to analyse the resulting algorithm.
The result of Corollary~\ref{thm:multinomial} was proved in \textcite[Corollary 1]{koskela2018}. Nonetheless it is included here, firstly to demonstrate the relative ease with which it is proved under Theorem~\ref{thm:FDDconv} as opposed to \textcite[Theorem 1]{koskela2018}, and secondly to outline the proof technique before proceeding to more complex resampling schemes.

\begin{corollary}\label{thm:multinomial}
Consider an SMC algorithm using multinomial resampling, such that \ref{standing_assumption} is satisfied. Assume there exist constants $\varepsilon\in (0,1], a\in [1,\infty)$ and probability density $h$ such that for all $x, x^\prime, t$,
\begin{equation}\label{eq:gq_bounds_mn}
\frac{1}{a} \leq g_t(x, x^\prime) \leq a , \quad
\varepsilon h(x^\prime) \leq q_t(x, x^\prime) \leq \frac{1}{\varepsilon} h(x^\prime) .
\end{equation}
Let $(G_t^{(n,N)})_{t\geq0}$ denote the genealogy of a random sample of $n$ terminal particles from the output of the algorithm when the total number of particles used is $N$. Then, for any fixed $n$, the time-scaled genealogy $(G_{\tau_N(t)}^{(n,N)})_{t\geq0}$ converges to Kingman's $n$-coalescent as $N\to \infty$, in the sense of finite-dimensional distributions.
\end{corollary}
The bounds on $g_t$ and $q_t$ in \eqref{eq:gq_bounds_mn} are rather strong; they can only reasonably be expected to hold if the state space is compact. However they are widespread in the literature, where they are known as the \emph{strong mixing conditions} \parencite[Section 3.5.2]{delmoral2004}, because they greatly facilitate the theoretical analysis of SMC algorithms. It is often possible to relax these conditions at the expense of considerable technical complication.
In the following proof, conditions \eqref{eq:gq_bounds_mn} ensure that the weights are all $O(N^{-1})$, none of them being too close to zero or one, which controls the relative rate of multiple mergers.

\begin{proof}
Define the conditioning $\sigma$-algebra $\mathcal{H}_t$ as in \eqref{eq:defn_Ht}.
The parental indices are conditionally independent, with conditional law
\begin{equation}\label{eq:parentslaw_mn}
\Prob \left[ a_t^{(i)} = a_i \mid \mathcal{H}_t \right] 
%\propto w_t^{(a_i)} q_{t-1}(X_t^{(a_i)}, X_{t-1}^{(i)})
\propto g_t( X_{t+1}^{a_t^{(a_i)}} , X_t^{(a_i)} ) q_{t-1}(X_t^{(a_i)}, X_{t-1}^{(i)})
\end{equation}
for each $i$, so the joint law is
\begin{equation*}
\Prob \left[ a_t^{(1:N)} = a_{1:N} \mid \mathcal{H}_t \right] 
%\propto \prod_{i=1}^N w_t^{(a_i)} q_{t-1}(X_t^{(a_i)}, X_{t-1}^{(i)}).
\propto \prod_{i=1}^N g_t( X_{t+1}^{a_t^{(a_i)}} , X_t^{(a_i)} ) 
        q_{t-1}(X_t^{(a_i)}, X_{t-1}^{(i)}).
\end{equation*}
%
%%%%%%%%%
%
%The bounds on $g_t$ in \eqref{eq:gq_bounds_mn} imply 
%almost sure bounds on the weights
%%the almost sure bounds
%%\begin{equation*}
%%\frac{1}{a^2 N}
%%\leq w_t^{(i)}
%%\leq \frac{a^2}{N}
%%\end{equation*}
%which, along with the bounds on $q_{t-1}$ from \eqref{eq:gq_bounds_mn}, allows us to apply a balls-in-bins coupling presented in \textcite[Proof of Lemma 3]{koskela2018} to obtain bounds on expectations of functions of $a_t^{(1:N)}$.
Using the bounds \eqref{eq:gq_bounds_mn} and the balls-in-bins coupling of \textcite[Proof of Lemma 3]{koskela2018}, we can obtain bounds on expectations of functions of $a_t^{(1:N)}$.
For any $k\in\mathbb{N}$ the function $a_t^{(1:N)} \to (\nu_t^{(i)})_k$ is $\{i\}$-increasing in the sense of \textcite{koskela2018}, so we may apply the bounds
\begin{equation*}
\E[ (V_1^{(i)})_k ]
\leq \E[ (\nu_t^{(i)})_k \mid \mathcal{H}_t ]
\leq \E[ (V_2^{(i)})_k ] ,
\end{equation*}
where
\begin{align*}
& V_1^{(i)} 
\sim \Bin\left(N, \frac{\varepsilon/a}{(\varepsilon/a) + (N-1)(a/\varepsilon)} \right),\\
& V_2^{(i)} 
\sim \Bin\left( N, \frac{a/\varepsilon}{(a/\varepsilon) + (N-1)(\varepsilon/a)} \right) .
\end{align*}
independently for each $i$ and independently of $\mathcal{F}_\infty$.
Furthermore, using the moments of the Binomial distribution \parencite[see for example][p. 67]{mosimann1962}
\begin{equation*}
\E[ (V_1^{(i)})_k ]
= (N)_k \left( \frac{ \varepsilon/a }{ (\varepsilon/a) + (N-1)(a/\varepsilon) } \right)^k
\geq (N)_k \left( \frac{ \varepsilon/a }{ N(a/\varepsilon) } \right)^k
= \frac{(N)_k}{N^k} \frac{\varepsilon^{2k}}{a^{2k}} .
\end{equation*}
Similarly, 
\begin{equation*}
\E[ (V_2^{(i)})_k ]
\leq \frac{(N)_k}{N^k} \frac{a^{2k}}{\varepsilon^{2k}} .
\end{equation*}
We therefore have the bounds
\begin{equation*}
\frac{(N)_k}{N^k} \frac{\varepsilon^{2k}}{a^{2k}}
\leq \E[ (\nu_t^{(i)})_k \mid \mathcal{H}_t ]
\leq \frac{(N)_k}{N^k} \frac{a^{2k}}{\varepsilon^{2k}} . %\label{eq:mn_nuk_bounds}
\end{equation*}
for each $k$. Consequently,
\begin{equation}
\frac{1}{(N)_2} \sum_{i=1}^N \E[ (\nu_t^{(i)})_2 \mid \mathcal{H}_t ]
\geq \frac{\varepsilon^4}{Na^4} \label{eq:mn_cN_LB}
\end{equation}
and
\begin{equation}
\frac{1}{(N)_3} \sum_{i=1}^N \E[ (\nu_t^{(i)})_3 \mid \mathcal{H}_t ]
\leq \frac{a^6}{N^2 \varepsilon^6} . \label{eq:mn_cN3_UB}
\end{equation}
%
%%%%%%%%%
%
%\textcite{koskela2018} makes the following observation, which follows from a balls-in-bins coupling.
%Assume \eqref{eq:gq_bounds_mn}. 
%Then for any function $f:\{1,\dots,N\}^N \to \mathbb{R}$ such that (for a fixed $i$) $f(a_t^{\prime(1:N)}) \geq f(a_t^{(1:N)})$ whenever $|\{j:a_t^{\prime(j)}=i\}| \geq |\{j:a_t^{(j)}=i\}|$,
%\begin{equation}\label{eq:mn_f_bound}
%\E[ f(A_{1,i}^{(1:N)}) ] 
%\leq \E[ f(a_t^{(1:N)}) \mid \mathcal{H}_t ]
%\leq \E[ f(A_{2,i}^{(1:N)}) ] 
%\end{equation}
%where the elements of $A_{1,i}^{(1:N)}, A_{2,i}^{(1:N)}$ are all mutually independent and independent of $\mathcal{F}_{\infty}$, and distributed according to
%\begin{align*}
%& A_{1,i}^{(j)} 
%\sim \Cat \left( (\varepsilon/a)^{\1{i=1} -\1{i\neq 1}} ,
%        \dots, (\varepsilon/a)^{\1{i=N} -\1{i\neq N}} \right) \\
%& A_{2,i}^{(j)} 
%\sim \Cat \left( (a/\varepsilon)^{\1{i=1} -\1{i\neq 1}} ,
%        \dots, (a/\varepsilon)^{\1{i=N} -\1{i\neq N}} \right)
%\end{align*}
%independently for each $j$, where the vector of probabilities is given up to a constant in the argument of Categorical distributions.
%We will use these random vectors to construct bounds that are independent of $\mathcal{F}_\infty$.
%Also define the corresponding offspring counts $V_1^{(i)} = |\{j: A_{1,i}^{(j)}=i\}|$, $V_2^{(i)} = |\{j: A_{2,i}^{(j)}=i\}|$, for $i=1,\dots,N$, which have marginal distributions
%\begin{align*}
%& V_1^{(i)} 
%\sim \Bin\left(N, \frac{\varepsilon/a}{(\varepsilon/a) + (N-1)(a/\varepsilon)} \right),\\
%& V_2^{(i)} 
%\sim \Bin\left( N, \frac{a/\varepsilon}{(a/\varepsilon) + (N-1)(\varepsilon/a)} \right) .
%\end{align*}
%\seb{ Could state that a particular consequence of \eqref{eq:mn_f_bound} and $V_1,V_2$ etc.\ is
%\begin{equation*}
%\frac{(N)_k}{N^k} \frac{\varepsilon^{2k}}{a^{2k}}
%\leq \E[ (\nu_t^{(i)})_k \mid \mathcal{H}_t ]
%\leq \frac{(N)_k}{N^k} \frac{a^{2k}}{\varepsilon^{2k}} .
%\end{equation*}
%for any $k\in\mathbb{N}$? This is easily proved in much the same way as the two special cases below. This could probably replace \eqref{eq:mn_f_bound} and the surrounding argument, since all $f$s we use are of this particular form.}
%Now consider the function $f_i(a_t^{(1:N)}) := (\nu_t^{(i)})_2$. We can apply \eqref{eq:mn_f_bound} to obtain the lower bound
%\begin{align}
%\frac{1}{(N)_2} \sum_{i=1}^N \E[ (\nu_t^{(i)})_2 \mid \mathcal{H}_t ]
%&\geq \frac{1}{(N)_2} \sum_{i=1}^N \E[ (V_1^{(i)})_2 ]
%= \frac{1}{(N)_2} \sum_{i=1}^N (N)_2 \left[ \frac{ \varepsilon/a}{ (\varepsilon/a) + (N-1)(a/\varepsilon) } \right]^2 \notag\\
%&\geq \sum_{i=1}^N \left[ \frac{ \varepsilon/a}{ N a/\varepsilon } \right]^2
%= \frac{\varepsilon^4}{Na^4} \label{eq:mn_cN_LB}
%\end{align}
%using the moments of the Binomial distribution \parencite[see for example]{mosimann1962}.
%Similarly, we derive an upper bound on $f_i(a_t^{(1:N)}) := (\nu_t^{(i)})_3$:
%\begin{align}
%\frac{1}{(N)_3} \sum_{i=1}^N \E[ (\nu_t^{(i)})_3 \mid \mathcal{H}_t]
%&\leq \frac{1}{(N)_3} \left[ \sum_{i=1}^N \E[ (V_2^{(i)})_3 ] \right]
%= \frac{1}{(N)_3} \sum_{i=1}^N (N)_3 \left[ \frac{ a/\varepsilon }{ (a/\varepsilon) + (N-1)(\varepsilon/a) } \right]^3 \notag\\
%&\leq \sum_{i=1}^N \left[ \frac{ a/\varepsilon }{ N \varepsilon/a } \right]^3
%= \frac{a^6}{N^2\varepsilon^6} . \label{eq:mn_cN3_UB}
%\end{align}
The definition of $\mathcal{H}_t$ is such that, for any suitable function $f$, by the tower property and conditional independence we have
\begin{equation}
\Et[ f(\nu_t^{(1:N)}) ] 
= \Et\left[ \E[ f(\nu_t^{(1:N)}) \mid \mathcal{H}_t, \mathcal{F}_{t-1} ] \right] 
= \Et\left[ \E[ f(\nu_t^{(1:N)}) \mid \mathcal{H}_t ] \right] .\label{eq:condexp_HtFt}
\end{equation}
Applying this identity to \eqref{eq:mn_cN_LB} and \eqref{eq:mn_cN3_UB} we find
\begin{align*}
\frac{\frac{1}{(N)_3} \sum_{i=1}^N \Et[ (\nu_t^{(i)})_3 ]}{\frac{1}{(N)_2} \sum_{i=1}^N \Et[ (\nu_t^{(i)})_2 ]}
&\leq \frac{ a^6 / (N^2 \varepsilon^6) }{ \varepsilon^4 / (N a^4) }
= \frac{ a^{10} }{ N\varepsilon^{10} }
=: b_N \underset{N\to\infty}{\longrightarrow} 0.
\end{align*}
Thus \eqref{eq:mainthmcond} is satisfied. 
It remains to show that, for $N$ sufficiently large, $\Prob[ \tau_N(t) = \infty ] =0$ for all finite $t$, a technicality which is proved in Lemma \ref{thm:mn_nontriviality}. 
Applying Theorem~\ref{thm:FDDconv} then yields the result.
\end{proof}



\begin{lemma}\label{thm:mn_nontriviality}
Consider an SMC algorithm using multinomial resampling, satisfying \ref{standing_assumption} and \eqref{eq:gq_bounds_mn}. 
Then, for all $N>2$, $\Prob[ \tau_N(t) = \infty ]=0$ for all finite $t$.
\end{lemma}

\begin{proof}
Since $c_N(t) \in [0,1]$ almost surely and has strictly positive expectation, for any fixed $N$ the distribution of $c_N(t)$ with given expectation that maximises $\Prob[ c_N(t)=0 \mid \mathcal{F}_{t-1} ]$ is two atoms, at 0 and 1 respectively. To ensure the correct expectation, the atom at 1 should have mass $\Prob[ c_N(t)=1 \mid \mathcal{F}_{t-1} ] = \Et [ c_N(t) ]$, which is bounded below by \eqref{eq:mn_cN_LB}.
If $c_N(t) > 0$ then $c_N(t) \geq 2/(N)_2 > 2/N^2$. Hence, in general $\Prob[ c_N(t) > 2/N^2 \mid \mathcal{F}_{t-1} ] \geq \Et [c_N(t)]$. Applying \eqref{eq:mn_cN_LB} along with \eqref{eq:condexp_HtFt}, we have for any finite $N$
\begin{equation*}
\sum_{t=0}^\infty \Prob[ c_N(t) > 2/N^2 \mid \mathcal{F}_{t-1} ]
\geq \sum_{t=0}^\infty \Et [ c_N(t) ]
\geq \sum_{t=0}^\infty \frac{\varepsilon^4}{Na^4}
= \infty .
\end{equation*}
By a filtered version of the second Borel--Cantelli lemma \parencite[see for example][Theorem 4.3.4]{durrett2019}, this implies that $c_N(t) >2/N^2$ for infinitely many $t$, almost surely.
This ensures, for all $t <\infty$, that $\Prob\left[ \exists s<\infty : \sum_{r=1}^s c_N(r) \geq t \right] =1$, which by definition of $\tau_N(t)$ is equivalent to $\Prob[ \tau_N(t) = \infty ] =0$.
\end{proof}




\section{Stratified resampling \seb{$\checkmark$} }

\begin{corollary}\label{thm:stratified}
Consider an SMC algorithm using stratified resampling, such that \ref{standing_assumption} is satisfied.
Assume that there exists a constant $a\in [1,\infty)$ such that for all $x, x^\prime, t$,
\begin{equation}\label{eq:gq_bounds_sr}
\frac{1}{a} \leq g_t(x, x^\prime) \leq a .
\end{equation}
Assume that $\Prob[ \tau_N(t) = \infty] =0$ for all finite $t$.
Let $(G_t^{(n,N)})_{t\geq0}$ denote the genealogy of a random sample of $n$ terminal particles from the output of the algorithm when the total number of particles used is $N$. Then, for any fixed $n$, the time-scaled genealogy $(G_{\tau_N(t)}^{(n,N)})_{t\geq0}$ converges to Kingman's $n$-coalescent as $N\to \infty$, in the sense of finite-dimensional distributions.
\end{corollary}

\draft{Remark that bounds on $q$ are no longer needed, and explain why. Explain why we now include the finite time scale condition in the statement: it doesn't necessarily hold under the conditions stated here, as was the case in Corollary~\ref{thm:multinomial}.}

\seb{By the way, does the lack of conditions of $q_t$ here imply that we do not even need the transition kernels to admit densities?}

\begin{proof}
Recall that the sequence of $\sigma$-algebras
\begin{equation}\label{eq:defn_Ht}
\mathcal{H}_t := \sigma(X_{t-1}^{(1:N)}, X_t^{(1:N)}, w_{t-1}^{(1:N)}, w_t^{(1:N)} )
\end{equation}
are such that $\nu_t^{(1:N)}$ is conditionally independent of the filtration $\mathcal{F}_{t-1}$ given $\mathcal{H}_t$.
With stratified resampling, conditional on the weights each offspring count almost surely takes one of four values: $\nu_t^{(i)} \in \{ \flnw -1, \flnw, \flnw +1, \flnw +2 \}$.  
Denote $p_j^{(i)} := \Prob[ \nu_t^{(i)} = \flnw +j \mid \mathcal{H}_t ]$ for $j=-1,0,1,2$.
Now
\begin{align*}
\E [(\nu_t^{(i)})_2 \mid \mathcal{H}_t ]
&= p_{-1}^{(i)} (\flnw -1)_2 + p_0^{(i)} (\flnw)_2 + p_1^{(i)} (\flnw +1)_2 \\
    &\hspace{3cm} + p_2^{(i)} (\flnw +2)_2
\end{align*}
and
\begin{align*}
\E [(\nu_t^{(i)})_3 \mid \mathcal{H}_t ]
&= p_{-1}^{(i)} (\flnw -1)_3 + p_0^{(i)} (\flnw)_3 + p_1^{(i)} (\flnw +1)_3 \\
    &\hspace{1cm} + p_2^{(i)} (\flnw +2)_3 \\
&= p_{-1}^{(i)} (\flnw -3)(\flnw -1)_2 + p_0^{(i)} (\flnw -2)(\flnw)_2 \\
     &\hspace{1cm} + p_1^{(i)} (\flnw -1)(\flnw +1)_2 
         + p_2^{(i)} \flnw (\flnw +2)_2 \\
&\leq \flnw \Big\{ p_{-1}^{(i)} (\flnw -1)_2 + p_0^{(i)} (\flnw)_2 
        + p_1^{(i)} (\flnw +1)_2 \\
    &\hspace{1cm} + p_2^{(i)} (\flnw +2)_2 \Big\} \\
&= \flnw ]\E [(\nu_t^{(i)})_2 \mid \mathcal{H}_t ] \\
&\leq a^2 \E [(\nu_t^{(i)})_2 \mid \mathcal{H}_t ]
\end{align*}
The last line uses the almost sure bound $w_t^{(i)} \leq a^2 /N$ which follows from \eqref{eq:gq_bounds_sr} along with the form of the weights in Algorithm \ref{alg:SMC}.
Note that some terms in the above expressions may be equal to zero when $w_t^{(i)}$ is small enough, but the bound always holds nonetheless.
Since the above holds for all $i$, applying the tower rule we have
\begin{equation*}
\frac{1}{(N)_3} \sum_{i=1}^{N} \Et [(\nu_t^{(i)})_3 ]
\leq \frac{a^2}{N-2} \frac{1}{(N)_2} \sum_{i=1}^{N} \Et [(\nu_t^{(i)})_2 ]
\end{equation*}
satisfying \eqref{eq:mainthmcond} with $b_N := a^2/(N-2) \rightarrow 0$.
The result then follows by applying Theorem~\ref{thm:FDDconv}.
\end{proof}



\begin{prop}\label{thm:strat_nontriviality}
Consider an SMC algorithm using stratified resampling.
Suppose that 
\begin{equation*}
\varepsilon \leq q_t(x, x^\prime) \leq \varepsilon^{-1}
\end{equation*}
uniformly in $x,x^\prime$ for some $\varepsilon \in (0,1]$, and that there exist $\zeta >0$ and $\delta >0$ such that 
\begin{equation*}
\Prob[ \max_i w_t^{(i)} - \min_i w_t^{(i)} \geq 2\delta/N \mid \mathcal{F}_{t-1} ] \geq \zeta
\end{equation*}
 for infinitely many $t$. Then, for all $N>1$, $\Prob[ \tau_N(t) = \infty ] =0$ for all finite $t$.
\end{prop}

\draft{Comment that we now invoke the bounds on $q$, which were not needed to prove the rest of the corollary, and it may well be possible to prove the finite time scale condition without bounds on $q$.}

The extra condition that the transition densities are bounded above and away from zero \seb{... say something intelligent}.
The second condition is required to ensure that, at least infinitely often, the weights are not equal to $(1,\dots,1)/N$, since stochastic rounding is degenerate under equal weights and this will cause the time scale to explode. 
It is hardly conceivable that any real SMC algorithm would fail to satisfy this condition, which effectively ensures that the weights cannot be ``too well-behaved''.


\begin{proof}
As argued in Lemma~\ref{thm:mn_nontriviality}, it is sufficient to prove that under the stated conditions
\begin{equation*}
\sum_{r=0}^\infty \Prob[ c_N(r) > 2/N^2  \mid \mathcal{F}_{r-1} ] = \infty .
\end{equation*}
Firstly,
\begin{align}
\Prob[ c_N(t) \leq 2/N^2 \mid \mathcal{H}_t ]
&= \Prob[ c_N(t) =0 \mid \mathcal{H}_t ]
= \Prob[ \nu_t^{(i)} =1 \,\forall i\in\{1,\dots,N\} \mid \mathcal{H}_t ] \notag\\
&\leq \Prob[ \nu_t^{(i^\star)} =1 \mid \mathcal{H}_t ] , \label{eq:cNnonzero}
\end{align}
where $i^\star := \argmax_i \{ w_t^{(i)} \}$ (but note that the inequality holds when $i^\star$ is taken to be any particular index).
Define for each $k\in\{-1,0,1,2\}$
\begin{equation*}
p_k^{(i)} := \Prob \left[ \nu_t^{(i)} = \flnw + k \midd \mathcal{H}_t \right] .
\end{equation*}
Since stratified resampling almost surely results in $\nu_t^{(i)} \in \{ \flnw-1, \flnw, \flnw+1, \flnw+2 \}$ we have that
\begin{equation*}
\sum_{k=-1}^2 p_k^{(i)} 
= \sum_{k=-1}^2 \Prob \left[ \nu_t^{(i)} = \flnw + k \midd w_t^{(1:N)} \right]
= 1 .
\end{equation*}
Up to a proportionality constant $C$,
\begin{align*}
p_k^{(i)} 
&= C \, \Prob \left[ \nu_t^{(i)} = \flnw + k \midd w_t^{(1:N)} \right] \\
    &\qquad \times \sum_{\substack{a_{1:N} \in \{1,\dots,N\}^N : 
        \\ |\{j: a_j=i\}|=\flnw +k }}
        \Prob\left[ a_t^{(1:N)} = a_{1:N} \midd \nu_t^{(i)}, w_t^{(1:N)} \right]
        \prod_{k=1}^N q_{t-1}( X_t^{(a_k)}, X_{t-1}^{(k)} ) 
\end{align*}
for each $k\in\{-1,0,1,2\}$.
We can bound each probability above and below using the almost sure bounds on $q_{t-1}$ stated in the Lemma:
\begin{equation*}
C \, \Prob \left[ \nu_t^{(i)} = \flnw + k \midd w_t^{(1:N)} \right] \varepsilon^N
\leq p_k^{(i)}
\leq C \, \Prob \left[ \nu_t^{(i)} = \flnw + k \midd w_t^{(1:N)} \right] \varepsilon^{-N}
\end{equation*}
then eliminate the constant $C$ by normalising, to obtain lower bounds
\begin{align}
p_k^{(i)} 
&\geq \frac{ C \, \Prob [ \nu_t^{(i)} = \flnw + k \mid w_t^{(1:N)} ] \varepsilon^N }{
        \sum_{j=-1}^2 C \, \Prob [ \nu_t^{(i)} = \flnw + j \mid w_t^{(1:N)} ] 
        \varepsilon^{-N} } \notag\\
%&= \frac{ \Prob [ \nu_t^{(i)} = \flnw + k \mid w_t^{(1:N)} ] \varepsilon^N }{
%        \varepsilon^{-N} } \notag\\
&= \Prob [ \nu_t^{(i)} = \flnw + k \mid w_t^{(1:N)} ] \varepsilon^{2N} 
        . \label{eq:strat_pbounds}
\end{align}

Suppose that $\max_i w_t^{(i)} - \min_i w_t^{(i)} \geq 2\delta/N$. Then that at least one of $\{ \max_i w_t^{(i)} \geq (1+\delta)/N \}$ and $\{ \min_i w_t^{(i)} \leq (1-\delta)/N \}$ occurs. We will now examine each of these possibilities.

We can always write the maximum weight as $w_t^{(i^\star)} = \frac{1+\delta^\prime}{N}$ for some $\delta^\prime \geq 0$. Then, using \eqref{eq:cNnonzero},
\begin{equation*}
\Prob[ c_N(t) > 2/N^2 \mid \mathcal{H}_t ]
\geq 1- \Prob[ \nu_t^{(i^\star)} =1 \mid \mathcal{H}_t ]
= \begin{cases}
    1 - p_0^{(i^\star)} & \text{if } \delta^\prime \in [0,1) \\
    1 - p_{-1}^{(i^\star)} & \text{if } \delta^\prime \in [1,2) \\
    1 & \text{if } \delta^\prime \geq 3 .
\end{cases}
\end{equation*}
If $\delta^\prime \in [0,1)$ then
\begin{equation*}
1 - p_0^{(i^\star)}
= p_{-1}^{(i^\star)} + p_1^{(i^\star)} + p_2^{(i^\star)}
\geq \left( 0 + \frac{\delta^\prime}{2} + 0 \right) \varepsilon^{2N} 
= \frac{\delta^\prime \varepsilon^{2N} }{2}
\end{equation*}
using \eqref{eq:strat_pbounds} and the lower bounds in Table~\ref{tab:strat_probs}.
Similarly, if $\delta^\prime \in [1,2)$ then
\begin{equation*}
1 - p_{-1}^{(i^\star)}
= p_0^{(i^\star)} + p_1^{(i^\star)} + p_2^{(i^\star)}
\geq \left( \frac{(1-\delta^\prime)^2}{2} + \frac{\delta^\prime}{2} + 0 \right)
        \varepsilon^{2N} 
\geq \frac{\delta^\prime \varepsilon^{2N} }{2} .
\end{equation*}
So overall, under the constraint $\max_i w_t^{(i)} \geq (1+\delta)/N$, we have
\begin{equation*}
\Prob[ c_N(t) > 2/N^2 \mid \mathcal{H}_t ]
\geq \min_{\delta^\prime \geq \delta} 
        \left\{ \frac{\delta^\prime \varepsilon^{2N} }{2} \right\}
= \frac{ \delta \varepsilon^{2N} }{2} .
\end{equation*}

Now for the other case. Let $j^\star := \argmin_i \{ w_t^{(i)} \}$ and write
$w_t^{(j^\star)} = \frac{1-\delta^\prime}{N}$, for some $\delta^\prime \in [0,1]$.
Then we have
\begin{equation*}
\Prob[ c_N(t) > 2/N^2 \mid \mathcal{H}_t ]
\geq 1- \Prob[ \nu_t^{(i^\star)} =1 \mid \mathcal{H}_t ]
=\begin{cases}
    1- p_1^{(j^\star)} & \text{if } \delta^\prime \in (0,1] \\
    0 & \text{if } \delta^\prime =0 .
\end{cases}
\end{equation*}
If $\delta^\prime \in (0,1]$ then
\begin{equation*}
1- p_1^{(j^\star)} = p_{-1}^{(j^\star)} + p_0^{(j^\star)} + p_2^{(j^\star)}
\geq \left( 0 + \frac{ (\delta^\prime)^2 }{2} + 0 \right) \varepsilon^{2N}
= \frac{ (\delta^\prime)^2 \varepsilon^{2N} }{2} ,
\end{equation*}
again using the lower bounds in Table~\ref{tab:strat_probs}.
Therefore, under the constraint $\min_i w_t^{(i)} \leq (1-\delta)/N$, we have
\begin{equation*}
\Prob[ c_N(t) > 2/N^2 \mid \mathcal{H}_t ]
\geq \min_{\delta^\prime \geq \delta} 
        \left\{ \frac{ (\delta^\prime)^2 \varepsilon^{2N} }{2} 
        \I{\delta^\prime \neq 0} \right\}
= \frac{ \delta^2 \varepsilon^{2N} }{2} .
\end{equation*}
Combining both cases, we find for arbitrary $r$
\begin{equation*}
\Prob[ c_N(r) > 2/N^2 \mid \mathcal{H}_r ] 
\geq  \frac{ \delta^2 \varepsilon^{2N} }{2}
        \I{ \max_i w_r^{(i)} - \min_i w_r^{(i)} \geq 2\delta/N }
\end{equation*}
so
\begin{align*}
\Prob[ c_N(r) > 2/N^2 \mid \mathcal{F}_{r-1} ] 
&\geq  \frac{ \delta^2 \varepsilon^{2N} }{2}
        \Prob[ \max_i w_r^{(i)} - \min_i w_r^{(i)} \geq 2\delta/N 
        \mid \mathcal{F}_{r-1} ] \\
&\geq \zeta \frac{ \delta^2 \varepsilon^{2N} }{2}
> 0
\end{align*}
for infinitely many $r$.
Hence
\begin{equation*}
\sum_{r=0}^\infty \Prob[ c_N(r) > 2/N^2  \mid \mathcal{F}_{r-1} ] = \infty
\end{equation*}
as required.
\end{proof}



\section{Stochastic rounding \seb{$\checkmark$} }

\begin{corollary}\label{thm:stochrounding}
Consider an SMC algorithm using any stochastic rounding as its resampling scheme, such that \ref{standing_assumption} is satisfied.
Assume that there exists a constant $a\in [1,\infty)$ such that for all $x, x^\prime, t$,
\begin{equation*}
\frac{1}{a} \leq g_t(x, x^\prime) \leq a . 
\end{equation*}
Assume that $\Prob[ \tau_N(t) = \infty] =0$ for all finite $t$.
Let $(G_t^{(n,N)})_{t\geq0}$ denote the genealogy of a random sample of $n$ terminal particles from the output of the algorithm when the total number of particles used is $N$. Then, for any fixed $n$, the time-scaled genealogy $(G_{\tau_N(t)}^{(n,N)})_{t\geq0}$ converges to Kingman's $n$-coalescent as $N\to \infty$, in the sense of finite-dimensional distributions.
\end{corollary}

\begin{proof}
We can apply exactly the proof of Corollary~\ref{thm:stratified}, except that stochastic rounding is more restrictive than stratified resampling, so that the only possible offspring counts (almost surely, conditional on weights) are $\nu_t^{(i)} \in \{ \flnw, \flnw +1 \}$. We simply set $p_{-1}^{(i)} = p_{2}^{(i)} = 0$ in the proof of Corollary~\ref{thm:stratified} to see that
\begin{equation*}
\frac{1}{(N)_3} \sum_{i=1}^{N} \Et [(\nu_t^{(i)})_3 ]
\leq \frac{a^2}{N-2} \frac{1}{(N)_2} \sum_{i=1}^{N} \Et [(\nu_t^{(i)})_2 ]
\end{equation*}
as required.
The result then follows by applying Theorem~\ref{thm:FDDconv}.
\end{proof}


We can also show, under additional conditions, that the assumption $\Prob[ \tau_N(t) = \infty ] =0$ for all finite $t$ holds.

\begin{prop}\label{thm:SR_nontriviality}
Consider an SMC algorithm using any stochastic rounding as its resampling scheme.
Suppose that 
\begin{equation*}
\varepsilon \leq q_t(x, x^\prime) \leq \varepsilon^{-1}
\end{equation*}
uniformly in $x,x^\prime$ for some $\varepsilon \in (0,1]$, and that there exist $\zeta >0$ and $\delta >0$ such that 
\begin{equation*}
\Prob[ \max_i w_t^{(i)} - \min_i w_t^{(i)} \geq 2\delta/N \mid \mathcal{F}_{t-1} ] \geq \zeta
\end{equation*}
 for infinitely many $t$. Then, for all $N>1$, $\Prob[ \tau_N(t) = \infty ] =0$ for all finite $t$.
\end{prop}

\seb{If I find a more elegant proof for e.g. stratified, I can use the techniques to simplify this one a bit too.}

\begin{proof}
Let $\mathcal{H}_t$ be defined as in \eqref{eq:defn_Ht}. The first step is to show that whenever $\max_i w_t^{(i)} \geq (1+\delta)/N$, $\Prob[  c_N(t) > 2/N^2 \mid \mathcal{H}_t ] = \Prob[ c_N(t) \neq 0 \mid \mathcal{H}_t ]$ is bounded below uniformly in $t$.
For this purpose we need consider only weight vectors such that $w_t^{(i)} \in (0,2/N)$ for all $i$; otherwise $\Prob[ c_N(t) \neq 0 \mid \mathcal{H}_t ] =1$ by the definition of stochastic rounding.

Denote $\mathcal{S}_{N-1}^\delta = \{ w^{(1:N)} \in \mathcal{S}_{N-1} :  \forall i, \, 0 <w^{(i)} <2/N ;\, \max_i w^{(i)} \geq (1 + \delta)/N \}$ for any $\delta \in (0, 1)$, where $\mathcal{S}_{k}$ denotes the $k$-dimensional probability simplex.
Fix arbitrary $w_t^{(1:N)} \in \mathcal{S}_{N-1}^\delta$. Set $i^\star = \arg\max_i w_t^{(i)}$ and denote $\mathcal{I} = \{i \in \{1,\dots,N\} : w^{(i)} > 1/N \}$.
Since all weights are in $(0, 2/N)$, for $i \in \mathcal{I}, \nu_t^{(i)} \in \{1,2\}$ and for $i \notin \mathcal{I}, \nu_t^{(i)} \in \{0,1\}$; and since the offspring counts must sum to $N$, we can write
\begin{align}\label{eq:smallcN_istar}
\Prob[ c_N(t) \leq 2/N^2 \mid \mathcal{H}_t ]
&= \Prob[ \nu_t^{(i)} =1 \,\forall i\in\{1,\dots,N\} \mid \mathcal{H}_t ] \notag\\
&= \Prob[ \nu_t^{(i)} =1 \,\forall i\in \mathcal{I} \mid \mathcal{H}_t ] \notag\\
&= \prod_{i \in \mathcal{I}} \Prob[ \nu_t^{(i)} =1 \mid \nu_t^{(j)}=1 \,\forall j \in \mathcal{I}: j<i; \mathcal{H}_t ] \notag\\
&= \Prob[ \nu_t^{(i^\star)} =1 \mid \mathcal{H}_t ] \prod_{\substack{i \in \mathcal{I} \\ i \neq i^\star}} \Prob[ \nu_t^{(i)} =1 \mid \nu_t^{(i^\star)}=1; \nu_t^{(j)}=1 \,\forall j \in \mathcal{I}: j<i ; \mathcal{H}_t ] \notag\\
&\leq \Prob[ \nu_t^{(i^\star)} =1 \mid \mathcal{H}_t ] .
\end{align}
The final inequality holds with equality when $|\mathcal{I}| =1$, i.e.\ the only weight larger than $1/N$ is $w_t^{(i^\star)}$.
Thus $\Prob[ c_N(t) > 2/N^2 \mid \mathcal{H}_t ]$ is minimised on $\mathcal{S}_{N-1}^\delta$ when only one weight is larger than $1/N$, in which case the values of the other weights do not affect this probability. 

Define $w_{\delta^\prime} = \{(1,\dots,1) + \delta^\prime e_{i^\star} - \delta^\prime e_{j^\star} \} /N$ for fixed $i^\star \neq j^\star$ and $\delta^\prime \in (0,1)$, where $e_i$ denotes the $i$th canonical basis vector in $\mathbb{R}^N$. 
As in the proof of Corollary~\ref{thm:stochrounding}, define $p_0^{(i)} = \Prob[ \nu_t^{(i)} = \flnw \mid \mathcal{H}_t ]$ and $p_1^{(i)} = \Prob[ \nu_t^{(i)} = \flnw +1 \mid \mathcal{H}_t ]$. Then from \eqref{eq:smallcN_istar} we have
\begin{equation*}
\Prob[ c_N(t) > 2/N^2 \mid \mathcal{H}_t, w_t^{(1:N)} = w_{\delta^\prime} ]
= 1- \Prob[ \nu_t^{(i^\star)} = 1 \mid \mathcal{H}_t, w_t^{(1:N)} = w_{\delta^\prime} ]
= p_1^{(i^\star)},
\end{equation*}
evaluated on $w_{\delta^\prime}$.
We will need a lower bound on $p_1^{(i^\star)}$ when $w_t^{(1:N)} = w_{\delta^\prime}$. 
We first derive expressions for $p_0^{(i)}$ and $p_1^{(i)}$ up to a constant, then use $p_0^{(i)} + p_1^{(i)} =1$ to get a normalised bound. We have
\begin{align*} 
p_0^{(i)} &= C (1- N w_t^{(i)} + \flnw) \\
&\qquad \times \sum_{\substack{a_{1:N} \in \{1,\dots,N\}^N : \\ |\{j: a_j=i\}|=\flnw }}
\Prob\left[ a_t^{(1:N)} = a_{1:N} \mid \nu_t^{(i)}, w_t^{(1:N)} \right]
\prod_{k=1}^N q_{t-1}( X_t^{(a_k)}, X_{t-1}^{(k)} ) ,\\
p_1^{(i)} &= C (N w_t^{(i)} - \flnw) \\
&\qquad \times \sum_{\substack{a_{1:N} \in \{1,\dots,N\}^N : \\ |\{j: a_j=i\}|=\flnw +1 }}
\Prob\left[ a_t^{(1:N)} = a_{1:N} \mid \nu_t^{(i)}, w_t^{(1:N)} \right]
\prod_{k=1}^N q_{t-1}( X_t^{(a_k)}, X_{t-1}^{(k)} ) .
\end{align*}
Applying the bounds on $q_t$, we have
\begin{align*}
C (1- N w_t^{(i)} + \flnw) \varepsilon^N &\leq p_0^{(i)} \leq C (1- N w_t^{(i)} + \flnw) \varepsilon^{-N} ,\\
C (N w_t^{(i)} - \flnw) \varepsilon^N &\leq p_1^{(i)} \leq C (N w_t^{(i)} - \flnw) \varepsilon^{-N}
\end{align*}
from which we construct the normalised bound
\begin{equation*}
p_1^{(i)} \geq \frac{ (Nw_t^{(i)} - \flnw) \varepsilon^{N} }{ (Nw_t^{(i)} - \flnw) \varepsilon^{-N} + (1- Nw_t^{(i)} +\flnw) \varepsilon^{-N}}
= (Nw_t^{(i)} - \flnw) \varepsilon^{2N} .
\end{equation*}
When $w_t^{(1:N)} = w_{\delta^\prime}$, we have $w_t^{(i^\star)} = (1+\delta^\prime)/N$, so $p_1^{(i^\star)} \geq \delta^\prime \varepsilon^{2N}$,
which is increasing in $\delta^\prime$.
We conclude that $\Prob[ c_N(t) > 2/N^2 | \mathcal{H}_t, \max_i w_t^{(i)} \geq (1+\delta)/N ] \geq \min_{\delta^\prime \geq \delta} \delta^\prime \varepsilon^{2N} = \delta \varepsilon^{2N}$.

A slight modification of this argument yields $\Prob[ c_N(t) > 2/N^2 | \mathcal{H}_t, \min_i w_t^{(i)} \leq (1-\delta)/N ] \geq \delta \varepsilon^{2N} $.
Whenever $\max_i w_t^{(i)} - \min_i w_t^{(i)} \geq 2\delta/N$, either $\max_i w_t^{(i)} \geq (1+\delta)/N$ or $\min_i w_t^{(i)} \leq (1-\delta)/N$, so we have 
$\Prob[ c_N(t) > 2/N^2 | \mathcal{H}_t, \max_i w_t^{(i)} - \min_i w_t^{(i)} \geq 2\delta/N ] \geq \delta \varepsilon^{2N}$.
Thus 
\begin{equation*}
\Prob[ c_N(t)>2/N^2 \mid \mathcal{H}_t ] \geq \delta \varepsilon^{2N}\1{\max_i w_t^{(i)} - \min_i w_t^{(i)} \geq 2\delta/N} .
\end{equation*}
By a modification of \eqref{eq:condexp_HtFt} we have
\begin{align*}
\Prob[ c_N(t)>2/N^2 \mid \mathcal{F}_{t-1} ]
& %=\Et\left[ \Prob[ c_N(t)>2/N^2 \mid \mathcal{H}_t, \mathcal{F}_{t-1} ] \right]
=\Et\left[ \Prob[ c_N(t)>2/N^2 \mid \mathcal{H}_t ] \right] \\
&\geq \delta \varepsilon^{2N} \Prob[ \max_i w_t^{(i)} - \min_i w_t^{(i)} \geq 2\delta/N \mid \mathcal{F}_{t-1} ] ,
\end{align*}
which is bounded below by $ \zeta \delta \varepsilon^{2N} $ for infinitely many $t$. 
Hence,
\begin{equation*}
\sum_{t=0}^\infty \Prob[ c_N(t) > 2/N^2 \mid \mathcal{F}_{t-1} ] = \infty .
\end{equation*}
As in Lemma~\ref{thm:mn_nontriviality} we conclude by applying a filtered Borel--Cantelli lemma.
\end{proof}





\section{Residual resampling with stratified residuals \seb{$\sim$} }

\begin{corollary}\label{thm:residual_stratified}
Consider an SMC algorithm using residual resampling with stratified residuals, such that \ref{standing_assumption} is satisfied.
Assume that there exists a constant $a\in [1,\infty)$ such that for all $x, x^\prime, t$,
\begin{equation*}
\frac{1}{a} \leq g_t(x, x^\prime) \leq a .
\end{equation*}
Assume that $\Prob[ \tau_N(t) = \infty] =0$ for all finite $t$.
Let $(G_t^{(n,N)})_{t\geq0}$ denote the genealogy of a random sample of $n$ terminal particles from the output of the algorithm when the total number of particles used is $N$. Then, for any fixed $n$, the time-scaled genealogy $(G_{\tau_N(t)}^{(n,N)})_{t\geq0}$ converges to Kingman's $n$-coalescent as $N\to \infty$, in the sense of finite-dimensional distributions.
\end{corollary}

\begin{proof}
We can apply exactly the proof of Corollary~\ref{thm:stratified}, except that residual-stratified resampling is more restrictive than stratified resampling, so that the only possible offspring counts (almost surely, conditional on weights) are $\nu_t^{(i)} \in \{ \flnw, \flnw +1, \flnw +2 \}$. We simply set $p_{-1}^{(i)} = 0$ in the proof of Corollary~\ref{thm:stratified} to see that
\begin{equation*}
\frac{1}{(N)_3} \sum_{i=1}^{N} \Et [(\nu_t^{(i)})_3 ]
\leq \frac{a^2}{N-2} \frac{1}{(N)_2} \sum_{i=1}^{N} \Et [(\nu_t^{(i)})_2 ]
\end{equation*}
as required.
The result then follows by applying Theorem~\ref{thm:FDDconv}.
\end{proof}

\draft{State and prove finite time-scale lemma.}





\section{Residual resampling with multinomial residuals \seb{$\sim$} }
\draft{If I manage to prove this corollary, it would make this chapter satisfyingly complete :-)}

\begin{corollary}\label{thm:residual_multinomial}
Consider an SMC algorithm using residual resampling with multinomial residuals, such that \ref{standing_assumption} is satisfied.
Assume that there exists a constant $a\in [1,\infty)$ such that for all $x, x^\prime, t$,
\begin{equation*}
\frac{1}{a} \leq g_t(x, x^\prime) \leq a .
\end{equation*}
Assume that $\Prob[ \tau_N(t) = \infty] =0$ for all finite $t$.
Let $(G_t^{(n,N)})_{t\geq0}$ denote the genealogy of a random sample of $n$ terminal particles from the output of the algorithm when the total number of particles used is $N$. Then, for any fixed $n$, the time-scaled genealogy $(G_{\tau_N(t)}^{(n,N)})_{t\geq0}$ converges to Kingman's $n$-coalescent as $N\to \infty$, in the sense of finite-dimensional distributions.
\end{corollary}

\begin{proof}
With residual-multinomial resampling, for each $i$
\begin{equation*}
\nu_t^{(i)} \mid w_t^{(1:N)}
\eqdist \flnw + X_i
\end{equation*}
where $X_i \sim \Bin(R, r_i)$. As usual, $R := N - \sum_{i=1}^N \flnw$ and $r_i := ( Nw_t^{(i)} - \flnw ) /R$.
\seb{If $R=0$ then $r_i = 0$ for all $i$ and the following calculations remain correct.}
We can therefore compute
\begin{align*}
\E [ (\nu_t^{(i)})_2 \mid w_t^{(1:N)} ]
&= \E\left[ (\flnw + X_i) (\flnw + X_i -1) \midd w_t^{(1:N)} \right] \\
&= (\flnw)_2 + 2\flnw E[ X_i \mid w_t^{(1:N)} ] + \E[ (X_i)_2 \mid w_t^{(1:N)} ] \\
&= (\flnw)_2 + 2\flnw R r_i + (R)_2 r_i^2
\end{align*}
using the moments of the Binomial distribution.
We also have
\begin{align*}
\E [ (\nu_t^{(i)})_3 \mid w_t^{(1:N)} ]
&= \E\left[ (\flnw + X_i) (\flnw + X_i -1) (\flnw + X_i -2) \midd w_t^{(1:N)} \right] \\
&= \flnw^3 + \flnw^2 \E[ 3X_i -3 \mid w_t^{(1:N)} ] \\
    &\hspace{1cm}+ \flnw 
        \E[ X_i(X_i-1) + X_i(X_i-2) + (X_i-1)(X_i-2) \mid w_t^{(1:N)} ] \\
    &\hspace{1cm}+ \E[ (X_i)_3 \mid w_t^{(1:N)} ] \\
&= \flnw^3 - 3\flnw^2 +3\flnw^2 \E[ X_i \mid w_t^{(1:N)} ] \\
    &\hspace{1cm}+ \flnw \E[ 3X_i^2 - 6X_i +2 \mid w_t^{(1:N)} ] 
        + E[(X_i)_3 \mid w_t^{(1:N)} ]\\
&= \left( \flnw^3 - 3\flnw^2 + 2\flnw \right)
        + 3 \left( \flnw^2 - \flnw \right) \E[ X_i \mid w_t^{(1:N)} ] \\
    &\hspace{1cm}+ 3 \flnw \E[ (X_i)_2 \mid w_t^{(1:N)} ] 
        + \E[ (X_i)_3 \mid w_t^{(1:N)} ] \\
&= (\flnw)_3 + 3(\flnw)_2 R r_i + 3\flnw (R)_2 r_i^2 + (R)_3 r_i^3 \\
&\leq \left( \flnw + R r_i \right) \left\{ (\flnw)_2 + 2\flnw R r_i 
        + (R)_2 r_i^2 \right\} \\
&= Nw_t^{(i)} \E[(\nu_t^{(i)})_2 \mid w_t^{(1:N)} ] \\
&\leq a^2 \E[(\nu_t^{(i)})_2 \mid w_t^{(1:N)} ] ,
\end{align*}
using the almost sure bound $w_t^{(i)} \leq a^2/N$.

...

\seb{To complete the proof we need to exchange the conditioning on $w_t^{(1:N)}$ for conditioning on $\mathcal{H}_t$ so we can then invoke the D-separation and tower property to get:}
\begin{equation*}
\frac{1}{(N)_3} \sum_{i=1}^N \Et [ (\nu_t^{(i)})_3 ]
\leq \frac{a^2}{N-2} \frac{1}{(N)_2} \sum_{i=1}^N \Et [ (\nu_t^{(i)})_2 ] .
\end{equation*}
Thus \eqref{eq:mainthmcond} is satisfied with $b_N = a^2/(N-2)$.
\seb{The $\varepsilon$ might want to get involved here as well once we switch the conditioning, coming from bounds on $q_t$ (which would then have to be included in the statement of this corollary).}

\end{proof}

\draft{State and prove finite time-scale lemma.}




\section{Star resampling \seb{$\checkmark$} }
One might ask the question: is it possible to construct an SMC algorithm whose genealogies converge to some non-trivial limit other than the $n$-coalescent?
The answer is yes, as we now illustrate.

Recall that star resampling assigns all of the offspring to a single parent which is sampled from the Categorical distribution parametrised by $w_t^{(1:N)}$.
It is easy enough to show that such a resampling scheme does not satisfy \eqref{eq:mainthmcond}.
The vector of offspring counts is at every generation some permutation of $(N,0,\dots,0)$, and hence we calculate
\begin{align*}
\frac{1}{(N)_2} \sum_{i=1}^N \E[ (\nu_t^{(i)})_2 \mid \mathcal{H}_t ]
&= \frac{1}{(N)_2} (N)_2 = 1 , \\
\frac{1}{(N)_3} \sum_{i=1}^N \E[ (\nu_t^{(i)})_3 \mid \mathcal{H}_t ]
&= \frac{1}{(N)_3} (N)_3 = 1 ,
\end{align*}
so no suitable sequence $b_N$ can be found.
Now we know that Theorem~\ref{thm:FDDconv} does not apply, but this is not enough because condition \eqref{eq:mainthmcond} was not proved to be necessary.
However, we know exactly what the genealogy of $n$ particles from this SMC algorithm looks like (Figure~\ref{fig:star_genealogy}).
\begin{figure}[ht]
\centering
\begin{tikzpicture}
\draw (0,0)--(6,0);
\draw (3,0)--(3,3.5);
\draw (0,0)--(0,-0.5);
\draw (0.5,0)--(0.5,-0.5);
\draw (1,0)--(1,-0.5);
\draw (5.5,0)--(5.5,-0.5);
\draw (6,0)--(6,-0.5);
\node at (3.25,-0.5) {$\cdots$};
\end{tikzpicture}
\caption{Sample genealogy induced by star resampling}
\label{fig:star_genealogy}
\end{figure}
Whatever time scale is used, we cannot get away from the fact that this genealogy involves multiple mergers; it cannot converge to the $n$-coalescent.

The limiting genealogy is more like a \emph{star coalescent}\seb{[citation(s)]}. This is the coalescent process comprising an $\Exp(1)$-distributed event time at which all of the lineages merge into one.
In the case of star resampling we have $c_N(t) \equiv 1$, so the time-scaling function $\tau_N(t)$ defined in \eqref{eq:defn_tauN} converges pointwise to the identity function $\tau(t) \equiv t$ as $N\to\infty$, and this does not yield a continuous-time limit.
Under any time scale that results in a continuous-time limiting process, the coalescent event time converges to $0$, rather than the usual $\Exp(1)$-distributed random variable.






\section{Conditional SMC \seb{$\checkmark$} }
In conditional SMC, one ``immortal'' particle is treated differently to the others when it comes to assigning offspring to parents. This immortal particle is guaranteed at least one offspring, and has on average one more offspring than each of the other parents in each generation.
This results in genealogies that are qualitatively different to those of a corresponding standard SMC algorithm. For one thing, the population MRCA is \emph{guaranteed} to be an immortal particle; there is a sense in which the immortal lineage \emph{attracts} coalescence events.

Given this, we should not have been surprised if conditional SMC genealogies converged to a quite different coalescent process, perhaps a \emph{structured coalescent}\seb{[citation(s)]}.
As it turns out, we still recover Kingman's $n$-coalescent in the large population limit (Theorem~\ref{thm:CSMC}). 
A possible explanation for this is that, as $N\to\infty$, the probability of a given sample of size $n$ interacting with the immortal lineage (at least early on) vanishes, leaving a process that looks very much like the one induced by the corresponding standard SMC algorithm.

\begin{corollary}\label{thm:CSMC}
Consider a conditional SMC algorithm using multinomial resampling, such that \ref{standing_assumption} is satisfied. Assume there exist constants $\varepsilon\in (0,1], a\in [1,\infty)$ and probability density $h$ such that for all $x, x^\prime, t$,
\begin{equation}\label{eq:gq_bounds_csmc}
\frac{1}{a} \leq g_t(x, x^\prime) \leq a , \quad
\varepsilon h(x^\prime) \leq q_t(x, x^\prime) \leq \frac{1}{\varepsilon} h(x^\prime) .
\end{equation}
Let $(G_t^{(n,N)})_{t\geq0}$ denote the genealogy of a random sample of $n$ terminal particles from the output of the algorithm when the total number of particles used is $N$. Then, for any fixed $n$, the time-scaled genealogy $(G_{\tau_N(t)}^{(n,N)})_{t\geq0}$ converges to Kingman's $n$-coalescent as $N\to \infty$, in the sense of finite-dimensional distributions.
\end{corollary}
We restrict here to the case of multinomial resampling, which seems to be the most commonly-used resampling scheme within conditional SMC. Implementing other resampling schemes while maintaining the immortal lineage is more involved, though by no means impossible.\seb{cite that unpublished paper by AMJ and Arnaud?}
We conjecture that similar results hold for conditional SMC with other resampling schemes, as in the preceding corollaries.

\draft{As one might expect, the conditions here (i.e. bounds on $g$, $q$) are the same as for multinomial resampling...}

\begin{proof}
Assume, without loss of generality, that the immortal particle takes index 1 in each generation. This significantly lightens the notation, but the same argument holds if the immortal indices are taken to be $a_{(0:T)}^\star$ rather than $(1,\dots,1)$.

The parental indices are conditionally independent, as in standard SMC with multinomial resampling, but we have to treat $i=1$ as a special case. The conditional law on the $i^{th}$ parental index is
\begin{equation*}
\Prob \left[ a_t^{(i)} = a_i \mid \mathcal{H}_t \right] \propto
\begin{cases}
\1{a_i=1} &i=1 \\
g_t( X_{t+1}^{a_t^{(a_i)}} , X_t^{(a_i)} ) q_{t-1}(X_t^{(a_i)}, X_{t-1}^{(i)}) &i=2,\dots,N ,
%w_t^{(a_i)} q_{t-1}(X_t^{(a_i)}, X_{t-1}^{(i)}) &i=2,\dots,N ,
\end{cases}
\end{equation*}
resulting in the joint law
\begin{equation*}
\Prob \left[ a_t^{(1:N)} = a_{1:N} \mid \mathcal{H}_t \right] 
\propto \1{a_1 = 1} \prod_{i=2}^N g_t( X_{t+1}^{a_t^{(a_i)}} , X_t^{(a_i)} )  
        q_{t-1}(X_t^{(a_i)}, X_{t-1}^{(i)}).
%\propto \1{a_1 = 1} \prod_{i=2}^N w_t^{(a_i)} q_{t-1}(X_t^{(a_i)}, X_{t-1}^{(i)}).
\end{equation*}
%%%
As in Corollary~\ref{thm:multinomial}, under \eqref{eq:gq_bounds_csmc} we have bounds
\begin{equation*}
\E[ (V_1^{(i)})_k ]
\leq \E[ (\nu_t^{(i)})_k \mid \mathcal{H}_t ]
\leq \E[ (V_2^{(i)})_k ] ,
\end{equation*}
where now
\begin{align*}
& V_1^{(i)} 
    \eqdist \1{i=1} + \operatorname{Binomial}\left(N-1, 
        \frac{\varepsilon/a}{(\varepsilon/a) + (N-1)(a/\varepsilon)} \right) , \\
& V_2^{(i)} 
    \eqdist \1{i=1} + \operatorname{Binomial}\left( N-1, 
        \frac{a/\varepsilon}{(a/\varepsilon) + (N-1)(\varepsilon/a)} \right) .
\end{align*}
independently for each $i$ and independently of $\mathcal{F}_\infty$.
Furthermore, using the Binomial moments and the identity $(X+1)_2 \equiv 2(X)_1 +(X)_2$, one can show that
\begin{equation*}
\E[ (V_1^{(i)} )_2 ]
\geq \begin{cases}
\frac{(N-1)_2}{N^2}\frac{\varepsilon^4}{a^4} 
        + \frac{2(N-1)}{N} \frac{\varepsilon^2}{a^2} &\text{if } i=1 \\
\frac{(N-1)_2}{N^2}\frac{\varepsilon^4}{a^4} &\text{if } i\neq 1 .
\end{cases}
\end{equation*}
Using the identity $(X+1)_3 \equiv 3(X)_2 +(X)_3$, we also have
\begin{equation*}
\E[ (V_2^{(i)} )_3 ]
\leq \begin{cases}
\frac{(N-1)_3}{N^3}\frac{a^6}{\varepsilon^6} 
        + \frac{3(N-1)_2}{N^2} \frac{a^4}{\varepsilon^4} &\text{if } i=1 \\
\frac{(N-1)_3}{N^3}\frac{a^6}{\varepsilon^6}  &\text{if } i\neq 1 .
\end{cases}
\end{equation*}
We therefore have
\begin{align}
\frac{1}{(N)_2} \sum_{i=1}^N \E[ (\nu_t^{(i)})_2 \mid \mathcal{H}_t ]
&\geq \frac{1}{(N)_2} \sum_{i=1}^N \E[ ( V_1^{(i)} )_2 ]
\geq \frac{1}{(N)_2} \left[ \frac{2(N-1)}{N} \frac{\varepsilon^2}{a^2}
        + \sum_{i=1}^N \frac{(N-1)_2}{N^2}\frac{\varepsilon^4}{a^4} \right] \notag\\
&= \frac{1}{N^2} \left[ 2 \frac{\varepsilon^2}{a^2} 
        + (N-2) \frac{\varepsilon^4}{a^4} \right]
   \geq \frac{\varepsilon^4}{Na^4} \label{eq:csmc_cN_LB}
\end{align}
and
\begin{align*}
\frac{1}{(N)_3} \sum_{i=1}^N \E[ (\nu_t^{(i)})_3 \mid \mathcal{H}_t ]
&\leq \frac{1}{(N)_3} \sum_{i=1}^N \E[ ( V_2^{(i)} )_3 ]
\leq \frac{1}{(N)_3} \left[ \frac{3(N-1)_2}{N^2} \frac{a^4}{\varepsilon^4}
        + \sum_{i=1}^N \frac{(N-1)_3}{N^3}\frac{a^6}{\varepsilon^6}  \right] \\
&= \frac{1}{N^3} \left[ 3 \frac{a^4}{\varepsilon^4} 
        + (N-3) \frac{a^6}{\varepsilon^6} \right]
    \leq \frac{a^6}{N^2 \varepsilon^6} .
\end{align*}
Hence, applying \eqref{eq:condexp_HtFt}, we can upper bound the ratio
\begin{equation*}
\frac{\frac{1}{(N)_3} \sum_{i=1}^N \Et[ (\nu_t^{(i)})_3 ]}{\frac{1}{(N)_2} 
        \sum_{i=1}^N \Et[ (\nu_t^{(i)})_2 ]}
\leq \frac{a^{10}}{N\varepsilon^{10}}
=: b_N 
\underset{N\to\infty}{\longrightarrow} 0
\end{equation*}
so \eqref{eq:mainthmcond} is satisfied. 
Proof that the time scale is finite is relegated to Lemma \ref{thm:CSMC_nontriviality}, whence we conclude by applying Theorem~\ref{thm:FDDconv}.

%%%%

%As in standard SMC with multinomial resampling, under \eqref{eq:gq_bounds_csmc} we have
%\begin{equation}\label{eq:csmc_f_bound}
%\E[ f(A_{1,i}^{(1:N)}) ] 
%\leq \E[ f(a_t^{(1:N)}) \mid \mathcal{H}_t ]
%\leq \E[ f(A_{2,i}^{(1:N)}) ] 
%\end{equation}
%where now
%\begin{align*}
%& A_{1,i}^{(j)} \sim \begin{cases}
%\delta_1 \qquad & j=1 \\
%\operatorname{Categorical}\left( (\varepsilon/a)^{\1{i=1} -\1{i\neq 1}} ,\dots, (\varepsilon/a)^{\1{i=N} -\1{i\neq N}} \right) & j\neq 1 
%\end{cases} \\
%& A_{2,i}^{(j)} \sim \begin{cases}
%\delta_1 \qquad & j=1\\
%\operatorname{Categorical}\left( (a/\varepsilon)^{\1{i=1} -\1{i\neq 1}} ,\dots, (a/\varepsilon)^{\1{i=N} -\1{i\neq N}} \right) & j\neq 1
% \end{cases}
%\end{align*}
%independently for each $j$.
%The corresponding offspring counts have marginal distributions
%\begin{align*}
%& V_1^{(i)} \overset{d}{=} \1{i=1} + \operatorname{Binomial}\left(N-1, \frac{\varepsilon/a}{(\varepsilon/a) + (N-1)(a/\varepsilon)} \right) , \\
%& V_2^{(i)} \overset{d}{=} \1{i=1} + \operatorname{Binomial}\left( N-1, \frac{a/\varepsilon}{(a/\varepsilon) + (N-1)(\varepsilon/a)} \right) .
%\end{align*}
%Now consider the function $f_i(a_t^{(1:N)}) := (\nu_t^{(i)})_2$. We can apply \eqref{eq:csmc_f_bound} to obtain the lower bound
%\begin{align*}
%\frac{1}{(N)_2} \sum_{i=1}^N \E[ (\nu_t^{(i)})_2 \mid \mathcal{H}_t ]
%&\geq \frac{1}{(N)_2} \sum_{i=1}^N \E[ (V_1^{(i)})_2 ]
%=  \frac{1}{(N)_2} \left[ \E[ (V_1^{(1)})_2 ] + \sum_{i=2}^N \E[ (V_1^{(i)})_2 ] \right] \\
%&= \frac{1}{(N)_2} \Bigg[ \frac{(N-1)_2 (\varepsilon/a)^2}{\{(\varepsilon/a) + (N-1)(a/\varepsilon)\}^2} + \frac{2(N-1)(\varepsilon/a)}{(\varepsilon/a) + (N-1)(a/\varepsilon)}  \\
%&\qquad\qquad\qquad + \sum_{i=2}^N \frac{(N-1)_2 (\varepsilon/a)^2}{\{(\varepsilon/a) + (N-1)(a/\varepsilon)\}^2} \Bigg] \\
%&= \frac{1}{(N)_2} \left[ \frac{2(N-1)(\varepsilon/a)}{(\varepsilon/a) + (N-1)(a/\varepsilon)} + \sum_{i=1}^N \frac{(N-1)_2 (\varepsilon/a)^2}{\{(\varepsilon/a) + (N-1)(a/\varepsilon)\}^2} \right]
%\end{align*}
%using the identity $(X+1)_2 \equiv 2(X)_1 +(X)_2$.
%This is further bounded by
%\begin{align}
%\frac{1}{(N)_2} \sum_{i=1}^N \E[ (\nu_t^{(i)})_2 \mid \mathcal{H}_t ]
%&\geq \frac{1}{(N)_2} \left\{ \frac{2(N-1)(\varepsilon/a)}{N(a/\varepsilon)} + \frac{(N)_3 (\varepsilon/a)^2}{N^2(a/\varepsilon)^2} \right\} \notag\\
%&= \frac{1}{N^2} \left\{\frac{2\varepsilon^2}{a^2} + \frac{(N-2)\varepsilon^4}{a^4}  \right\} . \label{eq:CSMC_cN_LB}
%\end{align}
%Similarly, we derive an upper bound on $f_i(a_t^{(1:N)}) := (\nu_t^{(i)})_3$, this time using the identity $(X+1)_3 \equiv 3(X)_2 +(X)_3 $:
%\begin{align*}
%\frac{1}{(N)_3} \sum_{i=1}^N \E[ (\nu_t^{(i)})_3 \mid \mathcal{H}_t]
%&\leq \frac{1}{(N)_3} \left[ \E[ (V_2^{(1)})_3 ] + \sum_{i=2}^N \E[ (V_2^{(i)})_3 ] \right] \\
%&\leq \frac{1}{(N)_3} \left[ \frac{ 3 (N-1)_2 (a/\varepsilon)^2}{\{(a/\varepsilon) + (N-1)(\varepsilon/a)\}^2} + \sum_{i=1}^N \frac{(N-1)_3 (a/\varepsilon)^3}{\{(a/\varepsilon) + (N-1)(\varepsilon/a)\}^3} \right] \\
%&\leq \frac{1}{(N)_3} \left\{ \frac{3(N-1)_2 (a/\varepsilon)^2}{N^2 (\varepsilon/a)^2} + \frac{(N)_4 (a/\varepsilon)^3}{N^3 (\varepsilon/a)^3} \right\} \\
%&= \frac{1}{(N)_3} \left\{ \frac{3(N-1)_2}{N^2} \frac{a^4}{\varepsilon^4} +\frac{(N)_4}{N^3} \frac{a^6}{\varepsilon^6} \right\} \\
%&= \frac{1}{N^3} \left\{ \frac{3a^4}{\varepsilon^4} + \frac{(N-3) a^6}{\varepsilon^6} \right\} .
%\end{align*}
%Applying \eqref{eq:condexp_HtFt}, we can therefore upper bound the ratio by
%\begin{align*}
%\frac{\frac{1}{(N)_3} \sum_{i=1}^N \Et[ (\nu_t^{(i)})_3 ]}{\frac{1}{(N)_2} \sum_{i=1}^N \Et[ (\nu_t^{(i)})_2 ]}
%&\leq \frac{N^2}{N^3} \frac{ \frac{3a^4}{\varepsilon^4} + \frac{(N-3)a^6}{\varepsilon^6} }{ \frac{2\varepsilon^2}{a^2} + \frac{(N-2)\varepsilon^4}{a^4} }
%\leq \frac{1}{N} \frac{a^6}{\varepsilon^6}\, \frac{3 + (N-3) a^2 / \varepsilon^2 }{2 + (N-2) \varepsilon^2 / a^2} \\
%&\leq \frac{1}{N} \frac{a^6}{\varepsilon^6} \left\{ \frac{3}{2} + \frac{N-3}{N-2} \frac{a^4}{\varepsilon^4} \right\}
%\leq \frac{1}{N} \left\{ \frac{3 a^6}{2 \varepsilon^6} + \frac{a^{10}}{\varepsilon^{10}} \right\}
%=: b_N \underset{N\to\infty}{\longrightarrow} 0
%\end{align*}
%so \eqref{eq:mainthmcond} is satisfied. 
%Proof that the time scale is finite is relegated to Lemma \ref{thm:CSMC_nontriviality}, whence we conclude by applying Theorem~\ref{thm:FDDconv}.
\end{proof}



\begin{lemma}\label{thm:CSMC_nontriviality}
Consider a conditional SMC algorithm using multinomial resampling, satisfying \ref{standing_assumption} and \eqref{eq:gq_bounds_csmc}. 
Then, for all $N>2$, $\Prob[ \tau_N(t) = \infty ]=0$ for all finite $t$.
\end{lemma}

\begin{proof}
The proof is exactly the same as Lemma~\ref{thm:mn_nontriviality}, since \eqref{eq:csmc_cN_LB} gives us exactly the same lower bound on $\Et [ c_N(t) ]$ that we had in standard SMC with multinomial resampling.
%%%
%Since $c_N(t) \in [0,1]$ almost surely and has strictly positive expectation, for any fixed $N$ the distribution of $c_N(t)$ with given expectation that maximises $\Prob[ c_N(t)=0 \mid \mathcal{F}_{t-1} ]$ is two atoms, at 0 and 1 respectively. To ensure the correct expectation, the atom at 1 should have mass $\Prob[ c_N(t)=1 \mid \mathcal{F}_{t-1} ] = \Et [ c_N(t) ]$, which is bounded below by \eqref{eq:CSMC_cN_LB}.
%If $c_N(t) > 0$ then $c_N(t) \geq 2/(N)_2 > 2/N^2$. Hence, in general $\Prob[ c_N(t) > 2/N^2 \mid \mathcal{F}_{t-1} ] \geq \Et [c_N(t)]$. Applying \eqref{eq:CSMC_cN_LB}, we have for any finite $N$,
%\begin{equation*}
%\sum_{t=0}^\infty \Prob[ c_N(t) > 2/N^2 \mid \mathcal{F}_{t-1} ]
%\geq \sum_{t=0}^\infty \Et [ c_N(t) ]
%\geq \sum_{t=0}^\infty \frac{1}{N^2} \left\{\frac{2\varepsilon^2}{a^2} + \frac{(N-2)\varepsilon^4}{a^4}  \right\}
%= \infty
%\end{equation*}
%By an argument analogous to the conclusion of Lemma \ref{thm:SR_nontriviality}, $\Prob[ \tau_N(t) = \infty ] =0$ for all $t < \infty$.
\end{proof}


\subsection{Effect of ancestor sampling}
Ancestor sampling breaks up the immortal lineage into sections, so it is not really a lineage, and we do not really have a pure coalescent process backwards in time. Regardless, we shall throw caution to the wind and examine the resulting ``genealogies''.

Using the parent sampling probabilities specified in \eqref{eq:ancsamp_probs},
now with time reversed and the notation made to fit \seb{preferably the presentation in Chapter 2 should use the notation we want to use here} with this study of genealogies, we obtain\seb{add one more step of working below to make it less ``magic''?}
\begin{equation*}
\Prob[a_t^{(i)} = a_i \mid \mathcal{H}_t ] \propto
\begin{cases}
w_t^{(a_i)} q_{t-1}(X_t^{(a_i)}, X_{t-1}^{(i)}) &\qquad i\in\text{non-immortal particles}\\
w_t^{(a_i)} q_{t-1}(X_t^{(a_i)}, X_{t-1}^*) &\qquad i=\text{immortal particle} .
\end{cases} 
\end{equation*}
But when $i$ is the index of the immortal particle, $X_{t-1}^{(i)} = X_{t-1}^*$, so the above simplifies to
\begin{equation*}
\Prob[a_t^{(i)} = a_i \mid \mathcal{H}_t ] \propto
w_t^{(a_i)} q_{t-1}(X_t^{(a_i)}, X_{t-1}^{(i)})
\end{equation*}
for each $i$, which is exactly \eqref{eq:parentslaw_mn}, the law on parental indices under standard SMC with multinomial resampling.
In other words, when parental indices are chosen, the immortal particle is treated exactly like all of the other particles; it has completely lost its ``reproductive advantage''.
This means it is no more likely for lineages to coalesce onto the ``immortal'' lineage than onto any other lineage, so we do not see the behaviour of Figure~\ref{fig:PG_ancdegen} which caused the particle Gibbs chain to mix slowly over the sequential component.
This supports the claim of Section~\ref{sec:ancsamp}: particle Gibbs with ancestor sampling still experiences ancestral degeneracy, but this no longer causes the sequential component to get stuck.

%This viewpoint gives us another intuition about the effect of ancestor sampling: the immortal particle is no longer specially favoured when it comes to producing offspring. In the basic algorithm, the immortal parent produces on average one more offspring at each generation than any other parent (and must produce at least one offspring). It follows that the immortal line cannot die out (is immortal), and also that it is a more likely site for coalescence events (i.e.\ producing at least two offspring) than any other parent. 
%With ancestor sampling, the immortal parent loses this advantage; it is treated just like the other parents. It is no longer a particularly favourable site for coalescing, so the trajectory onto which everyone coalesces might just as well be any of the other $N-1$ trajectories. This prevents the Markov chain getting stuck as in Figure \ref{fig:PG_ancdegen}.

